\section{L'accident vasculaire c\'er\'ebral isch\'emique}

%\subsection{Isch\'emie focale transitoire chez le rat}
%\subsection{IRM multiparam\'etrique}
%\setcounter{subsubsection}{-1}
%\subsubsection{Les donn\'ees brutes : $T_1$, $T_2$, $T^{\ast}_1$, $T^{\ast}_2$}
%\subsubsection{ADC}

%\subsection{Epid\'emiologie : l'accident vasculaire c\'er\'ebral (AVC)}

%Epidémiologie : cas général pour l'AVC.

\subsection{D\'efinition, \'epid\'emiologie}

Un accident vasculaire c\'er\'ebral est un arr\^et brutal de la circulation sanguine \g{a} l'int\'erieur du cerveau. %
110438 personnes \'etaient hospitalis\'ees pour cette raison en 2014 en France, soit pr\g{e}s de 160 hospitalisations sur 100 000 cette ann\'ee-l\g{a} \cite{epi_hem16}. %
Toujours d'apr\g{e}s \cite{epi_hem16}, l'AVC \'etait mortel dans 15744 cas ; %
52451 autres patients, soit 47,1\% des personnes hospitalis\'ees pour AVC, gardaient des s\'equelles de l'accident : %
Atteinte neurologique, motrice, aphasie \dots.

\par
Il existe deux sortes d'accidents vasculaires c\'er\'ebraux :
\begin{description}
\item[L'AVC h\'emorrhagique : ] c'est une cons\'equence de la rupture de la paroi d'un vaisseau sanguin \g{a} l'int\'erieur du cerveau, comme une rupture d'an\'evrisme ;
\item[L'AVC isch\'emique : ] c'est une baisse de la perfusion sanguine dans une r\'egion du cerveau, %
qui survient quand un caillot sanguin vient obstruer une art\g{e}re qui alimente cette r\'egion.
\end{description}

C'est ce deuxi\g{e}me type d'AVC qui nous int\'eresse, il touchait 78633 personnes hospitalis\'ees en 2014 en France \cite{epi_hem16}.

\par
Notons que dans ce cas, 52,6\% des personnes hospitalis\'ees ont gard\'e des s\'equelles de leur AVC.

%\subsubsection{Isch\'emie focale transitoire}

\subsection{D\'eroulement de l'AVC isch\'emique}

En premier lieu, remarquons que  que le cerveau est un organe particuli\g{e}rement sensible aux variations de son d\'ebit sanguin.

\par
En effet \cite{pat_neu}, il contient tr\g{e}s peu de r\'eserves \'energ\'etiques eu \'egard son importante consommation : %
le cerveau consomme \g{a} lui seul 20 \% du dioxyg\g{e}ne de l'organisme.

\par
Ainsi, ses r\'eserves en glucose et en glycog\g{e}ne peuvent r\'epondre \g{a} ses besoins m\'etaboliques pendant seulement trois minutes.

\par
Enfin, comme nous allons le voir dans la sous-section \ref{casc}, un manque de dioxyg\g{e}ne et de glucose se traduit tr\g{e}s rapidement par la lib\'eration, %
au sein du tissu c\'er\'ebral, de mol\'ecules cytotoxiques.

\etoile
La d\'ependance du cerveau aux apports de la circulation sanguine est donc tr\g{e}s importante. %
Deux m\'ecanismes de r\'egulation du d\'ebit sanguin c\'er\'ebral (CBF), puis du taux de dioxyg\g{e}ne m\'etabolis\'e (CMRO2), sont pr\'esent\'es sans la sous-section suivante.

\subsubsection{Baisse de la pression de perfusion sanguine}

C'est la premi\g{e}re cons\'equence de l'interruption de la circulation sanguine.

\par
La pression de perfusion sanguine $\Delta P$ est d\'efinie par : $\Delta P = P_a - P_v$, o\g{u} $P_a$ et $P_v$ sont respectivement les pressions sanguines art\'erielle et veineuse. %
Elle est essentiellement fonction de la pression art\'erielle \cite{vib_dsc}.

\par
La pression de perfusion $\Delta_P$ et le d\'ebit sanguin c\'er\'ebral CBF satisfont l'\'equation, analogue \g{a} la loi d'Ohm \'electrocin\'etique :
\begin{equation}
\Delta P = R\times CBF%
\text{ cette derni\g{e}re grandeur est d\'efinie dans la section d'introduction.}
\label{ohm}
\end{equation}

$R$ est naturellement appel\'ee la r\'esistance vasculaire c\'ebrale. D'apr\g{e}s \cite{vib_dsc}, elle d\'epend de :
\begin{enumerate}
\item La viscosit\'e du sang ;
\item L'\'etat anatomique du r\'eseau vasculaire : elle est proportionnelle \g{a} la longueur des vaisseaux ;
\item\label{reac_r_cbf} Le tonus vasculaire : elle est proportionnelle \g{a} la puissance quatri\g{e}me du rayon des vaisseaux ;
\item La pression du liquide c\'ephalorachidien.
\end{enumerate}

On observe, dans un cerveau sain, une r\'egulation du CBF : %
la figure suivante montre que, pour des variations de la pression de perfusion suffisamment basses, le CBF reste constant.

%%% Figure
\begin{figure}
\includegraphics[width=0.8\linewidth,height=7cm]{../../images_rapport/regCBF_1.png}
\caption{Le CBF ne change pas si $\Delta P\in \left[70\text{ mmHg};150\text{ mmHg}\right]$}
\label{regcbf_1}
\end{figure}

La cause de cette r\'egulation est une variation de $R$ qui compense celle de $\Delta P$, d'apr\g{e}s l'\'equation \ref{ohm}.

\par
Cette variation, li\'ee au diam\g{e}tre des vaisseaux sanguins, peut \^etre d'origine humorale : ad\'enosine \cite{vib_dsc} ; %
ou musculaire : r\'eflexe des muscles lisses qui se trouvent dans la paroi des vaisseaux.

\par
Les limites sup\'erieure et inf\'erieure de la r\'egulation peuvent \g{e}tre modifi\'ees par l'intervention du syst\g{e}me nerveux : voir la figure \ref{regcbf_2}.

%%% Figure
\begin{figure}
\includegraphics[width=0.8\linewidth,height=5.2cm]{../../images_rapport/regCBF_2.png}
\caption{Effets de l'excitation, ou de l'inhibition du syst\g{e}me nerveux sympathique sur la r\'egulation du CBF}
\label{regcbf_2}
\end{figure}

\etoile
Dans le cas d'un accident c\'er\'ebral isch\'emique, la pression de perfusion c\'er\'ebrale devient trop basse pour que cette r\'egulation soit efficace : %
le d\'ebit sanguin c\'er\'ebral baisse.

\ligneinter
Cette baisse ne se traduit pas imm\'ediatement par un dysfonctionnement des cellues nerveuses.

\par
En effet, la CMRO2, qui joue un r\^ole d\'ecisif dans la r\'eponse cellulaire \g{a} l'isch\'emie est li\'ee au CBF par l'\'equation suivante \cite{bufr97} :
\begin{equation}
OEF\times CBF\times [O_2]_a=CMRO2
\label{cmro2_cbf}
\end{equation}

Si l'on suppose $[O_2]_a$ constant -c'est une hypoth\g{e}se du mod\g{e}le expos\'e dans \cite{Duval_JCBFM_02}, %
cela signifie qu'une augmentation de OEF peut, seule, compenser une baisse du CBF %
que la r\'egulation \'evoqu\'ee pr\'ec\'edemment n'aura pas pu \'eviter. 

%\cite{Duval_JCBFM_02} r\'e\'ecrit cette \'equation :
%\[a\times OEF\times CBF = CMRO2\]
\par
Afin d'\'elaborer leurs simulations, les auteurs de \cite{Duval_JCBFM_02} ont fix\'e \g{a} 90 \% le seuil d'extraction (OEF) au del\g{a} duquel celle-ci ne peut plus augmenter. %
Si la baisse de CBF se poursuit apr\g{e}s que l'OEF ait d\'epass\'e 90\%, alors la CMRO2 baisse \g{a} son tour.

\etoile
La r\'egion du cerveau touch\'ee par une isch\'emie, apr\g{e}s les \'echecs successifs des deux r\'egulations qui pr\'ec\g{e}dent, %
est donc caract\'eris\'ee par :
\begin{itemize}
\item Des valeurs basses du CBF ;
\item Une CMRO2 faible.
\end{itemize}

La premi\g{e}re caract\'erisation ci-dessus sera utilis\'ee dans la section suivante pour d\'elimiter des r\'egions l\'es\'ees sur des images IRM, %
les cons\'equences d'une baisse de la CMRO2 sont r\'esum\'ees ci-dessous.

\subsubsection{Cascade isch\'emique}\label{casc}

Voici les principales \'etapes de l'isch\'emie c\'er\'ebrale, telles qu'elles sont r\'esum\'ees dans \cite{Durukan_PBB_07} et \cite{lar_poly_09}:

\begin{description}
\item[Insuffisance \'energ\'etique :] un apport insuffisant en dioxyg\g{e}ne et en glucose %
interrompt le fonctionnement des enzymes -ATP-ase et $H-$ATPase situ\'ees \g{a} la surface des cellules nerveuses. %
En cons\'equence, les concentrations intracellulaires en ions calcium, sodium et chlorure augmentent par effet d'osmose, %
de plus le potentiel de membrane des cellules touch\'ees par l'isch\'emie chute.
\item[Oed\g{e}me cytotoxique :] Les transferts ioniques d\'ecrits pr\'ec\'edemment, et l'entr\'ee d'eau qui accompagne les ions $Cl^-$, %
$Ca^{++}$ et $Na^+$ \g{a} l'int\'erieur d'une cellule nerveuse, provoquent la rupture de la membrane de cette cellule, donc sa mort par n\'ecrose.
\end{description}

Les neurones et cellules gliales concern\'es constituent la \emph{zone d'infarct} de l'isch\'emie, %
c'est-\g{a}-dire la r\'egion du tissu c\'er\'ebral qui est l\'es\'ee de fa\c con irr\'eversible.

\par
Cette r\'egion est entour\'ee de cellules nerveuses expos\'ees \g{a} %
un d\'eficit \'energ\'etique moins important que celui des cellules de la zone d'infarct. %
Elles adaptent leur activit\'e aux apports \'energ\'etiques, et constituent la \emph{p\'enombre isch\'emique}.

\begin{description}
\item[D\'epression envahissante :] La d\'epolarisation membranaire caus\'ee par l'insuffisance \'energ\'etique au d\'ebut de l'isch\'emie se propage \g{a} travers les axones. %
L'onde progressive qui s'ensuit est consommatrice d'\'energie, des cellules de la p\'enombre isch\'emique expos\'ees \g{a} ces ondes de fa\c con r\'ep\'et\'ee %
sont susceptibles de mourir par apoptose ou par n\'ecrose.
\item[G\'en\'eration de radicaux libres :] Elle est provoqu\'ee par l'arriv\'ee du calcium dans le milieu extracellulaire. %
Le pouvoir destructeur de ces mol\'ecules est tr\g{e}s important : membranes cellulaires, ADN \dots %
Certains radicaux libres modifient \'egalement les r\'eaction des cellules nerveuses \g{a} certains signaux :
ainsi, les ondes de d\'epression corticale provoquent un r\'etr\'ecissement du diam\g{e}tre des vaisseaux en r\'egion isch\'emi\'ee. %
Les migraines \g{a} aura sont un exemple de propagation d'ondes de d\'epolarisation qui, dans un tissu sain, entra\^inent une dilatation des vaisseaux sanguins.
\end{description}

%Dans son cours \ref{grdr1}, Emmanuel Grenier 
%Pénombre, reperfusion, neuroprotection, fenêtre thérapeutique.
%Mécanismes de régulation, saturations successives.
%Modèle dynamique : reconstitution d'images acquises pendant plusieurs jours d'examen, prédictions.
%Principales sources :
% 1- Durukan_PBB_07 donné le 11 août pour la cascade ischémique et des traitements ;
% 2- pin_RNAR_99 dans le dossier parent biblio ;
% 3- Girard dans cours_biologie pour les nécroses éparses en zone de pénombre ischémique ;
% 4- ...
%%% Important : mentionner deux échelles de temps pour la cascade ischémique. %%%
%% Oedème vasogénique, intervention du système immunitaire. %%
%\cite{grdr1}%Onde de d\'epolarisation

Les \'etapes de la cascade isch\'emique qui pr\'ec\g{e}dent se d\'eroulent dans les premi\g{e}res heures qui suivent le d\'ebut de l'accident. %
D'autres m\'ecanismes peuvent intervenir, quelques jours apr\g{e}s l'AVC. Citons, toujours d'apr\g{e}s \cite{Durukan_PBB_07} :

\begin{description}
\item[Oed\g{e}me vasog\'enique :] Une reperfusion trop violente peut endommager la barri\g{e}re h\'ematoenc\'ephalique, et entra\^iner sa rupture. %
Un oed\g{e}me, dit vasog\'enique, se forme avec l'infiltration de plasma dans le tissu c\'er\'ebral.
\item[R\'eaction du syst\g{e}me immunitaire :] c'est l'arriv\'ee programm\'ee de macrophages et de leucocytes dans le tissu c\'er\'ebral, plusieurs jours apr\g{e}s l'AVC. %
Malgr\'e leur r\^ole protecteur, ces cellules pourraient favoriser l'apoptose de cellules de la p\'enombre isch\'emique.
\end{description}

\subsection{Soins hospitaliers pour l'AVC isch\'emique}
%
%Dans toute la suite de ce travail, on consid\g{e}re et on mod\'elise des isch\'emies provoqu\'ees par l'obstruction 

\subsubsection{Traitement fibrinolytique}

Il s'agit d'un traitement chimique, administr\'e par intraveineuse, qui dissout le caillot sanguin \g{a} l'origine de l'isch\'emie. %
Cette intervention, bien connue du corps m\'edical, doit se pratiquer dans les heures qui suivent l'isch\'emie dans une unit\'e neuro-vasculaire \cite{lar_poly_09}.

\par
Cette op\'eration r\'etablit une perfusion c\'er\'ebrale normale, n\'ecessaire, mais non suffisante, \g{a} la r\'emission du patient.

\etoile
Il est rare, mais possible, qu'un traitement fibrinolytique entra\^ine une grave complication de l'AVC : %
la barri\g{e}re h\'ematoenc\'ephalique peut \^etre endommag\'ee par le traitement, ce qui provoque une h\'emorragie c\'er\'ebrale, %
souvent mortelle ou fortement invalidante.

\subsubsection{Neuroprotection}

% Diagramme avec les flèches vertes ?
Il s'agit de sauver les cellules qui se trouvent dans la zone de p\'enombre. Deux approches existent :
\begin{itemize}
\item Provoquer une hypothermie du tissu c\'erebral ;
\item Utilisation des mol\'ecules neuroprotectrices, qui \'eliminent par exemple les radicaux libres \cite{Durukan_PBB_07}.
\end{itemize}

L'efficacit\'e des traitements pharmacologiques d\'epend fortement du temps : %
la dur\'ee pendant laquelle l'injection d'une mol\'ecule peut \^etre b\'en\'efique est appel\'ee fen\^etre th\'erapeutique. %
L'utilisation combin\'ee de plusieurs traitements peut allonger cette dur\'ee selon \cite{Durukan_PBB_07}.

\subsection{Pr\'edictions}

Le probl\g{e}me pos\'e par la fen\^etre th\'erapeutique est difficile. %
L'approche pr\'econis\'ee ici consiste \g{a} mod\'eliser l'\'evolution temporelle d'un tissu isch\'emi\'e, %
puis d'impl\'ementer le mod\g{e}le afin de pouvoir effectuer des simulations.

\subsubsection{Simulation des premi\g{e}res heures qui suivent l'isch\'emie}

L'article \cite{Duval_JCBFM_02} donne un exemple de mod\'elisation, \g{a} l'\'echelle du voxel, %
des premi\g{e}res heures qui suivent l'isch\'emie caract\'eris\'ee par une baisse du d\'ebit sanguin c\'er\'ebral.

\par
Ce mod\g{e}le des stades pr\'ecoces de l'isch\'emie incorpore la r\'egulation de la CMRO2 avec l'\'equation \ref{cmro2_cbf}, %
et une diffusion de l'ADC suite \g{a} un oed\g{e}me cytotoxique.

\par
Le mod\g{e}le est impl\'ement\'e sur une grille de $7\times 7$ voxels. %
L'\'etat de chaque voxel se r\'esume aux valeurs prises par six variables, parmi lesquelles figurent l'ADC, le CBf et la CMRO2. %
L'une de ces variables, d\'efinie par les auteurs, caract\'erise l'\'etat du tissu \g{a} l'issue des six premi\g{e}res heures d'isch\'emie : %
sain, sauvable (p\'enombre) ou n\'ecros\'ee. Les valeurs prises par cette variable \g{a} l'issue des simulations doit \^etre conforme aux r\'esultats exp\'erimentaux. %
Ceux-ci concernent d'ailleurs des sujets humains.

\subsubsection{Simulation sur plusieurs semaines}

C'est l'objectif de ce travail. Plus pr\'ecis\'ement nous allons effectuer des simulations sur les voxels d'une coupe d'un cerveau de rat, %
examin\'e en IRM aux jours 0, 3, 8, 15 et 22 qui suivent une isch\'emie focale transitoire.
