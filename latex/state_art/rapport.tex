\section{L'accident vasculaire c\'er\'ebral isch\'emique}

%\subsection{Isch\'emie focale transitoire chez le rat}
%\subsection{IRM multiparam\'etrique}
%\setcounter{subsubsection}{-1}
%\subsubsection{Les donn\'ees brutes : $T_1$, $T_2$, $T^{\ast}_1$, $T^{\ast}_2$}
%\subsubsection{ADC}

%\subsection{Epid\'emiologie : l'accident vasculaire c\'er\'ebral (AVC)}

%Epidémiologie : cas général pour l'AVC.

\subsection{D\'efinition, \'epid\'emiologie}

Un accident vasculaire c\'er\'ebral est un arr\^et brutal de la circulation sanguine \g{a} l'int\'erieur du cerveau. %
110438 personnes \'etaient hospitalis\'ees pour cette raison en 2014 en France, soit pr\g{e}s de 160 hspitalisations sur 100 000 cette ann\'ee-l\g{a} \cite{epi_hem16}. %
Toujours d'apr\g{e}s \cite{epi_hem16}, l'AVC \'etait mortel dans 15744 cas ; %
52451 autres patients, soit 47,1\% des personnes hospitalis\'ees pour AVC, gardait des s\'equelles de l'accident : %
Atteinte neurologique, motrice aphasie \dots.

\par
Il existe deux sortes d'accidents vasculaires c\'er\'ebraux :
\begin{description}
\item[L'AVC h\'emorrhagique : ] c'est une cons\'equence de la rupture de la paroi d'un vaisseau sanguin \g{a} l'int\'erieur du cerveau, comme une rupture d'an\'evrisme ;
\item[L'AVC isch\'emique : ] c'est une baisse de la perfusion sanguine dans une r\'egion du cerveau, qui survient quand un caillot sanguin vient obstruer une art\g{e}re qui alimente cette r\'egion.
\end{description}

C'est ce deuxi\g{e}me type d'AVC qui nous int\'eresse, il touchait 78633 personnes hospitalis\'ees en 2014 en France.

\par
Notons que dans ce cas, 52,6\% des personnes hospitalis\'ees ont gard\'e des s\'equelles de leur AVC.

\subsection{Premiers stades de l'AVC isch\'emique}














%Pénombre, reperfusion, neuroprotection, fenêtre thérapeutique.


%Mécanismes de régulation, saturations successives.


%Modèle dynamique : reconstitution d'images acquises pendant plusieurs jours d'examen, prédictions.

%Principales sources :
% 1- Durukan_PBB_07 donné le 11 août pour la cascade ischémique et des traitements ;
% 2- pin_RNAR_99 dans le dossier parent biblio ;
% 3- Girard dans cours_biologie pour les nécroses éparses en zone de pénombre ischémique ;
% 4- ...

%%% Important : mentionner deux échelles de temps pour la cascade ischémique. %%%
%% Oedème vasogénique, intervention du système immunitaire. %%

\cite{vib_dsc}% pour la régulation myogénique du CBF
\cite{Duval_JCBFM_02}% pour l'absorption du déficit de CBF par l'OEF, ce qui stabilise la CMRO2

\cite{Durukan_PBB_07}%cascade ischémique


\subsection{Fen\^etre th\'erapeutique}


% Diagramme avec les flèches vertes ?

\subsection{Etat actuel des connaissances : traitements, mod\g{e}les, simulations}

\cite{Duval_JCBFM_02}

% Duval :
% - Echelle de temps ;
% - Progressivité : ADC ;
% - ...

%\subsection{A suivre : vers une nouvelle \'echelle de temps}