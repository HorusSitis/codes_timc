\section{L'accident vasculaire c\'er\'ebral isch\'emique}






\subsection{D\'efinition, \'epid\'emiologie}

\begin{frame}
%\frametitle{D\'efinition, \'epid\'emiologie}
\begin{itemize}
\item<+-> Arr\^et brutal de la circulation sanguine \g{a} l'int\'erieur du cerveau.
\item<+-> 110438 personnes hospitalis\'ees pour cette raison en 2014 en France.
\end{itemize}
\end{frame}





\begin{frame}
\frametitle{AVC isch\'emique}
\begin{block}{Deux sortes d'AVC}
\begin{enumerate}
\item<+-> AVC h\'emorragique :
\begin{itemize}
\item<+-> Rupture de la paroi d'un vaisseau sanguin.
\item<+-> Exemple : rupture d'an\'evrisme.
\end{itemize}
\item<+-> AVC isch\'emique :
\begin{itemize}
\item<+-> Obstruction d'une art\g{e}re par un vaisseau sanguin ;
\item<+-> 78633 personnes hospitalis\'ees en France (2014).
\end{itemize}
\end{enumerate}
\end{block}
\end{frame}

\subsection{D\'eroulement}

\begin{frame}
\begin{itemize}
\item<+-> \cite{pat_neu} : la cerveau consomme \g{a} lui seul 20 \% du dioxyg\g{e}ne de l'organisme.
\item<+-> Tr\g{e}s peu de r\'eserves \'energ\'etiques.
\end{itemize}
\end{frame}

\subsubsection{Baisse de la perfusion sanguine}

\begin{frame}
\frametitle{D\'ebit sangin c\'ebral et AVC}
\begin{block}{Perfusion sanguine $\Delta P$}
\begin{itemize}
\item<+-> C'est elle qui chute en cas d'obstruction d'une art\g{e}re ;
\item<+-> Lien avec le CBF : $\Delta P = R\times CBF$ \cite{vib_dsc}
\item<+-> $R$ : r\'esistance vasculaire c\'er\'ebrale
\end{itemize}
\end{block}
\end{frame}

\subsubsection{R\'egulation du CBF}

\begin{frame}
%\frametitle{R\'egulation du CBF}
\begin{block}{R\'esistance vasculaire c\'er\'ebrale \cite{vib_dsc}}
Elle d\'epend de :
%
\begin{enumerate}
\item<+-> La viscosit\'e du sang ;
\item<+-> L'\'etat anatomique du r\'eseau vasculaire : elle est proportionnelle \g{a} la longueur des vaisseaux ;
\item<+-> Le tonus vasculaire : elle est proportionnelle \g{a} la puissance quatri\g{e}me du rayon des vaisseaux ;
\item<+-> La pression du liquide c\'ephalorachidien.
\end{enumerate}
\end{block}
\end{frame}

\begin{frame}
\begin{tabular}{|c|c|}
\hline
\includegraphics[width=0.4\linewidth,height=3cm]{../../images_rapport/regCBF_1.png}
&
%\\
%\hline
%\par
\pause
\includegraphics[width=0.4\linewidth,height=3cm]{../../images_rapport/regCBF_2.png}%<+->
\\
\hline
\end{tabular}
%\pause
\end{frame}

\begin{frame}
\begin{block}{Baisse du CBF}
\begin{itemize}
\item<+-> AVC isch\'emique : $\Delta P$ est trop faible pour que le CBF soit stable.
\item<+-> Ceci se traduit par une baisse du CBF.
\end{itemize}
\end{block}
\end{frame}

\subsubsection{R\'egulation de la CMRO2}

\begin{frame}
%
\begin{block}{CBF et CMRO2}
\begin{itemize}
\item $OEF\times CBF\times [O_2]_a=CMRO2$
\item<+-> $[O_2]_a$ constante :
\item<+-> Seule une augemtation de l'OEF peut compenser la baisse du CBF.
\end{itemize}
\end{block}
\end{frame}

\begin{frame}
%
\begin{block}{Baisse de la CMRO2}<+->
\begin{itemize}
\item<+-> Dans \cite{Duval_JCBFM_02}, le maximum d'OEF est fix\'e \g{a} 90 \% .
\item<+-> Au del\g{a} de cette limite, la CMRO2 baisse.
\end{itemize}
\end{block}

%\par
\begin{block}{Isch\'emie : caract\'erisation}<+->
\begin{itemize}
\item<+-> Des valeurs basses du CBF ;
\item<+-> Une CMRO2 faible.
\end{itemize}
\end{block}
%
\end{frame}



\subsubsection{Cascade isch\'emique}

\begin{frame}
\begin{description}
\item[Insuffisance \'energ\'etique :]<+-> Baissse de la CMRO2 et de CMRGlu :
\begin{itemize}
\item<+-> Baisse de la production d'ATP ;
\item<+-> Les enzymes -ATP-ase et $H-$ATPase ne fonctionnent plus ;
\item<+-> Les concentrations intracellulaires en ions calcium, sodium et chlorure augmentent ;
\item<+-> Chute du potentiel de membrane des cellules nerveuse.
\end{itemize}
%
\item[Oed\g{e}me cytotoxique :]<+-> Rupture de la membrane de la cellule nerveuse, puis mort par n\'ecrose : infarctus c\'er\'ebral.
\end{description}
\end{frame}

\begin{frame}
\begin{description}
\item[D\'epression envahissante :]<+-> La chute du potentiel membranaire se propage \g{a} travers les axones :
\begin{itemize}
\item<+-> Consomation d'\'energie ;
\item<+-> Morts par n\'ecrose et par apoptose.
\end{itemize}
%
\item[G\'en\'eration de radicaux libres :]<+-> Provoqu\'ee par l'augmentation de $[Ca^{++}]_{inrtacellulaire}$.
\begin{itemize}
\item<+-> Mol\'ecules nocives ;
\item<+-> Modifient la r\'eactivit\'e du cerveau \g{a} certains signaux (ondes \'electriques corticales).
\end{itemize}
\end{description}
\end{frame}


\subsection{Soins}

\subsubsection{Soins hospitaliers}

\begin{frame}
\begin{block}{Traitement fibrinolytique}
\begin{itemize}
\item<+-> Traitement pharacologique qui dissout le caillot sanguin \g{a} l'origine de l'AVC ;
\item<+-> Doit \^etre pratiqu\'e en urgence.
\end{itemize}
\end{block}
%
\begin{block}{Neuroprotection}<+->
\begin{itemize}
\item<+-> Hypothermie du tissu c\'er\'ebral ;
\item<+-> Traitements pharmacologiques.
\end{itemize}
\end{block}
\end{frame}

\begin{frame}
\frametitle{Pistes pour des traitements pharmacologique}

\includegraphics[width=0.7\linewidth,height = 6.8cm]{../../images_rapport/chimio.png}
\end{frame}

\subsubsection{Enjeu de la mod\'elisation}

\begin{frame}
\begin{description}
\item[Fen\^etre th\'erapeutique]<+-> Dur\'ee pendant laquelle l'injection d'une mol\'ecule est b\'en\'efique ;
\item[Solution : ]<+-> pr\'edire l'\'evolution du tissu c\'ebral.
\end{description}
\end{frame}

\begin{frame}
\begin{itemize}
\item<+-> \cite{Duval_JCBFM_02} : les auteurs ont simul\'e l'\'evolution d'un tissu isch\'emi\'e dans les trois heures qui suivent un AVC isch\'emique ;
\item<+-> Travail actuel : simulation de l'\'evolution d'un tissu sur plusieurs semaines.
\end{itemize}
\end{frame}




