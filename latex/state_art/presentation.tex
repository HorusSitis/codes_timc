\section{L'accident vasculaire c\'er\'ebral isch\'emique}






\subsection{D\'efinition, \'epid\'emiologie}

\begin{frame}
%\frametitle{D\'efinition, \'epid\'emiologie}
\begin{itemize}
\item<+-> Arr\^et brutal de la circulation sanguine \g{a} l'int\'erieur du cerveau.
\item<+-> 110438 personnes hospitalis\'ees pour cette raison en 2014 en France.
\end{itemize}
\end{frame}





\begin{frame}
\frametitle{AVC isch\'emique}
\begin{block}{Deux sortes d'AVC}
\begin{enumerate}
\item<+-> AVC h\'emorragique :
\begin{itemize}
\item<+-> Rupture de la paroi d'un vaisseau sanguin.
\item<+-> Exemple : rupture d'an\'evrisme.
\end{itemize}
\item<+-> AVC isch\'emique :
\begin{itemize}
\item<+-> Obstruction d'une art\g{e}re par un vaisseau sanguin ;
\item<+-> 78633 personnes hospitalis\'ees en France (2014).
\end{itemize}
\end{enumerate}
\end{block}
\end{frame}

\subsection{D\'eroulement}

\begin{frame}
\begin{itemize}
\item<+-> \cite{pat_neu} : la cerveau consomme \g{a} lui seul 20 \% du dioxyg\g{e}ne de l'organisme.
\item<+-> Tr\g{e}s peu de r\'eserves \'energ\'etiques.
\end{itemize}
\end{frame}

\begin{frame}
\frametitle{D\'ebit sangin c\'ebral et AVC}
\begin{block}{Perfusion sanguine $\Delta P$}
\begin{itemize}
\item<+-> C'est elle qui chute en cas d'obstruction d'une art\g{e}re ;
\item<+-> Lien avec le CBF : $\Delta P = R\times CBF$
\end{itemize}
\end{block}
\end{frame}

\begin{frame}
\frametitle{R\'egulation}



\end{frame}
