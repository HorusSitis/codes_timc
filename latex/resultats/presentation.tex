\section{Simulations}

\subsection{Principe}

\begin{frame}
%\begin{block}
\begin{itemize}
%\item<+-> Le logiciel utilis\'e pur les simulations est R.
\item<+-> On enregistre les valeurs de BVf, CBF, CMRO2, SO2 et VSI sur les voxels de la coupe 9 du rat 19 ;
\item<+-> On choisit ensuite, sur cette coupe, une grille carr\'ee de $5\times 5$ pixels.
\item<+-> La simulation s'effectue sur un temps discret : on calcule, sur un mois, l'\'etat de la grille chaque jour ;
\item<+-> Equations discr\g{e}tes : $\text{CBF}_{trans}-\text{CBF}(t+1)=q\left(\text{CBF}_{trans}-\text{CBF}(t)\right)$
\end{itemize}
%\end{block}
\end{frame}

\subsection{Exemples : avec la CMRO2}

\begin{frame}
\includegraphics[width=0.6\textwidth,height=4cm]{../../images_rapport/19-J08-CBF_clust_lesion.pdf}

On d\'efinit deux carr\'es de $5\times 5$ pixels, par la position de leurs sommets inf\'erieurs gauches $S_1$ et $S_2$ :
\begin{enumerate}%[label=(Carre\arabic*)]
\item<+-> $S_1$ est le pixel (30,45) ;
\item<+-> $S_2$ est le pixel (30,45) .
\end{enumerate}
\end{frame}



\begin{frame}
\frametitle{L\'esion 1}

\begin{tabular}{|c|c|}
\hline
\includegraphics[width=0.4\textwidth,height=3cm]{../../images_rapport/cmro2demo_00.pdf}
&
\includegraphics[width=0.4\textwidth,height=3cm]{../../images_rapport/cmro2demo_08.pdf}
\\
\hline
\includegraphics[width=0.4\textwidth,height=3cm]{../../images_rapport/cmro2demo_15.pdf}
&
\includegraphics[width=0.4\textwidth,height=3cm]{../../images_rapport/cmro2demo_22.pdf}
\\
\hline
\end{tabular}
\end{frame}



\begin{frame}
\frametitle{L\'esion 2}

\begin{tabular}{|c|c|}
\hline
\includegraphics[width=0.4\textwidth,height=3cm]{../../images_rapport/Cmro2demo_00.pdf}
&
\includegraphics[width=0.4\textwidth,height=3cm]{../../images_rapport/Cmro2demo_08.pdf}
\\
\hline
\includegraphics[width=0.4\textwidth,height=3cm]{../../images_rapport/Cmro2demo_15.pdf}
&
\includegraphics[width=0.4\textwidth,height=3cm]{../../images_rapport/Cmro2demo_22.pdf}
\\
\hline
\end{tabular}
\end{frame}


%\subsection{}

\begin{frame}
\frametitle{Mises en garde pour la suite}
\begin{itemize}
\item<+-> Les param\g{e}tres du mod\g{e}le doivent \^etre ajust\'es ;
\item<+-> Il est difficile de s\'electionner un \'echantillon de tissu homog\g{e}ne ;
\item<+-> Pour un voxel donn\'e : les valeurs de certaines variables peuvent manquer.
\end{itemize}
\end{frame}
