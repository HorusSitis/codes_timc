\section{R\'esultats des simulations}

\subsection{Calculs \g{a} partir du jour 0}


\subsection{Calculs \g{a} partir du jour 8}


\begin{figure}[!p]
\begin{center}
\begin{tabular}{|c|c|}
\hline
%\subfloat[BVf : d\'elimitation approximative, parasites.]
&
%\subfloat[T1map : structures visibles, %
%mais sans lien avec la l\'esion.]
\\
\hline
%\subfloat[CBF : pas mieux que l'oeil.. %
%Clusters vert et rouge difficiles \g{a} interpr\'eter.]
&
%\subfloat[CMRO2 : mauvaise qualit\'e, taches bleues que l'h\'emisph\g{e}re contralat\'eral.]
\\
\hline
%\subfloat[ADC : r\'esultat encourageant]
&
%\subfloat[CBF : pas exploitable pour distinguer la r\'egion l\'es\'ee, sur le cerveau entier.]
\\
\hline
\end{tabular}
\end{center}
\caption{Simulations pour le rat 19 : les six modalit\'es. Suivi des jours 8 \g{a} 22.}
\label{19_sim_822}
\end{figure}