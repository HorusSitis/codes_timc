\section{R\'esultats des simulations}

\subsection{Principe}

Dans cette section, on pr\'esente le principe de deux algorithmes de sipmulation, %
inspir\'es des \'equations de la section pr\'ec\'edente.

\par
Chacun de ces deux algorithmes prend en entrées :
\begin{itemize}
\item Un carr\'e de $5\times 5$ pixels des IRM disponibles au jour 0, pour le rat num\'ero 19 ;
\item Un nom correspondant \g{a} leur emplacement dans la l\'esion ;
\item Le nombre de jours que dure la simulation : dans les simulations qui vont suivre, 5, 10, 20 ou 30 jours.
\end{itemize}

Le sc\'ema ci-dessous r\'esume le d\'eroulement des simulations :

%\begin{comment}
\[%
\xymatrixcolsep{1.5cm}%
\xymatrix{%
&%
\fbox{\text{\textbf{L\'esion 1-2}}}\ar[r]^{\scriptstyle CBF<100}_{\scriptstyle CMRO2<10}&%
\fbox{\text{\textbf{L\'esion 1-3}}}%  R\'egime asymptotique}%
%&
%
\\
\fbox{\text{\textbf{R\'egime commun}}}\ar[ru]^{T_{\chi}}\ar[rd]_{T_{\chi}}&%
&%
%&
%
\\
&%
\fbox{\text{\textbf{L\'esion 2-2}}}\ar[r]^{\scriptstyle CBF>100}_{\scriptstyle CMRO2>10}&%
\fbox{\text{\textbf{L\'esion 2-3}}}%
%\ar[d]^{SO2\dots}&
%\mbox{\textbf{L\'esion 2-4}}
}
\]
%\end{comment}

Les encarts repr\'esentent chacun des syst\g{e}mes d'\'equations pr\'esent\'es \g{a} la section pr\'ec\'edente.

\par
Les fl\g{e}ches sont index\'ees avec les \'ev\g{e}nements qui, %
lorsqu'ils se produisent, entra\^inent un changement dans les \'equations qui r\'egissent l'\'evolution du voxel concern\'e.

\par
Ainsi, le crit\g{e}re qui marque la fin des r\'egimes : %
\og{} l\'esion 1.2\fg{} et \og{} l\'esion 2.2\fg{} est le franchissement, par le CBF et par la CMRO2, %
des valeurs seuil de 100 et 10 respectivement.

\etoile
On note que les valeurs des diff\'erentes variables aux jours 0 et 3 ne permettent pas, seules, %
de distinguer les pixels qui vont \'evoluer vers un type de l\'esion donn\'e.

\par
On se propose, pour nos simulations, de quitter le r\'egime initial apr\g{e}s un temps $T_{\chi}$ al\'eatoire, %
avec une distribution centr\'ee en 5 : $\mathcal{N}(5,2)$. %
Nous avons vu dans une section pr\'ec\'edente que certains m\'ecanismes physiologiques, %
caus\'es par l'isch\'emie, interviennent tardivement (dans les premiers jours qui suivent l'AVC).

\ligneinter
Le logiciel utilis\'e pur les simulations est R.

\par
Pour avoir une condition initiale, on enregistre les valeurs des cinq variables BVf, CBF, CMRO2, SO2 et VSI sur les pixels de la coupe 9 du rat 19.

\par
On choisit ensuite, sur cette coupe, une grille carr\'ee de $5\times 5$ pixels. %
Dans les exemples qui vont suivre, la grille est choisie au sein de la partie l\'es\'ee.

\par
On indique \'egalement \g{a} l'algorithme le type d'\'evolution que l'on souhaite simuler : l\'esion 1, l\'esion 2.

\par
La simulation s'effectue sur un temps discret : on calcule, sur un mois, l'\'etat de la grille chaque jour. %
On comparera ensuite les r\'esultats obtenus aux jours 3, 8, 15 et 22 aux donn\'ees exp\'erimentales.

\etoile
Les \'equations d'\'evolution des cinq variables BVf, \dots , VSI ne sont pas impl\'ement\'ees %
telles quelles pour cette simulation \g{a} temps discret. %
Pour fixer les id\'ees, on remplace l'\'equation :
\[\der{CBF}=\frac{1}{\tau}\times \left(CBF_{trans}-CBF\right)\]

par :
\[\text{CBF}_{trans}-\text{CBF}(t+1)=q\left(\text{CBF}_{trans}-\text{CBF}(t)\right)\]
Un calcul analogue \g{a} celui du cas continu donne : %
$q =\left(\dfrac{m^3-m^0}{m^3-Q^3_1}\right)^{\frac{1}{3}}$ soit $q\simeq 0,6$.


\subsection{Deux exemples avec la CMRO2}


Afin de bien s\'electionner les pixels sur lesquels on effectue des simulation, on se reporte au diagramme ci-dessous :

\includegraphics[width=0.6\textwidth,height=11cm]{../../images_rapport/19-J08-CBF_clust_lesion.pdf}

On d\'efinit deux carr\'es de $5\times 5$ pixels, par la position de leurs sommets inf\'erieurs gauches $S_1$ et $S_2$ :
\begin{enumerate}[label=(Carre\arabic*)]
\item $S_1$ est le pixel (30,45). Le carr\'e ainsi d\'efini se situe \g{a} gauche de la r\'egion d\'elimit\'ee en bleu ;
\item $S_2$ est le pixel (30,45). Le carr\'e ainsi d\'efini se situe \g{a} l'extr\^eme gauche de la figure, %
dans la r\'egion d\'elimit\'ee en rouge.
\end{enumerate}

Les deux figures qui suivent repr\'esentent les distributions des valeurs simul\'ees (courbes rouges) %
et mesur'ees (bleu pointillé) de la CMRO2 sur la coupe 9 du rat num\'ero 9. 

\begin{figure}[!p]
\begin{center}
\begin{tabular}{|c|c|}
\hline
\includegraphics[width=0.4\textwidth,height=6cm]{../../images_rapport/cmro2demo_00.pdf}
&
\includegraphics[width=0.4\textwidth,height=6cm]{../../images_rapport/cmro2demo_08.pdf}
\\
\hline
\includegraphics[width=0.4\textwidth,height=6cm]{../../images_rapport/cmro2demo_15.pdf}
&
\includegraphics[width=0.4\textwidth,height=6cm]{../../images_rapport/cmro2demo_22.pdf}
\\
\hline
\end{tabular}
\end{center}
\caption{Comparaison : valeurs prévues vs mesurées pour la CMRO2.%
\\
Carr\'e : 1}
\label{les1simc}
\end{figure}

\begin{figure}[!p]
\begin{center}
\begin{tabular}{|c|c|}
\hline
\includegraphics[width=0.4\textwidth,height=6cm]{../../images_rapport/Cmro2demo_00.pdf}
&
\includegraphics[width=0.4\textwidth,height=6cm]{../../images_rapport/Cmro2demo_08.pdf}
\\
\hline
\includegraphics[width=0.4\textwidth,height=6cm]{../../images_rapport/Cmro2demo_15.pdf}
&
\includegraphics[width=0.4\textwidth,height=6cm]{../../images_rapport/Cmro2demo_22.pdf}
\\
\hline
\end{tabular}
\end{center}
\caption{Comparaison : valeurs prévues vs mesurées pour la CMRO2.%
\\
Carr\'e : 2}
\label{les2simC}
\end{figure}

%\FloatBarrier
\subsection{Pour conclure \dots}

Les observations des figures \ref{les1simc} et \ref{les2simC}, qui concernant les pr\'edistions de CMRO2, %
montrent que le mod\g{e}le doit \^etre corrig\'e.

\par
En effet, la distribution asymptotique de la CMRO2, dans le premier cas qui concerne la l\'esin 1, %
est d\'ecal\'ee par rapport aux r\'esultats exp\'erimentaux.

\par
Les pr\'edictions effectu\'ees sur les deuxi\g{e}me carr\'e, et qui concernent la deuxi\g{e}me r\'egion l\'es\'ee, %
sont proches des r\'esultats exp\'erimentaux, sauf pour le jour 15.

\etoile
Remarquons toutefois que les valeurs de CMRO2 mesur\'ees sur les deux carr\'es de pixels sont h\'et\'erog\g{e}nes. %
Cela peut signifier que la partition \og{} l\'esion1-l\'esion 2\fg{} doit \^etre affin\'ee, %
ou que les carr\'es empi\g{e}tent sur chacune des deux zones.

\par
D'une mani\g{e}re g\'en\'erale, il est difficile de s\'electionner, m\^eme pour un petit nombre de voxels, %
un \'echantillon de tissu homog\g{e}ne.

\par
En outre, il est fr\'equent que les valeurs des diff\'erentes variables ne soient pas toutes disponibles, pour un voxel donn\'e; %
les \'echantillons de voxels utilisables sont donc assez rares.

