\documentclass[a4paper,10pt]{article}
\usepackage[utf8]{inputenc}

%\documentclass[a4paper,12pt]{article}
%\usepackage[utf8]{inputenc}

\usepackage[top=1cm,bottom=1.2cm,left=1.55cm,right=1.55cm]{geometry}% On peut changer en cours de route avec la commande \newgeometry

\usepackage[utf8]{inputenc}
\usepackage[english,french]{babel}
\usepackage{amsmath,amsfonts,amssymb,graphicx}
%\usepackage{amsthm}
\usepackage{framed}
\usepackage[amsthm,thmmarks,framed]{ntheorem}
\usepackage[dvipsnames]{pstricks}
%\usepackage{pstricks-add,pst-plot,pst-node}
\usepackage{pstricks,pst-plot,pst-text,pst-tree}%,pst-eps,pst-fill,pst-node,pst-math,pstricks-add,pst-xkey}
\usepackage{epic,eepic}
% Versions propos\'ees par Frédéric Junier
%\usepackage[inline]{asymptote}

\usepackage{placeins}
\usepackage[lofdepth,lotdepth]{subfig}
% Entre autres, pour sauter des lignes dans un titre de figure.
\usepackage{caption}%[hang,small]


%\usepackage{color} %(black, white, red, green, blue, yellow, magenta et cyan)
\usepackage[dvipsnames]{xcolor}
\usepackage{colortbl}

\usepackage{verbatim}

\usepackage{enumitem}
\frenchbsetup{StandardLists=true}

\usepackage[all]{xy}

\usepackage{multicol}				%%%	Jusqu' \`a nouvel ordre pas de multicol, pour cause de compatibilit\'e avec des fonctions graphiques de TeX.
%\begin{multicols}[titre]{nb colonnes}
\setlength{\columnseprule}{0.25pt}

\usepackage{parskip}
\setlength{\parindent}{0cm}

\usepackage{float}
% Pour des figures non flottantes par exemple : placer une figure où je veux.

%Je veux une mise en page pour les théorèmes lemmes preuves, définitions etc, suffisamment aérée. Est-il possible de commander un encadrement systématique ?

\newcommand{\g}[1]{\`#1}%\g{lettre}

\providecommand{\abs}[1]{\lvert#1\rvert}

%Pour tout le m\'emoire !



%Annotations d'un m\'emoire en cours de r\'edaction :
\newcommand{\tr}{\textbf{\textcolor{red}{D\'emonstration \g{a} trouver }}}
\newcommand{\re}{\textbf{\textcolor{NavyBlue}{D\'emonstration \g{a} recopier }}}
\newcommand{\es}{\textbf{\textcolor{OliveGreen}{Esquisse de d\'emonstration }}}

%Pour les formules alg\'ebriques casse-pieds :


%Pour les formules concernant les \dots :


%Autres formules pour le m\'emoire :
\newcommand{\tc}[1]{\text{#1}}


\newcommand*{\etoile}
{
\begin{center}
\hspace{1pt}\par
*\hspace{5pt}*\hspace{5pt}*
\end{center}
}


\newcommand*{\ligneinter}
{
\begin{center}
\vspace{2pt}
\hfill\rule{0.5\linewidth}{0.1pt}\hfill\null
\end{center}
\vspace{7pt}
}

{
\theoremstyle{break}
\theoremprework{\vspace{0.2cm}\begin{minipage}{\textwidth}} %Pris en compte uniquement pour le premier newtheorem
\theorempostwork{\ligneinter\end{minipage}} %Même chose
\theoremheaderfont{\scshape}
\theorembodyfont{\itshape}
\theoremseparator{ :\newline\vspace{0.2cm}}
\newtheorem{defi}{D\'efinition}%[section]
%\newtheorem{prop}{Proposition}[section]
%\newtheorem{lemm}{Lemme}[section]
} 

{%
\theoremstyle{break}
\theoremprework{\vspace{0.2cm}\begin{minipage}{\textwidth}} %Pris en compte uniquement pour le premier newtheorem
\theorempostwork{\ligneinter\end{minipage}} %Même chose
\theoremheaderfont{\bfseries}
\theorembodyfont{\itshape}
\theoremseparator{ :\newline\vspace{0.2cm}}
%\newtheorem{lemm}{Lemme}[section]
\newtheorem{prop}{Proposition}[section]
}

{%
\theoremstyle{break}
\theoremprework{\vspace{0.2cm}\begin{minipage}{\textwidth}} %Pris en compte uniquement pour le premier newtheorem
\theorempostwork{\etoile\end{minipage}} %Même chose
\theoremheaderfont{\bfseries}
\theorembodyfont{\itshape}
\theoremseparator{ :\newline\vspace{0.2cm}}
%\newtheorem{lemm}{Lemme}[section]
\newtheorem{pref}{Proposition-d\'efinition}[section]
}

{%
\theoremstyle{break}
%\theoremprework{\begin{tabular}{|p{\textwidth}|}\hline}
%\theorempostwork{\\ \hline\end{tabular}}
\theoremheaderfont{\scshape}
\theorembodyfont{\normalfont}
\theoremseparator{ :\newline\vspace{0.2cm}}
%\newtheorem{lemm}{Lemme}[section]
\newframedtheorem{theo}{Théor\`eme}[section]
}

{%
\theoremstyle{break}
\theoremprework{\vspace{0.2cm}\begin{minipage}{\textwidth}}
\theorempostwork{\ligneinter\end{minipage}}
\theoremheaderfont{\scshape}
\theorembodyfont{\itshape}
\theoremseparator{ :\newline\vspace{0.2cm}}
\newtheorem{lemm}{Lemme}[theo]
}
  
{%
\theoremstyle{break}
%\theoremprework{\begin{tabular}{|p{\textwidth}|}\hline}
%\theorempostwork{\\ \hline\end{tabular}}
\theoremheaderfont{\scshape}
\theorembodyfont{\itshape}
\theoremseparator{ :\newline\vspace{0.2cm}}
%\newtheorem{lemm}{Lemme}[section]
\newframedtheorem{coro}{Corollaire}[theo]
}



{
\theoremstyle{break}
\theoremprework{\vspace{0.5cm}}
\theorempostwork{\vspace{0.5cm}\ligneinter}
\theoremheaderfont{\scshape}
\theorembodyfont{\normalfont\small}
\theoremseparator{ :\newline\vspace{0.2cm}}
\newtheorem{exem}{Exemple}[section]
} 

{
\theoremstyle{break}
\theoremprework{\vspace{0.5cm}\begin{minipage}{\textwidth}}
\theorempostwork{\end{minipage}\ligneinter}
\theoremheaderfont{\scshape}
\theorembodyfont{\small}
\theoremseparator{ :\newline\vspace{0.2cm}}
\newtheorem{rema}{Remarque}[section]
} 

%%% Quelques environnements bien comodes %%%

\newenvironment{modmerate}[2]%
  {%
  \renewcommand{\arraystretch}{2.1}
  \begin{tabular}{|c|}
  \hline
  %\hrule%[0.2cm]{1mm}{\linewidth}
  %\paragraph{#1}
  \textbf{#1}
  \\
  \hline
  %\hrule%[0.2cm]{1mm}{\linewidth}
  \begin{minipage}{0.8\linewidth}
  \begin{enumerate}[label=#2]
  }
  %
  %%% Début ... et fin. %%%
  %
  {\end{enumerate}
  \end{minipage}
  \\
  %\hrule%[0.2cm]{1mm}{\linewidth}
  \hline
  \end{tabular}
  }

%Pour de belles EDO.
\newcommand{\der}[1]{\frac{\text{d}}{\text{d}t}#1}

% Pour les types d'équations d'évolution, modèles 2 et 3demi.
\newcommand{\Tb}{\ \mathbf{Type 0_{\mathbb{P}}}}
\newcommand{\Ta}{\ \mathbf{Type 1}}
\newcommand{\Td}{\ \mathbf{Type 2}}

%Pour stocker des valeurs enumi par exemple, si l'on quitte provisoirement l'environnement eummerate.
\newcounter{stock}

%\newcommand{\Tm}{T1}

%\setcounter{section}{-1}
%\renewcommand{\thesection}{\Roman{section}}
%\renewcommand{\thesubsection}{\Roman{section}-\Alph{subsection}}

%\includeonly{resultats/rapport}

%%%%%%%%%%%%%%%%% Préambule commun aux papiers du stage %%%%%%%%%%%%%%%%

%opening
\title{Mod\'elisation des suites d'un AVC isch\'emique chez le rat : \'evolution sur trois semaines}
\author{Antoine Moreau}

\begin{document}

\maketitle

\begin{abstract}
Dans ce travail, on se propose de trouver une m\'ethode pour pr\'edire, %
\g{a} l'aide d'images IRM obtenues sur un cerveau de rat isch\'emi\'e, %
l'\'evolution du tisu l\'es\'e au cours de semaines qui suivent la reperfusion. %
%
On utilise pour cela des images IRM obtenues, pendant plusieurs jours, %
sur les cerveaux de quatre rats ayant subi une occusion intraluminale d'une art\g{e}re c\'er\'ebrale. %
%
En confrontant les informations obtenues, pour plusieurs modalit\'es d'examen %
-IRM de diffusion, utilisation d'agents de contrastes- %
on propose deux mod\g{e}les ph\'enom\'enologiques de tissu isch\'emi\'es. %
%
Une impl\'ementation est propos\'ee pour ces mod\g{e}les d'\'evolution. %
%
Les param\g{e}tres de ces mod\g{e}les doivent encore \^etre ajust\'es, %
afin d'avoir des pr\'edictions conformes aux r\'esultats exp\'erimentaux.
\end{abstract}

%\section{Introduction : l'imagerie par r\'esonnance magn\'etique (IRM)}
\section{Introduction}

\begin{comment}
Mon stage s'est d\'eroul\'e au laboratoire TIMC-IMAG (Techniques de l'Ingénierie M\'edicale et de la Complexit\'e - %
Informatique, Mathématiques et Applications, Grenoble), sous la direction d'Angélique Stéphanou (CNRS). %
Il s'inscrit dans le projet Imagerie par Résonnance Magn\'etique Augment\'ee (IRMA) qui vise \g{a} effectuer des pr\'edictions, %
\g{a} partir d'images IRM, sur des param\g{e}tres qui ne sont pas directement mesurables comme la pression art\'erielle en dioxyg\g{e}ne $P_aO_2$ %
ou avec une meilleure r\'esolution spatiotemporelle que celle des images brutes. %
Un autre aspect de l'augmentation virtuelle de l'imagerie par résonnance magnétique est la simulation, \g{a} partir d'une image initiale, %
de l'évolution de l'état d'un patient sur des \'echelles de temps plus ou moins longues. %
C'est l'objet de ce travail, qui doit permettre de retracer l'\'evolution, sur plusieurs semaines, de l'\'etat d'un cerveau de rat isch\'emi\'e.
\end{comment}

Augmentation virtuelle de l'IRM : un projet du TIMC et du GIN

IRM augment\'ee : mettre au point des programmes informatiques permettant d'interpr\'eter syst\'ematiquement des images IRM.
\begin{enumerate}
\item Effectuer des pr\'edictions : simuler l'\'evolution future de l'\'etat du tissu c\'er\'ebral chez un patient%
%\begin{itemize}

%\end{itemize}
\item Am\'eliorer 
\item Calculer des grandeurs, comme la pression partielle en $O_2$, qui en sont pas mesurables directement \g{a} partir des images.
\end{enumerate}






\subsection{Principe de l'IRM}%State of the art, pour cette question plus générale que l'objet de ce stage.


%IRM : historique ? Jusqu'\g{a} l'IRM multiparam\'etrique.
%Avantages et limitations, bri\g{e}vement.





%\subsection{Une utilisation remarquable : L'IRM fonctionnelle (1991)}% spécifique au cerveau







\subsection{Mesures effectu\'ees \g{a} l'aide d'un agent de contraste}

%Pratiquée au GIN






%QUITOXIC MRI pour la CMRO2 et l'OEF.

\subsection{L'effet BOLD}

%Effet BOLD




%SO2

\cite{christen2012l}


%\subsection{Une alternative : la tomographie par \'emission de positrons (TEP)}




%Sc\'ema : grille de pixels et de cellules.
%Automate cellulaire.

%\section{L'accident vasculaire c\'er\'ebral isch\'emique}
\section{Introduction}

\begin{comment}
Mon stage s'est d\'eroul\'e au laboratoire TIMC-IMAG (Techniques de l'Ingénierie M\'edicale et de la Complexit\'e - %
Informatique, Mathématiques et Applications, Grenoble), sous la direction d'Angélique Stéphanou (CNRS). %
Il s'inscrit dans le projet Imagerie par Résonnance Magn\'etique Augment\'ee (IRMA) qui vise \g{a} effectuer des pr\'edictions, %
\g{a} partir d'images IRM, sur des param\g{e}tres qui ne sont pas directement mesurables comme la pression art\'erielle en dioxyg\g{e}ne $P_aO_2$ %
ou avec une meilleure r\'esolution spatiotemporelle que celle des images brutes. %
Un autre aspect de l'augmentation virtuelle de l'imagerie par résonnance magnétique est la simulation, \g{a} partir d'une image initiale, %
de l'évolution de l'état d'un patient sur des \'echelles de temps plus ou moins longues. %
C'est l'objet de ce travail, qui doit permettre de retracer l'\'evolution, sur plusieurs semaines, de l'\'etat d'un cerveau de rat isch\'emi\'e.
\end{comment}

Augmentation virtuelle de l'IRM : un projet du TIMC et du GIN

IRM augment\'ee : mettre au point des programmes informatiques permettant d'interpr\'eter syst\'ematiquement des images IRM.
\begin{enumerate}
\item Effectuer des pr\'edictions : simuler l'\'evolution future de l'\'etat du tissu c\'er\'ebral chez un patient%
%\begin{itemize}

%\end{itemize}
\item Am\'eliorer 
\item Calculer des grandeurs, comme la pression partielle en $O_2$, qui en sont pas mesurables directement \g{a} partir des images.
\end{enumerate}






\subsection{Principe de l'IRM}%State of the art, pour cette question plus générale que l'objet de ce stage.


%IRM : historique ? Jusqu'\g{a} l'IRM multiparam\'etrique.
%Avantages et limitations, bri\g{e}vement.





%\subsection{Une utilisation remarquable : L'IRM fonctionnelle (1991)}% spécifique au cerveau







\subsection{Mesures effectu\'ees \g{a} l'aide d'un agent de contraste}

%Pratiquée au GIN






%QUITOXIC MRI pour la CMRO2 et l'OEF.

\subsection{L'effet BOLD}

%Effet BOLD




%SO2

\cite{christen2012l}


%\subsection{Une alternative : la tomographie par \'emission de positrons (TEP)}




%Sc\'ema : grille de pixels et de cellules.
%Automate cellulaire.% et \'etat actuel des connaissances}

%\section{M\'ethodes}
\section{Introduction}

\begin{comment}
Mon stage s'est d\'eroul\'e au laboratoire TIMC-IMAG (Techniques de l'Ingénierie M\'edicale et de la Complexit\'e - %
Informatique, Mathématiques et Applications, Grenoble), sous la direction d'Angélique Stéphanou (CNRS). %
Il s'inscrit dans le projet Imagerie par Résonnance Magn\'etique Augment\'ee (IRMA) qui vise \g{a} effectuer des pr\'edictions, %
\g{a} partir d'images IRM, sur des param\g{e}tres qui ne sont pas directement mesurables comme la pression art\'erielle en dioxyg\g{e}ne $P_aO_2$ %
ou avec une meilleure r\'esolution spatiotemporelle que celle des images brutes. %
Un autre aspect de l'augmentation virtuelle de l'imagerie par résonnance magnétique est la simulation, \g{a} partir d'une image initiale, %
de l'évolution de l'état d'un patient sur des \'echelles de temps plus ou moins longues. %
C'est l'objet de ce travail, qui doit permettre de retracer l'\'evolution, sur plusieurs semaines, de l'\'etat d'un cerveau de rat isch\'emi\'e.
\end{comment}

Augmentation virtuelle de l'IRM : un projet du TIMC et du GIN

IRM augment\'ee : mettre au point des programmes informatiques permettant d'interpr\'eter syst\'ematiquement des images IRM.
\begin{enumerate}
\item Effectuer des pr\'edictions : simuler l'\'evolution future de l'\'etat du tissu c\'er\'ebral chez un patient%
%\begin{itemize}

%\end{itemize}
\item Am\'eliorer 
\item Calculer des grandeurs, comme la pression partielle en $O_2$, qui en sont pas mesurables directement \g{a} partir des images.
\end{enumerate}






\subsection{Principe de l'IRM}%State of the art, pour cette question plus générale que l'objet de ce stage.


%IRM : historique ? Jusqu'\g{a} l'IRM multiparam\'etrique.
%Avantages et limitations, bri\g{e}vement.





%\subsection{Une utilisation remarquable : L'IRM fonctionnelle (1991)}% spécifique au cerveau







\subsection{Mesures effectu\'ees \g{a} l'aide d'un agent de contraste}

%Pratiquée au GIN






%QUITOXIC MRI pour la CMRO2 et l'OEF.

\subsection{L'effet BOLD}

%Effet BOLD




%SO2

\cite{christen2012l}


%\subsection{Une alternative : la tomographie par \'emission de positrons (TEP)}




%Sc\'ema : grille de pixels et de cellules.
%Automate cellulaire.

%\section{R\'esultats}
\section{Introduction}

\begin{comment}
Mon stage s'est d\'eroul\'e au laboratoire TIMC-IMAG (Techniques de l'Ingénierie M\'edicale et de la Complexit\'e - %
Informatique, Mathématiques et Applications, Grenoble), sous la direction d'Angélique Stéphanou (CNRS). %
Il s'inscrit dans le projet Imagerie par Résonnance Magn\'etique Augment\'ee (IRMA) qui vise \g{a} effectuer des pr\'edictions, %
\g{a} partir d'images IRM, sur des param\g{e}tres qui ne sont pas directement mesurables comme la pression art\'erielle en dioxyg\g{e}ne $P_aO_2$ %
ou avec une meilleure r\'esolution spatiotemporelle que celle des images brutes. %
Un autre aspect de l'augmentation virtuelle de l'imagerie par résonnance magnétique est la simulation, \g{a} partir d'une image initiale, %
de l'évolution de l'état d'un patient sur des \'echelles de temps plus ou moins longues. %
C'est l'objet de ce travail, qui doit permettre de retracer l'\'evolution, sur plusieurs semaines, de l'\'etat d'un cerveau de rat isch\'emi\'e.
\end{comment}

Augmentation virtuelle de l'IRM : un projet du TIMC et du GIN

IRM augment\'ee : mettre au point des programmes informatiques permettant d'interpr\'eter syst\'ematiquement des images IRM.
\begin{enumerate}
\item Effectuer des pr\'edictions : simuler l'\'evolution future de l'\'etat du tissu c\'er\'ebral chez un patient%
%\begin{itemize}

%\end{itemize}
\item Am\'eliorer 
\item Calculer des grandeurs, comme la pression partielle en $O_2$, qui en sont pas mesurables directement \g{a} partir des images.
\end{enumerate}






\subsection{Principe de l'IRM}%State of the art, pour cette question plus générale que l'objet de ce stage.


%IRM : historique ? Jusqu'\g{a} l'IRM multiparam\'etrique.
%Avantages et limitations, bri\g{e}vement.





%\subsection{Une utilisation remarquable : L'IRM fonctionnelle (1991)}% spécifique au cerveau







\subsection{Mesures effectu\'ees \g{a} l'aide d'un agent de contraste}

%Pratiquée au GIN






%QUITOXIC MRI pour la CMRO2 et l'OEF.

\subsection{L'effet BOLD}

%Effet BOLD




%SO2

\cite{christen2012l}


%\subsection{Une alternative : la tomographie par \'emission de positrons (TEP)}




%Sc\'ema : grille de pixels et de cellules.
%Automate cellulaire.

%\section{Discussion}
%\section{Introduction}

\begin{comment}
Mon stage s'est d\'eroul\'e au laboratoire TIMC-IMAG (Techniques de l'Ingénierie M\'edicale et de la Complexit\'e - %
Informatique, Mathématiques et Applications, Grenoble), sous la direction d'Angélique Stéphanou (CNRS). %
Il s'inscrit dans le projet Imagerie par Résonnance Magn\'etique Augment\'ee (IRMA) qui vise \g{a} effectuer des pr\'edictions, %
\g{a} partir d'images IRM, sur des param\g{e}tres qui ne sont pas directement mesurables comme la pression art\'erielle en dioxyg\g{e}ne $P_aO_2$ %
ou avec une meilleure r\'esolution spatiotemporelle que celle des images brutes. %
Un autre aspect de l'augmentation virtuelle de l'imagerie par résonnance magnétique est la simulation, \g{a} partir d'une image initiale, %
de l'évolution de l'état d'un patient sur des \'echelles de temps plus ou moins longues. %
C'est l'objet de ce travail, qui doit permettre de retracer l'\'evolution, sur plusieurs semaines, de l'\'etat d'un cerveau de rat isch\'emi\'e.
\end{comment}

Augmentation virtuelle de l'IRM : un projet du TIMC et du GIN

IRM augment\'ee : mettre au point des programmes informatiques permettant d'interpr\'eter syst\'ematiquement des images IRM.
\begin{enumerate}
\item Effectuer des pr\'edictions : simuler l'\'evolution future de l'\'etat du tissu c\'er\'ebral chez un patient%
%\begin{itemize}

%\end{itemize}
\item Am\'eliorer 
\item Calculer des grandeurs, comme la pression partielle en $O_2$, qui en sont pas mesurables directement \g{a} partir des images.
\end{enumerate}






\subsection{Principe de l'IRM}%State of the art, pour cette question plus générale que l'objet de ce stage.


%IRM : historique ? Jusqu'\g{a} l'IRM multiparam\'etrique.
%Avantages et limitations, bri\g{e}vement.





%\subsection{Une utilisation remarquable : L'IRM fonctionnelle (1991)}% spécifique au cerveau







\subsection{Mesures effectu\'ees \g{a} l'aide d'un agent de contraste}

%Pratiquée au GIN






%QUITOXIC MRI pour la CMRO2 et l'OEF.

\subsection{L'effet BOLD}

%Effet BOLD




%SO2

\cite{christen2012l}


%\subsection{Une alternative : la tomographie par \'emission de positrons (TEP)}




%Sc\'ema : grille de pixels et de cellules.
%Automate cellulaire.
%\section{Annexes}


\newpage
\bibliography{Biblio}
\bibliographystyle{alpha}

\end{document}
