\section{M\'ethodes}

%%%% debut macro %%%%
\makeatletter
\renewcommand{\thefigure}{\ifnum \c@section>\z@ \thesection.\fi
 \@arabic\c@figure}
\@addtoreset{figure}{section}
\makeatother
%%%% fin macro %%%%
\begin{comment}
Remarque : pour renuméroter les sous-figures de la même manière
           (avec le package 'subfigure'), il suffit de rajouter
	   la ligne \let\p@subfigure\thefigure dans le préambule.
\end{comment}

\subsection{Protocole exp\'erimental : isch\'emie induite sur des rats par suture intraluminale}

Ces exp\'eriences ont \'et\'e men\'ees \g{a} l'Institut de Neurosciences de Grenoble (GIN) par Benjamin Lemasson, %
et m'ont \'e\'e transmises \g{a} la suite d'une r\'eunion \g{a} laquelle j'ai assist\'e au GIN %
avec Ang\'elique St\'ephanou, Emmanuel Barbier, et Benjamin Lemasson.

%\par
% Demander aux coll\g{e}gues du GIN.

%\subsubsection{Calcul de l'ADC}

\subsubsection{BVf, VSI}




\subsection{Traitement d'images avec ImageJ}

J'ai choisi de traiter pr\'ealablement les images r\'ealis\'ees en niveaux de gris par Benjamin Lemasson, \g{a} l'aide du logiciel ImageJ : %
c'est une collection d'instructions en Java qui permet de manipuler des images. Dans le cas de mon travail, il pr\'esente plusieurs avantages :
%
\begin{description}
\item[Travail sur des images individuelles] Une fonctionnalit\'e importante est la d\'elimitation de r\'egions d'int\'er\^et : %
enc\'ephale dans une coupe bidimensionnelle, zone l\'es\'ee caract\'eris\'ee, typiquement, par des valeurs basses d'ADC ou de CBF. %
%On cr\'ee ainsi des objets r\'eiutilisables
D'autres fonctionnalit\'es plus classiques sont disponobles : remplissage, rognage etc.
%\item[Cr\'eation d'objets graphiques]
%
\begin{comment}
\begin{tabular}{ccc}%width=0.2\linewidth,height=2cm
\includegraphics[width=0.2\linewidth,height=4cm]{methodes/11-J00-segADC-slice-10.jpg}%<+->
&
\includegraphics[width=0.2\linewidth,height=4cm]{methodes/11-J03-segADC-slice-10.jpg}%<+->
&
\includegraphics[width=0.2\linewidth,height=4cm]{methodes/11-J08-segADC-slice-10.jpg}%<+->%
\\
\includegraphics[width=0.2\linewidth,height=2cm]{methodes/11-J00-ADC-cropped-slice10.jpg}%<+->
&
\includegraphics[width=0.2\linewidth,height=2cm]{methodes/11-J03-ADC-cropped-slice10.jpg}%<+->
&
\includegraphics[width=0.2\linewidth,height=2cm]{methodes/11-J08-ADC-cropped-slice10.jpg}%<+->
\\
\end{tabular}
\end{comment}
%
\begin{figure}[H]
\begin{center}
\includegraphics[width=0.2\linewidth,height=4cm]{methodes/11-J00-segADC-slice-10.jpg}%<+->
\hfill
\includegraphics[width=0.2\linewidth,height=4cm]{methodes/11-J03-segADC-slice-10.jpg}%<+->
\hfill
\includegraphics[width=0.2\linewidth,height=4cm]{methodes/11-J08-segADC-slice-10.jpg}%<+->%
\end{center}
\caption{Coupe segment\'ee : rat 11, images en ADC sur trois jours}
\label{ex_cer}
\end{figure}
%
\begin{figure}[H]
\begin{center}
\includegraphics[width=0.2\linewidth,height=3cm]{methodes/11-J00-ADC-cropped-slice10.jpg}%<+->
\hfill
\includegraphics[width=0.2\linewidth,height=3cm]{methodes/11-J03-ADC-cropped-slice10.jpg}%<+->
\hfill
\includegraphics[width=0.2\linewidth,height=3cm]{methodes/11-J08-ADC-cropped-slice10.jpg}%<+->
\end{center}
\caption{L\'esion d\'elimit\'ee au jour 0 : faibles valeurs d'ADC. On r\'eutilise la r\'egion d'int\'er\^et pour les autres jours.}
\label{ex_les}
\end{figure}
%
\item[Exploitation de donn\'ees sous diff\'erents formats] On peut convertir des fichiers afin de les rendre exploitables par d'autres logiciels : %
images en .jpg pour une pr\'esentation, fichiers texte avec les coordonn\'ees et les valeurs, repr\'esent\'ees en niveaux de gris, des diff\'erents pixels.
%
\item[Traitement syst\'ematique avec des macros] Le logiciel permet de cr\'eer des macros afin de traiter syst\'ematiquement un grand nombre d'images : %
en effet la d\'elimitation de r\'egions d'int\'er\^et et leur exploitation est fastidieuse, et ces petits programmes permettent de traiter rapidement et avec fiabilit\'e des centaines d'images.
%
\begin{itemize}
\item
\item 
\end{itemize}
\end{description}










\subsection{Le paquet R Mclust : une id\'ee de Nicolas Glade (TIMC)}% On peut utiliser sweaver



S\'electionner des images de bonnes clusterisations.

\subsection{}% Utilisation de R : statistiques sur les cerveaux segment\'es, h\'emisph\g{e}res sains.

% Quelques exemples de diagrammesgrammes