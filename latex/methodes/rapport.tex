\section{M\'ethodes}

%%%% debut macro %%%%
\makeatletter
\renewcommand{\thefigure}{\ifnum \c@section>\z@ \thesection.\fi
 \@arabic\c@figure}
\@addtoreset{figure}{section}
\makeatother
%%%% fin macro %%%%
\begin{comment}
Remarque : pour renuméroter les sous-figures de la même manière
           (avec le package 'subfigure'), il suffit de rajouter
	   la ligne \let\p@subfigure\thefigure dans le préambule.
\end{comment}

\subsection{Protocole exp\'erimental : isch\'emie induite sur des rats par suture intraluminale}

Ces exp\'eriences ont \'et\'e men\'ees \g{a} l'Institut de Neurosciences de Grenoble (GIN) par Benjamin Lemasson, %
et m'ont \'e\'e transmises \g{a} la suite d'une r\'eunion \g{a} laquelle j'ai assist\'e au GIN %
avec Ang\'elique St\'ephanou, Emmanuel Barbier, et Benjamin Lemasson.



\subsubsection{Chirurgie et examens}

% Occlusion intraluminale.

% Sessions d'examen : jours 00, 03 etc.

%\par
Chaque session d'examens consistait \g{a} produire $13$ images, en coupe frontale, de la t\^ete du sujet, %
sur une longueur correspondant au cerveau du rat, et pour les modalit\'es qui suivent :

\subsubsection{Confection des images multiparam\'etriques, fiabilit\'e, probl\g{e}mes rencontr\'es}





\begin{figure}
\begin{tabular}{|c|c|c|c|}
\hline
\includegraphics[width=0.2\linewidth,height=3cm]{../../images_rapport/11-J03-Coreg01_Anat-masked-slice-10.jpg}
&
\includegraphics[width=0.2\linewidth,height=3cm]{../../images_rapport/11-J03-CoregADC-slice-10.jpg}
&
\includegraphics[width=0.2\linewidth,height=3cm]{../../images_rapport/11-J03-CoregBVf-slice-10.jpg}
&
\includegraphics[width=0.2\linewidth,height=3cm]{../../images_rapport/11-J03-CoregCBF-slice-10.jpg}
\\
\hline
\includegraphics[width=0.2\linewidth,height=3cm]{../../images_rapport/11-J03-CoregCMRO2-slice-10.jpg}
&
\includegraphics[width=0.2\linewidth,height=3cm]{../../images_rapport/11-J03-CoregSO2map-slice-10.jpg}
&
\includegraphics[width=0.2\linewidth,height=3cm]{../../images_rapport/11-J03-CoregT1map-slice-10.jpg}
&
\includegraphics[width=0.2\linewidth,height=3cm]{../../images_rapport/11-J03-CoregVSI-slice-10.jpg}
\\
\hline
\end{tabular}
\caption{Images disponibles pour le rat 11, au jour 03, tranche num\'ero 10.
\\%\par
De gauche \g{a} droite, et de haut en bas : %
l'image brute en T2 \og{} Anatomique\fg{}, l'ADC, le BVf, le CBF, la CMRO2, la SO2, l'image en T1 et le VSI.}
\label{ex_irm_multipar}
\end{figure}

\subsection{Traitement d'images avec ImageJ}

J'ai choisi de traiter pr\'ealablement les images r\'ealis\'ees en niveaux de gris par Benjamin Lemasson, \g{a} l'aide du logiciel ImageJ : %
c'est un logiciel \'ecrit en Java qui permet de manipuler des images.
\begin{comment}% Format à utiliser pour la présentation, pertinent pour le rapport ?
% Dans le cas de mon travail, il pr\'esente plusieurs avantages :
%
\begin{description}
\item[Travail sur des images individuelles] Une fonctionnalit\'e importante est la d\'elimitation de r\'egions d'int\'er\^et : %
enc\'ephale dans une coupe bidimensionnelle, zone l\'es\'ee caract\'eris\'ee, typiquement, par des valeurs basses d'ADC ou de CBF. %
%On cr\'ee ainsi des objets r\'eiutilisables
D'autres fonctionnalit\'es plus classiques sont disponobles : remplissage, rognage etc.
%\item[Cr\'eation d'objets graphiques]
%
\begin{figure}[H]
\begin{center}
\includegraphics[width=0.2\linewidth,height=4cm]{methodes/11-J00-segADC-slice-10.jpg}%<+->
\hfill
\includegraphics[width=0.2\linewidth,height=4cm]{methodes/11-J03-segADC-slice-10.jpg}%<+->
\hfill
\includegraphics[width=0.2\linewidth,height=4cm]{methodes/11-J08-segADC-slice-10.jpg}%<+->%
\end{center}
\caption{Coupe segment\'ee : rat 11, images en ADC sur trois jours}
\label{ex_cer}
\end{figure}
%
\begin{figure}[H]
\begin{center}
\includegraphics[width=0.2\linewidth,height=3cm]{methodes/11-J00-ADC-cropped-slice10.jpg}%<+->
\hfill
\includegraphics[width=0.2\linewidth,height=3cm]{methodes/11-J03-ADC-cropped-slice10.jpg}%<+->
\hfill
\includegraphics[width=0.2\linewidth,height=3cm]{methodes/11-J08-ADC-cropped-slice10.jpg}%<+->
\end{center}
\caption{L\'esion d\'elimit\'ee au jour 0 : faibles valeurs de l'ADC. On r\'eutilise la r\'egion d'int\'er\^et pour les autres jours.}
\label{ex_les}
\end{figure}
%
\item[Exploitation de donn\'ees sous diff\'erents formats] On peut convertir des fichiers afin de les rendre exploitables par d'autres logiciels : %
images en .jpg pour une pr\'esentation, fichiers texte avec les coordonn\'ees et les valeurs, repr\'esent\'ees en niveaux de gris, des diff\'erents pixels.
%
\item[Traitement syst\'ematique avec des macros] Le logiciel permet de cr\'eer des macros afin de traiter syst\'ematiquement un grand nombre d'images : %
en effet la d\'elimitation de r\'egions d'int\'er\^et et leur exploitation est fastidieuse, et ces petits programmes permettent de traiter rapidement et avec fiabilit\'e des centaines d'images.
%
\begin{itemize}
\item
\item 
\end{itemize}
\end{description}
\end{comment}
%
\par
A partir des images, au format .tif, transmises par Benjamin Lemasson, %
j'ai utilis\'e l'outil ImageJ pour d\'elimiter les r\'egions correspondant au cerveau, pour toutes les modalit\'es -voir figure \ref{cephcer}.
\par
Ma d\'emarche a \'et\'e la suivante :
\begin{enumerate}
\item Tracer manuellement le contour des images anatomiques, prises en T2, pr\'ealablement masqu\'ees par Benjamin et dont on voit un exemple \g{a} gauche de la figure \ref{cephcer} ;
\item Charger individuellement ces contours sur les images correspondant aux autres modalit\'es, ici l'ADC : les deux images centrales de la figure ;
\item Retirer toutes les valeurs correspondant \g{a} l'ext\'ertieur du contour : voir l'image \g{a} droite de \ref{cephcer} ;
\item Automatiser la proc\'edure pour traiter et trier, \g{a} partir des contours d\'efinis manuellement, toutes les images disponibles.
\end{enumerate}

J'ai ainsi obtenu des images, par tranche

\begin{figure}%Ajouter les colonnes correspondant à anatomique, ADC sans contour ce ADC segmentée.
\includegraphics[width=0.2\linewidth,height=4cm]{../../images_rapport/11-J03-Coreg01_Anat-masked-slice-10.jpg}
\hfill
\includegraphics[width=0.2\linewidth,height=4cm]{../../images_rapport/11-J03-CoregADC-slice-10.jpg}
\hfill
\includegraphics[width=0.2\linewidth,height=4cm]{../../images_rapport/11-J03-segADC-slice-10.jpg}
\hfill
\includegraphics[width=0.2\linewidth,height=4cm]{../../images_rapport/11-J03-ADC-bg-slice10.jpg}
\caption{D\'elimitation du cerveau du rat num\'ero 11, jour 03, avec la modalit\'e ADC. On utilise les images anatomiques prises en ...}% en T2 ?
\label{cephcer}
\end{figure}





\begin{comment}
\caption{IRM multiparam\'etrique : on distingue, sur les images ADC, CBF et T1map une zone sombre dans l'\'emisph\g{e}re gauche.
\\
Ces r\'egions varient peu suivant la modalit\'e, et sont quasiment identiques pour l'ADC et en T1. La zone d'int\'er\^et du CBF, %
l\'eg\g{e}rement plus grande et qui correspond \g{a} un crit\g{e}re physiologique d'insch\'emie, sera retenue.}
\ref{lesion_R11_J00}
\end{comment}

Les images aux tranches : 10 pour le rat num\'ero 11, 9 et 10 pour le rat 19, 6 \g{a} 8 pour le rat 26 et 11 \g{a} 13 pour le rat 30 sont disponibles pour toutes les modalit\'es, %
pour tous les jours d'examens. %
Ces tranches sont donc utilisables pour effectuer un suivi temporel des valeurs mesur\'ees ou calcul\'ees, pixel par pixel, ou globalement sur les tranches enti\g{e}res ou des r\'egions d'int\'er\^et.

\par
Avec des choix judicieux de contraste, on peut discerner, \g{a} l'oeil, des r\'egions l\'es\'ees sur les tranchse segment\'ees de cerveau. %
Plus particuli\g{e}rement, et pour tous les rats, on discerne une aire sombre sur les images en CBF au jour 00, c'est-\g{a}-dire produites par l'examen r\'ealis\'e 30 minutes apr\g{e}s la reperfusion. %
Cette aire se situe sur l'h\'emisph\g{e}re droit, et on retrouve des r\'egions similaires, mais plus petites, sur des images obtenues dans les autres modalit\'es : %
voir les figures \ref{11_dark_00}, \ref{19_dark_00}, \ref{26_dark_00} et \ref{30_dark_00}. %
L'isch\'emie focale transitoire se traduit pr\'ecis\'ement par une baisse du d\'ebit sanguin c\'er\'ebral %et la reperfusion ?
L'utilisation de cette r\'egion d'int\'er\^et est donc pertinente pour mod\'eliser l'ensemble des pixels initialement l\'es\'es.

\begin{figure}
\begin{tabular}{|c|c|c|c|}
\hline
\subfloat[Anatomique]{\includegraphics[width=0.2\linewidth,height=2cm]{../../images_rapport/11-J00-Coreg01_Anat-masked-Cropped-slice10.jpg}}
&
\subfloat[ADC]{\includegraphics[width=0.2\linewidth,height=2cm]{../../images_rapport/11-J00-ADC-Cropped-slice10.jpg}}
&
\subfloat[BVf]{\includegraphics[width=0.2\linewidth,height=2cm]{../../images_rapport/11-J00-BVf-Cropped-slice10.jpg}}
&
\subfloat[CBF]{\includegraphics[width=0.2\linewidth,height=2cm]{../../images_rapport/11-J00-CBF-Cropped-slice10.jpg}}
\\
\hline
\subfloat[CMRO2]{\includegraphics[width=0.2\linewidth,height=2cm]{../../images_rapport/11-J00-CMRO2-Cropped-slice10.jpg}}
&
\subfloat[SO2map]{\includegraphics[width=0.2\linewidth,height=2cm]{../../images_rapport/11-J00-SO2map-Cropped-slice10.jpg}}
&
\subfloat[T1map]{\includegraphics[width=0.2\linewidth,height=2cm]{../../images_rapport/11-J00-T1map-Cropped-slice10.jpg}}
&
\subfloat[VSI]{\includegraphics[width=0.2\linewidth,height=2cm]{../../images_rapport/11-J00-VSI-Cropped-slice10.jpg}}
\\
\hline
\end{tabular}
\caption{Rat 11, jour 00, tranche 10}
\label{11_dark_00}
\end{figure}

\begin{figure}
\begin{tabular}{|c|c|c|c|}
\hline
\subfloat[Anatomique]{\includegraphics[width=0.2\linewidth,height=2cm]{../../images_rapport/19-J00-Coreg01_Anat-masked-Cropped-slice9.jpg}}
&
\subfloat[ADC]{\includegraphics[width=0.2\linewidth,height=2cm]{../../images_rapport/19-J00-ADC-Cropped-slice9.jpg}}
&
\subfloat[BVf]{\includegraphics[width=0.2\linewidth,height=2cm]{../../images_rapport/19-J00-BVf-Cropped-slice9.jpg}}
&
\subfloat[CBF]{\includegraphics[width=0.2\linewidth,height=2cm]{../../images_rapport/19-J00-CBF-Cropped-slice9.jpg}}
\\
\hline
\subfloat[CMRO2]{\includegraphics[width=0.2\linewidth,height=2cm]{../../images_rapport/19-J00-CMRO2-Cropped-slice9.jpg}}
&
\subfloat[SO2map]{\includegraphics[width=0.2\linewidth,height=2cm]{../../images_rapport/19-J00-SO2map-Cropped-slice9.jpg}}
&
\subfloat[T1map]{\includegraphics[width=0.2\linewidth,height=2cm]{../../images_rapport/19-J00-T1map-Cropped-slice9.jpg}}
&
\subfloat[VSI]{\includegraphics[width=0.2\linewidth,height=2cm]{../../images_rapport/19-J00-VSI-Cropped-slice9.jpg}}
\\
\hline
\end{tabular}
\caption{Rat 19, jour 00, tranche 9}
\label{19_dark_00}
\end{figure}

\begin{figure}
\begin{tabular}{|c|c|c|c|}
\hline
\subfloat[Anatomique]{\includegraphics[width=0.2\linewidth,height=2cm]{../../images_rapport/26-J00-Coreg01_Anat-masked-Cropped-slice8.jpg}}
&
\subfloat[ADC]{\includegraphics[width=0.2\linewidth,height=2cm]{../../images_rapport/26-J00-ADC-Cropped-slice8.jpg}}
&
\subfloat[BVf]{\includegraphics[width=0.2\linewidth,height=2cm]{../../images_rapport/26-J00-BVf-Cropped-slice8.jpg}}
&
\subfloat[CBF]{\includegraphics[width=0.2\linewidth,height=2cm]{../../images_rapport/26-J00-CBF-Cropped-slice8.jpg}}
\\
\hline
\subfloat[CMRO2]{\includegraphics[width=0.2\linewidth,height=2cm]{../../images_rapport/26-J00-CMRO2-Cropped-slice8.jpg}}
&
\subfloat[SO2map]{\includegraphics[width=0.2\linewidth,height=2cm]{../../images_rapport/26-J00-SO2map-Cropped-slice8.jpg}}
&
\subfloat[T1map]{\includegraphics[width=0.2\linewidth,height=2cm]{../../images_rapport/26-J00-T1map-Cropped-slice8.jpg}}
&
\subfloat[VSI]{\includegraphics[width=0.2\linewidth,height=2cm]{../../images_rapport/26-J00-VSI-Cropped-slice8.jpg}}
\\
\hline
\end{tabular}
\caption{Rat 26, jour 00, tranche 8}
\label{26_dark_00}
\end{figure}

\begin{figure}
\begin{tabular}{|c|c|c|c|}
\hline
\subfloat[Anatomique]{\includegraphics[width=0.2\linewidth,height=2cm]{../../images_rapport/30-J00-Coreg01_Anat-masked-Cropped-slice11.jpg}}
&
\subfloat[ADC]{\includegraphics[width=0.2\linewidth,height=2cm]{../../images_rapport/30-J00-ADC-Cropped-slice11.jpg}}
&
\subfloat[BVf]{\includegraphics[width=0.2\linewidth,height=2cm]{../../images_rapport/30-J00-BVf-Cropped-slice11.jpg}}
&
\subfloat[CBF]{\includegraphics[width=0.2\linewidth,height=2cm]{../../images_rapport/30-J00-CBF-Cropped-slice11.jpg}}
\\
\hline
\subfloat[CMRO2]{\includegraphics[width=0.2\linewidth,height=2cm]{../../images_rapport/30-J00-CMRO2-Cropped-slice11.jpg}}
&
\subfloat[SO2map]{\includegraphics[width=0.2\linewidth,height=2cm]{../../images_rapport/30-J00-SO2map-Cropped-slice11.jpg}}
&
\subfloat[T1map]{\includegraphics[width=0.2\linewidth,height=2cm]{../../images_rapport/30-J00-T1map-Cropped-slice11.jpg}}
&
\subfloat[VSI]{\includegraphics[width=0.2\linewidth,height=2cm]{../../images_rapport/30-J00-VSI-Cropped-slice11.jpg}}
\\
\hline
\end{tabular}
\caption{Rat 30, jour 00, tranche 11}
\label{30_dark_00}
\end{figure}

J'ai utilis\'e une technique similaire \g{a} celle utilis\'ee pour d\'elimiter les cerveaux afin de s\'electionner les r\'egions l\'es\'ees en CBF, %
et de les convertir dans des fichiers texte exploitables sous R.

\subsection{Traitement statistique avec R}% Utilisation de R : statistiques sur les cerveaux segment\'es, h\'emisph\g{e}res sains.

% Quelques exemples de diagrammes


% Comparaison entre les deux segmentation, éventuellement dans le section résultats : peu de différences entre ADC et CBF.





\subsection{Le paquet R Mclust : une id\'ee de Nicolas Glade (TIMC)}% On peut utiliser sweaver

% Genèse de l'idée.


S\'electionner des images de bonnes clusterisations.



\subsection{}