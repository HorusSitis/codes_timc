\section{M\'ethodes}

%%%% debut macro %%%%
\makeatletter
\renewcommand{\thefigure}{\ifnum \c@section>\z@ \thesection.\fi
 \@arabic\c@figure}
\@addtoreset{figure}{section}
\makeatother
%%%% fin macro %%%%
\begin{comment}
Remarque : pour renuméroter les sous-figures de la même manière
           (avec le package 'subfigure'), il suffit de rajouter
	   la ligne \let\p@subfigure\thefigure dans le préambule.
\end{comment}

\subsection{Protocole exp\'erimental : isch\'emie focale transitoire induite sur des rats par suture intraluminale}

Ces exp\'eriences ont \'et\'e men\'ees \g{a} l'Institut de Neurosciences de Grenoble (GIN) par Benjamin Lemasson, %
et m'ont \'e\'e transmises \g{a} la suite d'une r\'eunion \g{a} laquelle j'ai assist\'e au GIN %
avec Ang\'elique St\'ephanou, Emmanuel Barbier, et Benjamin Lemasson.



\subsubsection{Chirurgie et examens}

% Occlusion intraluminale.

% Sessions d'examen : jours 00, 03 etc.

%\par
Chaque session d'examens consistait \g{a} produire $13$ images, en coupe frontale, de la t\^ete du sujet, %
sur une longueur correspondant au cerveau du rat, et pour les modalit\'es qui suivent :

\subsubsection{Confection des images multiparam\'etriques, fiabilit\'e, probl\g{e}mes rencontr\'es}





\begin{figure}[!h]
\begin{tabular}{|c|c|c|c|}
\hline
\includegraphics[width=0.2\linewidth,height=3cm]{../../images_rapport/11-J03-Coreg01_Anat-masked-slice-10.jpg}
&
\includegraphics[width=0.2\linewidth,height=3cm]{../../images_rapport/11-J03-CoregADC-slice-10.jpg}
&
\includegraphics[width=0.2\linewidth,height=3cm]{../../images_rapport/11-J03-CoregBVf-slice-10.jpg}
&
\includegraphics[width=0.2\linewidth,height=3cm]{../../images_rapport/11-J03-CoregCBF-slice-10.jpg}
\\
\hline
\includegraphics[width=0.2\linewidth,height=3cm]{../../images_rapport/11-J03-CoregCMRO2-slice-10.jpg}
&
\includegraphics[width=0.2\linewidth,height=3cm]{../../images_rapport/11-J03-CoregSO2map-slice-10.jpg}
&
\includegraphics[width=0.2\linewidth,height=3cm]{../../images_rapport/11-J03-CoregT1map-slice-10.jpg}
&
\includegraphics[width=0.2\linewidth,height=3cm]{../../images_rapport/11-J03-CoregVSI-slice-10.jpg}
\\
\hline
\end{tabular}
\caption{Images disponibles pour le rat 11, au jour 03, tranche num\'ero 10.
\\%\par
De gauche \g{a} droite, et de haut en bas : %
l'image brute en T2 \og{} Anatomique\fg{}, l'ADC, le BVf, le CBF, la CMRO2, la SO2, l'image en T1 et le VSI.}
\label{ex_irm_multipar}
\end{figure}

\FloatBarrier
\subsection{Traitement d'images avec ImageJ}

J'ai choisi de traiter pr\'ealablement les images r\'ealis\'ees en niveaux de gris par Benjamin Lemasson, \g{a} l'aide du logiciel ImageJ : %
c'est un logiciel \'ecrit en Java qui permet de manipuler des images.
%
\par
A partir des images, au format .tif, transmises par Benjamin Lemasson, %
j'ai utilis\'e l'outil ImageJ pour d\'elimiter les r\'egions correspondant au cerveau, pour toutes les modalit\'es -voir figure \ref{cephcer}.
\par
Ma d\'emarche a \'et\'e la suivante :
\begin{enumerate}
\item Tracer manuellement le contour des images anatomiques, prises en T2, pr\'ealablement masqu\'ees par Benjamin et dont on voit un exemple \g{a} gauche de la figure \ref{cephcer} ;
\item Charger individuellement ces contours sur les images correspondant aux autres modalit\'es, ici l'ADC : les deux images centrales de la figure ;
\item Retirer toutes les valeurs correspondant \g{a} l'ext\'ertieur du contour : voir l'image \g{a} droite de \ref{cephcer} ;
\item Automatiser la proc\'edure pour traiter et trier, \g{a} partir des contours d\'efinis manuellement, toutes les images disponibles.
\end{enumerate}

\newcounter{stock}
\setcounter{stock}{\value{enumi}}

\begin{figure}[!h]%Ajouter les colonnes correspondant à anatomique, ADC sans contour ce ADC segmentée.
\includegraphics[width=0.2\linewidth,height=4cm]{../../images_rapport/11-J03-Coreg01_Anat-masked-slice-10.jpg}
\hfill
\includegraphics[width=0.2\linewidth,height=4cm]{../../images_rapport/11-J03-CoregADC-slice-10.jpg}
\hfill
\includegraphics[width=0.2\linewidth,height=4cm]{../../images_rapport/11-J03-segADC-slice-10.jpg}
\hfill
\includegraphics[width=0.2\linewidth,height=4cm]{../../images_rapport/11-J03-ADC-bg-slice10.jpg}
\caption{D\'elimitation du cerveau du rat num\'ero 11, jour 03, avec la modalit\'e ADC. On utilise les images anatomiques prises en ...}% en T2 ?
\label{cephcer}
\end{figure}

\FloatBarrier
J'ai ainsi obtenu des images, par tranche examin\'ee, pour toutes les modalit\'es, restreintes \g{a} l'enc\'ephale. L'\'etape finale pour l'exploitation des donn\'ees IRM a \'et\'e :
\begin{enumerate}
\setcounter{enumi}{\value{stock}}
\item Regrouper les niveaux de gris, qui correspondent aux valeurs mesur\'ees ou calcul\'ees de chacune des images, dans un fichier texte exploitable sous R.
\end{enumerate}

\etoile
Les images aux tranches : 10 pour le rat num\'ero 11, 9 et 10 pour le rat 19, 6 \g{a} 8 pour le rat 26 et 11 \g{a} 13 pour le rat 30 sont disponibles pour toutes les modalit\'es, %
pour tous les jours d'examens. %
Ces tranches sont donc utilisables pour effectuer un suivi temporel des valeurs mesur\'ees ou calcul\'ees, pixel par pixel, %
ou globalement sur les tranches enti\g{e}res ou des r\'egions d'int\'er\^et.

\begin{figure}[!p]
\begin{center}
\begin{tabular}{|c|c|c|c|}
\hline
\subfloat[Anatomique]{\includegraphics[width=0.2\linewidth,height=2cm]{../../images_rapport/11-J00-Coreg01_Anat-masked-Cropped-slice10.jpg}}
&
\subfloat[ADC]{\includegraphics[width=0.2\linewidth,height=2cm]{../../images_rapport/11-J00-ADC-Cropped-slice10.jpg}}
&
\subfloat[BVf]{\includegraphics[width=0.2\linewidth,height=2cm]{../../images_rapport/11-J00-BVf-Cropped-slice10.jpg}}
&
\subfloat[CBF]{\includegraphics[width=0.2\linewidth,height=2cm]{../../images_rapport/11-J00-CBF-Cropped-slice10.jpg}}
\\
\hline
\subfloat[CMRO2]{\includegraphics[width=0.2\linewidth,height=2cm]{../../images_rapport/11-J00-CMRO2-Cropped-slice10.jpg}}
&
\subfloat[SO2map]{\includegraphics[width=0.2\linewidth,height=2cm]{../../images_rapport/11-J00-SO2map-Cropped-slice10.jpg}}
&
\subfloat[T1map]{\includegraphics[width=0.2\linewidth,height=2cm]{../../images_rapport/11-J00-T1map-Cropped-slice10.jpg}}
&
\subfloat[VSI]{\includegraphics[width=0.2\linewidth,height=2cm]{../../images_rapport/11-J00-VSI-Cropped-slice10.jpg}}
\\
\hline
\end{tabular}
\end{center}
\caption{Rat 11, jour 00, tranche 10}
\label{11_dark_00}
\end{figure}

\begin{figure}[!p]
\begin{center}
\begin{tabular}{|c|c|c|c|}
\hline
\subfloat[Anatomique]{\includegraphics[width=0.2\linewidth,height=2cm]{../../images_rapport/19-J00-Coreg01_Anat-masked-Cropped-slice9.jpg}}
&
\subfloat[ADC]{\includegraphics[width=0.2\linewidth,height=2cm]{../../images_rapport/19-J00-ADC-Cropped-slice9.jpg}}
&
\subfloat[BVf]{\includegraphics[width=0.2\linewidth,height=2cm]{../../images_rapport/19-J00-BVf-Cropped-slice9.jpg}}
&
\subfloat[CBF]{\includegraphics[width=0.2\linewidth,height=2cm]{../../images_rapport/19-J00-CBF-Cropped-slice9.jpg}}
\\
\hline
\subfloat[CMRO2]{\includegraphics[width=0.2\linewidth,height=2cm]{../../images_rapport/19-J00-CMRO2-Cropped-slice9.jpg}}
&
\subfloat[SO2map]{\includegraphics[width=0.2\linewidth,height=2cm]{../../images_rapport/19-J00-SO2map-Cropped-slice9.jpg}}
&
\subfloat[T1map]{\includegraphics[width=0.2\linewidth,height=2cm]{../../images_rapport/19-J00-T1map-Cropped-slice9.jpg}}
&
\subfloat[VSI]{\includegraphics[width=0.2\linewidth,height=2cm]{../../images_rapport/19-J00-VSI-Cropped-slice9.jpg}}
\\
\hline
\end{tabular}
\end{center}
\caption{Rat 19, jour 00, tranche 9}
\label{19_dark_00}
\end{figure}

\begin{figure}[!p]
\begin{center}
\begin{tabular}{|c|c|c|c|}
\hline
\subfloat[Anatomique]{\includegraphics[width=0.2\linewidth,height=2cm]{../../images_rapport/26-J00-Coreg01_Anat-masked-Cropped-slice8.jpg}}
&
\subfloat[ADC]{\includegraphics[width=0.2\linewidth,height=2cm]{../../images_rapport/26-J00-ADC-Cropped-slice8.jpg}}
&
\subfloat[BVf]{\includegraphics[width=0.2\linewidth,height=2cm]{../../images_rapport/26-J00-BVf-Cropped-slice8.jpg}}
&
\subfloat[CBF]{\includegraphics[width=0.2\linewidth,height=2cm]{../../images_rapport/26-J00-CBF-Cropped-slice8.jpg}}
\\
\hline
\subfloat[CMRO2]{\includegraphics[width=0.2\linewidth,height=2cm]{../../images_rapport/26-J00-CMRO2-Cropped-slice8.jpg}}
&
\subfloat[SO2map]{\includegraphics[width=0.2\linewidth,height=2cm]{../../images_rapport/26-J00-SO2map-Cropped-slice8.jpg}}
&
\subfloat[T1map]{\includegraphics[width=0.2\linewidth,height=2cm]{../../images_rapport/26-J00-T1map-Cropped-slice8.jpg}}
&
\subfloat[VSI]{\includegraphics[width=0.2\linewidth,height=2cm]{../../images_rapport/26-J00-VSI-Cropped-slice8.jpg}}
\\
\hline
\end{tabular}
\end{center}
\caption{Rat 26, jour 00, tranche 8}
\label{26_dark_00}
\end{figure}

\begin{figure}[!p]
\begin{center}
\begin{tabular}{|c|c|c|c|}
\hline
\subfloat[Anatomique]{\includegraphics[width=0.2\linewidth,height=2cm]{../../images_rapport/30-J00-Coreg01_Anat-masked-Cropped-slice11.jpg}}
&
\subfloat[ADC]{\includegraphics[width=0.2\linewidth,height=2cm]{../../images_rapport/30-J00-ADC-Cropped-slice11.jpg}}
&
\subfloat[BVf]{\includegraphics[width=0.2\linewidth,height=2cm]{../../images_rapport/30-J00-BVf-Cropped-slice11.jpg}}
&
\subfloat[CBF]{\includegraphics[width=0.2\linewidth,height=2cm]{../../images_rapport/30-J00-CBF-Cropped-slice11.jpg}}
\\
\hline
\subfloat[CMRO2]{\includegraphics[width=0.2\linewidth,height=2cm]{../../images_rapport/30-J00-CMRO2-Cropped-slice11.jpg}}
&
\subfloat[SO2map]{\includegraphics[width=0.2\linewidth,height=2cm]{../../images_rapport/30-J00-SO2map-Cropped-slice11.jpg}}
&
\subfloat[T1map]{\includegraphics[width=0.2\linewidth,height=2cm]{../../images_rapport/30-J00-T1map-Cropped-slice11.jpg}}
&
\subfloat[VSI]{\includegraphics[width=0.2\linewidth,height=2cm]{../../images_rapport/30-J00-VSI-Cropped-slice11.jpg}}
\\
\hline
\end{tabular}
\end{center}
\caption{Rat 30, jour 00, tranche 11}
\label{30_dark_00}
\end{figure}

%\FloatBarrier
\par
Avec des choix judicieux de contraste, on peut discerner, \g{a} l'oeil, des r\'egions l\'es\'ees sur les tranches segment\'ees de cerveau. %
Plus particuli\g{e}rement, et pour tous les rats, on discerne une aire sombre sur les images en CBF au jour 00, %
c'est-\g{a}-dire produites par l'examen r\'ealis\'e 30 minutes apr\g{e}s la reperfusion. %
Cette aire se situe sur l'h\'emisph\g{e}re visible \g{a} gauche, et on retrouve des r\'egions similaires, mais plus petites, %
sur des images obtenues dans les autres modalit\'es, notamment l'ADC : %
voir les figures \ref{11_dark_00}, \ref{19_dark_00}, \ref{26_dark_00} et \ref{30_dark_00}. %
L'isch\'emie focale transitoire se traduit pr\'ecis\'ement par une baisse du d\'ebit sanguin c\'er\'ebral %et la reperfusion ?
L'utilisation de cette r\'egion d'int\'er\^et est donc pertinente pour mod\'eliser l'ensemble des pixels initialement l\'es\'es.

\etoile
J'ai utilis\'e une technique similaire \g{a} celle d\'etaill\'ee ci-dessus pour la segmentation des cerveaux -voir la figure \ref{cbf_seg_19}, %
afin de s\'electionner les r\'egions l\'es\'ees en CBF, et de les convertir dans des fichiers texte exploitables sous R.

\par
Le tableau ci-dessous r\'ecapitule la taille, en nombre de pixels, des l\'esions ainsi d\'elimit\'ees sur les tranches suivables temporellement.

\begin{multicols}{2}
\begin{tabular}{|c|c|c|c|c|c|c|c|c|}
\hline
\small{Tranche}&6&7&8&9&10&11&12&13
\\
\hline
Rat 11&&&&&813&&&
\\
\hline
Rat 19&&&&742&654&&&
\\
\hline
Rat 26&772&791&558&&&&&
\\
\hline
Rat 30&&&&&&88&139&83
\\
\hline
\end{tabular}

\columnbreak
\begin{figure}[H]
\begin{center}
\includegraphics[width=0.45\linewidth, height=3cm]{../../images_rapport/19-J00-segCBF-slice9.jpg}
\end{center}
\caption{D\'elimitation d'un contour : rat 19, tranche 9. On utilise une image du CBF.}
\label{cbf_seg_19}
\end{figure}
\end{multicols}



\newpage
\FloatBarrier
\subsection{Traitement statistique avec R}% Utilisation de R : statistiques sur les cerveaux segment\'es, h\'emisph\g{e}res sains.

Encourag\'e par Ang\'elique St\'ephanou et par Nicolas Glade, j'ai utilis\'ee les donn\'ees ainsi collect\'ees pour observer l'\'evolution, au cours de l'exp\'erience, %
des diff\'erentes modalit\'es.

\par
Les diagrammes \ref{11_box_lch00}, \ref{19_box_lch00}, \ref{26_box_lch00} et \ref{30_box_lch00} permettent de lire facilement la r\'epartition des valeurs des diff\'erentes modalit\'es : ADC, BVf, CBF, CMRO2, SO2map et VSI. %
Ces valeurs sont compar\'ees avec celles mesur\'ees sur le cerveau entier, et sur l'h\'emisph\g{e}re contralat\'eral -non desservi par l'art\g{e}re op\'er\'ee- le jour 0.

\par
Notons au passage que ce dernier choix ne revient pas \g{a} comparer avec un tissu sain : bien que ne subissant pas de dysfonctionnement significatif, %
l'\'emisph\^g{e}re \og{}sain\fg{} subit \'egalement, notamment le permier jour, des r\'epercussions de l'intervention ou de la reperfusion. %
L'\'evolution de l'\'etat de l'h\'emisph\g{e}re contralat\'eral est visible \g{a} la figure \ref{} et sera incorpor\'e au mod\g{e}le %\ref{}.

%\ligneinter

\begin{figure}[!p]
\begin{center}
\begin{tabular}{|c|c|c|}
\hline
\subfloat[ADC]{\includegraphics[width=0.3\linewidth,height=5cm]{../../images_rapport/11_suivi_box_volCBFdark00_ADC.pdf}}
&%
\subfloat[BVf]{\includegraphics[width=0.3\linewidth,height=5cm]{../../images_rapport/11_suivi_box_volCBFdark00_BVf.pdf}}
&%
\subfloat[CBF]{\includegraphics[width=0.3\linewidth,height=5cm]{../../images_rapport/11_suivi_box_volCBFdark00_CBF.pdf}}
\\
\hline
\subfloat[CMRO2]{\includegraphics[width=0.3\linewidth,height=5cm]{../../images_rapport/11_suivi_box_volCBFdark00_CMRO2.pdf}}
&%
\subfloat[SO2map]{\includegraphics[width=0.3\linewidth,height=5cm]{../../images_rapport/11_suivi_box_volCBFdark00_SO2map.pdf}}
&%
\subfloat[VSI]{\includegraphics[width=0.3\linewidth,height=5cm]{../../images_rapport/11_suivi_box_volCBFdark00_VSI.pdf}}
\\
\hline
\end{tabular}
\end{center}
\caption{Rat num\'ero 11 : \'evolution par jour d'examen}
\label{11_box_lch00}
\end{figure}

\begin{figure}[!p]
\begin{center}
\begin{tabular}{|c|c|c|}
\hline
\subfloat[ADC]{\includegraphics[width=0.3\linewidth,height=5cm]{../../images_rapport/19_suivi_box_volCBFdark00_ADC.pdf}}
&%
\subfloat[BVf]{\includegraphics[width=0.3\linewidth,height=5cm]{../../images_rapport/19_suivi_box_volCBFdark00_BVf.pdf}}
&%
\subfloat[CBF]{\includegraphics[width=0.3\linewidth,height=5cm]{../../images_rapport/19_suivi_box_volCBFdark00_CBF.pdf}}
\\
\hline
\subfloat[CMRO2]{\includegraphics[width=0.3\linewidth,height=5cm]{../../images_rapport/19_suivi_box_volCBFdark00_CMRO2.pdf}}
&%
\subfloat[SO2map]{\includegraphics[width=0.3\linewidth,height=5cm]{../../images_rapport/19_suivi_box_volCBFdark00_SO2map.pdf}}
&%
\subfloat[VSI]{\includegraphics[width=0.3\linewidth,height=5cm]{../../images_rapport/19_suivi_box_volCBFdark00_VSI.pdf}}
\\
\hline
\end{tabular}
\end{center}
\caption{Rat num\'ero 19 : \'evolution par jour d'examen}
\label{19_box_lch00}
\end{figure}

\begin{figure}[!p]
\begin{center}
\begin{tabular}{|c|c|c|}
\hline
\subfloat[ADC]{\includegraphics[width=0.3\linewidth,height=5cm]{../../images_rapport/26_suivi_box_volCBFdark00_ADC.pdf}}
&%
\subfloat[BVf]{\includegraphics[width=0.3\linewidth,height=5cm]{../../images_rapport/26_suivi_box_volCBFdark00_BVf.pdf}}
&%
\subfloat[CBF]{\includegraphics[width=0.3\linewidth,height=5cm]{../../images_rapport/26_suivi_box_volCBFdark00_CBF.pdf}}
\\
\hline
\subfloat[CMRO2]{\includegraphics[width=0.3\linewidth,height=5cm]{../../images_rapport/26_suivi_box_volCBFdark00_CMRO2.pdf}}
&%
\subfloat[SO2map]{\includegraphics[width=0.3\linewidth,height=5cm]{../../images_rapport/26_suivi_box_volCBFdark00_SO2map.pdf}}
&%
\subfloat[VSI]{\includegraphics[width=0.3\linewidth,height=5cm]{../../images_rapport/26_suivi_box_volCBFdark00_VSI.pdf}}
\\
\hline
\end{tabular}
\end{center}
\caption{Rat num\'ero 26 : \'evolution par jour d'examen}
\label{26_box_lch00}
\end{figure}

\begin{figure}[!p]
\begin{center}
\begin{tabular}{|c|c|c|}
\hline
\subfloat[ADC]{\includegraphics[width=0.3\linewidth,height=5cm]{../../images_rapport/30_suivi_box_volCBFdark00_ADC.pdf}}
&%
\subfloat[BVf]{\includegraphics[width=0.3\linewidth,height=5cm]{../../images_rapport/30_suivi_box_volCBFdark00_BVf.pdf}}
&%
\subfloat[CBF]{\includegraphics[width=0.3\linewidth,height=5cm]{../../images_rapport/30_suivi_box_volCBFdark00_CBF.pdf}}
\\
\hline
\subfloat[CMRO2]{\includegraphics[width=0.3\linewidth,height=5cm]{../../images_rapport/30_suivi_box_volCBFdark00_CMRO2.pdf}}
&%
\subfloat[SO2map]{\includegraphics[width=0.3\linewidth,height=5cm]{../../images_rapport/30_suivi_box_volCBFdark00_SO2map.pdf}}
&%
\subfloat[VSI]{\includegraphics[width=0.3\linewidth,height=5cm]{../../images_rapport/30_suivi_box_volCBFdark00_VSI.pdf}}
\\
\hline
\end{tabular}
\end{center}
\caption{Rat num\'ero 30 : \'evolution par jour d'examen}
\label{30_box_lch00}
\end{figure}

%\FloatBarrier
Les diagrammes \ref{11_box_lch00} \g{a} \ref{30_box_lch00} montrent syst\'ematiquement une baisse des valeurs d'ADC 30 minutes apr\g{e}s la reperfusion, %
ce qui correspond \g{a} un \'etat cytotoxique -oed\g{e}me- comme nous l'avons vu dans la section pr\'ec\'edente.

\par
Ces diagrammes indiquent aussi que les valeurs de BVf et de VSI sont, sauf pour le rat 11, corr\'el\'ees : %
les tendances d'\'evolution en position ou en dispersion sont les m\^emes ; %
rappelons qu'il s'agit l\g{a} respectivement de la proportion, en volume, de sang dans l'enc\'ephale et du diam\g{e}tre des vaisseaux sanguins.

\etoile
La distribution des valeurs \ref{26_box_lch00}, pour toutes les modalit\'es, est peu variable chez le rat num\'ero 26, et presque plus \g{a} partir du jour 8.

\par
Les diagrammes de la figure \ref{30_box_lch00}, qui correspondent aux seuls jours 0, 8 et 15 de l'exp\'erience, %
indiquent une croissance globale de toutes les modalit\'es pour le rat num\'ero 30, sans que l'on puisse voir de lien plus particulier entre deux variables ; %
la CMRO2, seule, se stabilise d\g{e}s le jour 8 apr\g{e}s avoir pris des valeurs initiales basses.

\par
Pour l'\'elaboration, et la validation du mod\g{e}le qui va suivre, j'ai donc choisi d'\'ecarter les rats num\'eros 26 et 30. %
Compte tenu de la position : ant\'erieure \g{a} $8$ pour le rat 26, post\'erieure \g{a} $11$ pour le rat 30, %
il est possible que les donn\'ees dont on dispose corresponde \g{a} une r\'egion p\'eriph\'erique de la l\'esion.

\etoile
Les rats 11 et 19 -figures \ref{11_box_lch00} et \ref{19_box_lch00}, pr\'esentent quelques similitudes : %
outre le comportement de la SO2 d\'ej\g{a}mentionn\'e, on remarque une croissance et une grande dispersion des valeurs d'ADC de la l\'esion avec le temps.

\par
On constate aussi une chute des valeurs du CBF :
\begin{itemize}
\item le jour 8, chez le rat 11, elle accompagne une hausse de la BVf ;
\item le jour 15, chez le rat 19, elle co\"incide avec des valeurs basses et peu dispers\'ees de la CMRO2.
\end{itemize}

\etoile
Enfin, on remarque des valeurs de VSI beaucoup plus \'elev\'ees chez le rat 11 que chez tous les autres rats, %
ceci est valable pour la r\'egion l\'es\'ee et pour l'h\'emisph\g{e}re contralat\'eral.

\par
En raison de cette diff\'erence inn\'ee entre le rat 11 et les autres rats, %
j'ai choisi d'utiliser les donn\'ees disponibles sur le rat num\'ero 19 comme r\'ef\'erence pour mon mod\g{e}le pr\'edictif.

\FloatBarrier
\subsection{Le paquet R Mclust : une id\'ee de Nicolas Glade (TIMC)}% On peut utiliser sweaver

% Genèse de l'idée.
%Nicolas Glade \'etait parti du constat suivant : pour certaines tranches examin\'ees, %
%les valeurs calcul\'ees en ADC pour le jour 0 peuvent \^etre regroup\'ees suivant un m\'elange de lois gaussiennes, %


%\par
%Pour cela, il avait utilis\'e  

A partir d'une base de donn\'ees g\'en\'er\'ee sur la segmentation des cerveaux de rats, Nicolas Glade a utilis\'e la fonction \'eponyme du paquet R Mclust %
pour classifier les valeurs prises par l'ADC sur une tranche de cerveau, au premier jour d'examen.

\par
En effet, la librairie Mclust permet de classifier un ensemble de donn\'ees, que l'on suppose \^etre r\'eparties selon un m\'elange gaussien, %
en utilisant une m\'ethode de maximum de vraisemblance.

\par
La figure \ref{exem_ADC_19} illustre, dans le cas tridimensionnel o\g{u} l'on effectue la classification sur plusieirs tranches, juxtapos\'ees, simultan\'ement, %
le premier r\'esultat obtenu par Nicolas Glade.

\begin{figure}[H]
\includegraphics[width=0.8\linewidth,height=13cm]{../../images_rapport/19-J00-ADC-cerveau_clhist_clust.pdf}
\caption{Classification des valeurs d'ADC suivant un mod\g{e}le de m\'elange gaussien, avec quatre composantes.
%\\
%
}
\label{exem_ADC_19}
\end{figure}

\par
Les deux premiers clusters, en bleu et en rouge, de la classification correspondent aux deux pics, bien visibles, sur l'histogramme de la figure \ref{exem_ADC_19}. %
Les distributions d'ADC peuvent \g{e}tre r\'esum\'ees par les diagrammes en bo\^ite de la figure \ref{19_box_lch00}, %
et qui concernent respectivement la r\'egion l\'es\'ee et l'h\'emisph\g{e}re contralat\'eral.

\par
L'emplacement et la forme du cluster en bleu \'evoque les zones sombres visibles en ADC, BVf ou CBF sur les images de la figure \ref{19_box_lch00} %
et dont la derni\g{e}re a servi \g{a} d\'efinir la l\'esion -voir \ref{cbf_seg_19}.

\par
Le troisi\g{e}me cluster du m\'elange correspond, selon Nicolas Glade, \g{a} un m\'elange de diff\'erentes structures ; %
en effet, l'existence de deux composantes connexes sur le graphique laisse penser que ce cluster pourrait \^etre encore d\'ecompos\'e. %
Peut-\^etre touche-t-on d\g{e}s lors aux limites de la d\'ecomposition en densit\'es gaussiennes, ou que le nombre de clusters pourrait simplement \^etre plus important, %
au d\'etriment d'une bonne lisibilit\'e de la figure.

\par
Le dernier cluster est principalement compos\'e des pixels avec la diffusivit\'e la plus forte. La signification physiologique de cette propri\'et\'e n'est pas un objectif de ce stage.

\begin{figure}[!p]
\begin{center}
\begin{tabular}{|c|c|}
\hline
\includegraphics[height=7cm,width=0.4\linewidth]{../../images_rapport/19-J00-BVf-cerveau_clust.pdf}
&
\includegraphics[height=7cm,width=0.4\linewidth]{../../images_rapport/19-J00-CBF-cerveau_clust.pdf}
\\
\hline
\end{tabular}
\end{center}
\caption{Deux autres exemples pour le jour 00.%
\\%
On remarque beaucoup de donn\'ees manquantes en CBF, au niveau de la l\'esion pr\'esum\'ee.%
}
\label{19_CBF-BVf_3d00}
\end{figure}

\etoile
\ref{19_CBF-BVf_3d00}

\subsubsection{La classification meilleure que l'oeil ?}

%% Pour ou contre : listes. %%

% Jours autres que le premier : travail difficile pour l'oeil.
% De bonnes choses en ADC pour la deuxième moitié du mois.

% On peut être tenté d'utiliser la classification avec mclust pour suivre l'évolution temporelle de l'étendue de la lésion.
% Mieux : la classification ouvrirait une discuttion sur la caractérisation de la lésion : quelle modalité est la plus pertinente ?

% Problème des parasites pour l'extraction des données.
% Beaucoup d'images sont de mauvaise qualité.
% Souvent, les clusters sont dépourvus d'une signification qui serait comparable à celle dégagée dans le cas précédent : ADC au jour 0.
% La classification met peut-être au jour certaines structures physiologiques ! VSI, vascularisation ? ... Mais cela ne permet pas, la plupart du temps de caractériser la lésion cf...


%\begin{figure}[!p]
%\end{figure}


\subsubsection{Mclust au service d'une mod\'elisation pour l'\'evolution de la l\'esion}

% Point intéressant : la fonction mclust prend en argument, outre la base de données à classifier,
% un vecteur constitué d'entiers positifs qui paramètre une famille de modèles que mclust se propose de calculer.
% Ainsi, la fonction calcule \emph{plusieurs} mélanges gaussiens, un pour chacun des entiers du vecteur pris en argument,
% puis sélectionne le meilleur modèle avec le critère BIC.
% Il est d'ailleurs possibles de demander à R d'inclure, dans une représentation graphique,
% les différentes valeurs de BIC associées à chaque possibilité de nombre de clusters.
% On peut ainsi visualiser la pertinence du choix d'un nombre de clusters particulier pour la classsification.
% De telles représentations graphiques apparaissent dans les figures \ref{}...












\FloatBarrier
\subsection{Vers un mod\g{e}le pour le rat num\'ero 19}

Les r\'esultats obtenus en fin de section pr\'ec\'edente m'ont encourag\'e \g{a} suivre l'\'evolution des six modalit\'es %
ADC, BVf, CBF, CMRO2, SO2 et VSI sur des r\'egions d'int\'er\^et d\'efinies par la clusterisation de CBF, SMRO2, SO2 et BVf aux jours 8 ou 15.


\ref{}



%\Floatbarrier
%\subsection{}%Modèle proprement dit, évetuellement dans la section suivante.


