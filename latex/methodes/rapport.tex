\section{M\'ethodes}

\subsection{Protocole exp\'erimental : isch\'emie induite sur des rats par suture intraluminale}

Ces exp\'eriences ont \'et\'e men\'ees \g{a} l'Institut de Neurosciences de Grenoble (GIN) par Benjamin Lemasson, %
et m'ont \'e\'e transmises \g{a} la suite d'une r\'eunion \g{a} laquelle j'ai assist\'e au GIN %
avec Ang\'elique St\'ephanou, Emmanuel Barbier, et Benjamin Lemasson.

%\par
% Demander aux coll\g{e}gues du GIN.

\subsection{Traitement d'images avec ImageJ}

J'ai choisi de traiter pr\'ealablement les images r\'ealis\'ees en niveaux de gris par Benjamin Lemasson, \g{a} l'aide du logiciel ImageJ : %
c'est une collection d'instructions en Java qui permet de manipuler des images. Dans le cas de mon travail, il pr\'esente plusieurs avantages :
\begin{description}
\item[Travail sur des images indivuduelles] Il est possible de d\'elimiter manuellement des r\'egions d'int\'er\^et : enc\'ephale dans une coupe % de t\^ete, %
zone l\'es\'ee caract\'eris\'ee par des valeurs basses d'ADC ou de CBF.
\item[Exploitation de donn\'ees sous diff\'erents formats] On peut convertir des fichiers afin de les rendre exploitables : %
images en .jpg pour une pr\'esentation, fichiers texte avec les ccordonn\'ees et les valeurs, repr\'esent\'ees en niveaux de gris, des diff\'erents pixels.
\item[Traitement syst\'ematique avec des macros] Le logiciel permet de cr\'eer des macros %
afin de traiter syst\'ematiquement un grand nombre d'images : il suffit de d\'elimiter des r\'egions d'int\'r\^et %
comme le cerveau %etc
ce qui est un travail fastidieux, pour une seule modalit\'e et d'ex\'ecuter ces petits programmes pour avoir un nombre relativement grand de donn\'ees exploitables.
\begin{itemize}
\item
\item 
\end{itemize}
\end{description}


\subsection{Le paquet R Mclust : une id\'ee de Nicolas Glade (TIMC)}% On peut utiliser sweaver



S\'electionner des images de bonnes clusterisations.

\subsection{}% Utilisation de R : statistiques sur les cerveaux segment\'es, h\'emisph\g{e}res sains.

% Quelques exemples de diagrammesgrammes