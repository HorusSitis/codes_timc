\section{M\'ethodes}

%%%% debut macro %%%%
\makeatletter
\renewcommand{\thefigure}{\ifnum \c@section>\z@ \thesection.\fi
 \@arabic\c@figure}
\@addtoreset{figure}{section}
\makeatother
%%%% fin macro %%%%
\begin{comment}
Remarque : pour renuméroter les sous-figures de la même manière
           (avec le package 'subfigure'), il suffit de rajouter
	   la ligne \let\p@subfigure\thefigure dans le préambule.
\end{comment}

\subsection{Mod\g{e}le exp\'erimental}

%Ces exp\'eriences ont \'et\'e men\'ees \g{a} l'Institut de Neurosciences de Grenoble (GIN) par Benjamin Lemasson.
%, %
%et m'ont \'et\'e transmises \g{a} la suite d'une r\'eunion \g{a} laquelle j'ai assist\'e au GIN %
%avec Ang\'elique St\'ephanou, Emmanuel Barbier, et Benjamin Lemasson.
\subsubsection{Chirurgie}

Les donn\'ees exploit\'ees dans cette section sont issues d'une exp\'erience men\'ee au GIN (Grenoble Institude of Neurosciences) par Benjamin Lemasson, %
sur des cerveaux de rats.

\par
Les sujets ont d'abord subi une occlusion intraluminale d'une art\g{e}re c\'er\'ebrale post\'erieure -voir \cite{Durukan_PBB_07}. pour une illustration de la technique.

%\par
%d'une art\g{e}re situ\'ee en aval du polygone de Willis. %
%Des exemples d'occlusions intraluminale sont sch\'ematis\'es dans %

\begin{figure}[H]
\includegraphics[width=0.8\linewidth, height = 6.8cm]{../../images_rapport/willis.png}
\caption{Polygone de Willis, aliment\'e par les art\g{e}res carotides et celle du tronc basilaire.
\\
Il permet d'\'eviter une isch\'emie globale, en cas de dysfonctionnement d'une art\g{e}re situ\'ee en amont, %
mais pas une isch\'emie focale en cas de probl\g{e}me sur une art\g{e}re post\'erieure.
}
\label{willis}
\end{figure}

\par
Les rats subissent d'abord une anesth\'esie g\'en\'erale pendant 60 minutes. %
Un filament en plastique est ensuite introduit dans le r\'eseau art\'eriel de chaque rat, %
et remonte jusqu'\g{a} une art\g{e}re post\'erieure, en aval du polygone de Willis.

\par
Cette art\g{e}re est ainsi obstru\'ee, dans chaque cerveau de rat, pendant 90 minutes. %
Ceci provoque une isch\'emie focale : celle-ci concerne seulement la partie du cerveau aliment\'ee par l'art\g{e}re, %
contrairement \g{a} une isch\'emie globale qui touche l'ensemble du cerveau.

\par
L'isch\'emie provoqu\'ee, lors d'un AVC, par l'obstruction d'une art\g{e}re par un caillot sanguin est elle-m\^eme focale, %
ce qui prouve la pertinence du mod\g{e}le exp\'erimental.


\subsubsection{Suivi pour quatre rats}

Les rats font l'objet d'un suivi : chaque rat subit une s\'erie d'examens IRM 30 minutes, 3 jours, 8 jours, 15 jours et 22 jours apr\g{e}s la fin de l'isch\'emie.

\par
Quatre rats, num\'erot\'es 11, 19, 26 et 30, ont v\'ecu suffisamment longtemps apr\g{e}s le d\'ebut de l'exp\'erience, %
pour que ces examens soient pratiqu\'es.

\par
13 coupes frontales ont ainsi \'et\'e produites, chaque jour et pour chaque rat, %
selon les modalit\'es \'evoqu\'ees en introduction.

\par
Des images en ADC, BVf, CBF, CMRO2, SO2, et en pond\'eration T1 et T2 ont \'et\'e obtenues. %
Elles sont utilis\'ees, dans la suite de ce travail, pour effectuer un suivi temporel de r\'egions l\'es\'ees, ou saines, %
du cerveau de chaque rat.

\par
Toutefois, le nombre d'images disponibles pour toutes les variables \'evoqu\'ees ci-dessus sont seulement disponibles pour un petit nombre de coupes : %
trois coupes au maximum, pour un rat. Ceci n'emp\^eche ps d'obtenir des r\'esultats int\'eressants.

\etoile
Une derni\g{e}re variable, d\'efinie pour chaque coupe des cerveaux de rats examin\'es, est le diam\g{e}tre moyen des vaisseaux (Vessel Size index ou VSI). %
Il est donn\'ee par l'\'equation \ref{vsi_adc}, que l'on trouve dans \cite{Lem_PHD_10} :

\begin{equation}
VSI = 0,425\left(\frac{ADC}{\Delta_{\chi}B_0}\right)^{\frac{1}{2}}\left(\frac{\Delta R_2^{\ast}}{\Delta R_2}\right)^{\frac{3}{2}}
\label{vsi_adc}
\end{equation}

Habituellement, on exprime le VSI en microm\g{e}tres.

\par
Les notations $R_2$, $R_2^{\ast}$ -respectivement $\frac{1}{T_2}$ et $\frac{1}{T_2^{\ast}}$ - et $\Delta_{\chi}$ sont d\'efinies en introduction.

\par
Les images ci-dessous, dites multiparam\'etriques car elles concernent plusieurs variables, ont ainsi \'et\'e obtenues.

\begin{figure}[!p]
\begin{center}
\begin{tabular}{|c|c|c|c|}
\hline
\subfloat[T2 anatomique]{\includegraphics[width=0.2\linewidth,height=3.5cm]{../../images_rapport/11-J03-Coreg01_Anat-masked-slice-10.jpg}}
&
\subfloat[ADC]{\includegraphics[width=0.2\linewidth,height=3.5cm]{../../images_rapport/11-J03-CoregADC-slice-10.jpg}}
&
\subfloat[BVf]{\includegraphics[width=0.2\linewidth,height=3.5cm]{../../images_rapport/11-J03-CoregBVf-slice-10.jpg}}
&
\subfloat[CBF]{\includegraphics[width=0.2\linewidth,height=3.5cm]{../../images_rapport/11-J03-CoregCBF-slice-10.jpg}}
\\
\hline
\subfloat[CMRO2]{\includegraphics[width=0.2\linewidth,height=3.5cm]{../../images_rapport/11-J03-CoregCMRO2-slice-10.jpg}}
&
\subfloat[SO2map]{\includegraphics[width=0.2\linewidth,height=3.5cm]{../../images_rapport/11-J03-CoregSO2map-slice-10.jpg}}
&
\subfloat[T1map]{\includegraphics[width=0.2\linewidth,height=3.5cm]{../../images_rapport/11-J03-CoregT1map-slice-10.jpg}}
&
\subfloat[VSI]{\includegraphics[width=0.2\linewidth,height=3.5cm]{../../images_rapport/11-J03-CoregVSI-slice-10.jpg}}
\\
\hline
\end{tabular}
\end{center}
\caption{Images disponibles pour le rat 11, au jour 03, coupe num\'ero 10.
\\%\par
De gauche \g{a} droite, et de haut en bas : %
l'image brute en T2 \og{} Anatomique\fg{}, l'ADC, le BVf, le CBF, la CMRO2, la SO2, l'image en T1 et le VSI.}
\label{ex_irm_multipar}
\end{figure}

%\FloatBarrier
\subsection{Segmentation des images avec ImageJ}

Les images en niveau de gris ont été pr\'ealablement trait\'ees \g{a} l'aide du logisiel ImageJ couramment utilis\'e dans la manipulation d'images.
%
\par
A partir des images, au format .tif, %transmises par Benjamin Lemasson, %
ImageJ a \'et\'e untilis\'e pour d\'elimiter les r\'egions correspondant au cerveau, pour toutes les modalit\'es -voir figure \ref{cephcer}.
\par
Ma d\'emarche a \'et\'e la suivante :
\begin{enumerate}
\item Tracer manuellement le contour des cerveaux sur les coupes anatomiques, prises en T2, figure \ref{cephcer} ;
\item Charger individuellement ces contours sur les images correspondant aux autres modalit\'es, ici l'ADC : les deux images centrales de la figure ;
\item Retirer toutes les valeurs correspondant \g{a} l'ext\'erieur du contour : voir l'image segment\'ee de la figure \ref{cephcer} ;
\item Automatiser la proc\'edure pour traiter et trier, \g{a} partir des contours d\'efinis manuellement, toutes les images disponibles.
\end{enumerate}

%\newcounter{stock}
%\setcounter{stock}{\value{enumi}}

\begin{figure}[H]%Ajouter les colonnes correspondant à anatomique, ADC sans contour ce ADC segmentée.
\begin{center}
\begin{tabular}{|c|c|c|c|}
\hline
\subfloat[Image en anatomique]{\includegraphics[width=0.2\linewidth,height=4cm]{../../images_rapport/11-J03-Coreg01_Anat-masked-slice-10.jpg}}
&%\hfill
\subfloat[Image brute en ADC]{\includegraphics[width=0.2\linewidth,height=4cm]{../../images_rapport/11-J03-CoregADC-slice-10.jpg}}
&%\hfill
\subfloat[Image + contour]{\includegraphics[width=0.2\linewidth,height=4cm]{../../images_rapport/11-J03-segADC-slice-10.jpg}}
&%\hfill
\subfloat[Image segment\'ee]{\includegraphics[width=0.2\linewidth,height=4cm]{../../images_rapport/11-J03-ADC-bg-slice10.jpg}}
\\
\hline
\end{tabular}
\caption{Segmentation du cerveau du rat num\'ero 11, jour 03, avec la modalit\'e ADC.}
\end{center}
\label{cephcer}
\end{figure}

%\FloatBarrier
Des images ont ainsi \'et\'e obtenues, par coupe, pour toutes les modalit\'es, restreintes \g{a} l'enc\'ephale.% L'\'etape finale pour l'exploitation des donn\'ees IRM a \'et\'e :
%\begin{enumerate}
%\setcounter{enumi}{\value{stock}}
%\item Regrouper les niveaux de gris, qui correspondent aux valeurs mesur\'ees ou calcul\'ees de chacune des images, dans un fichier texte exploitable sous R.
%\end{enumerate}

\etoile
Les images aux coupes : 10 pour le rat num\'ero 11, 9 et 10 pour le rat 19, 6 \g{a} 8 pour le rat 26 et 11 \g{a} 13 pour le rat 30 sont disponibles pour toutes les modalit\'es, %
pour tous les jours d'examens. %
Ces coupes sont donc utilisables pour effectuer un suivi temporel des valeurs mesur\'ees ou calcul\'ees, pixel par pixel, %
ou globalement sur les coupes enti\g{e}res ou des r\'egions d'int\'er\^et.

\begin{figure}[!p]
\begin{center}
\begin{tabular}{|c|c|c|c|}
\hline
\subfloat[Anatomique]{\includegraphics[width=0.2\linewidth,height=2.3cm]{../../images_rapport/11-J00-Coreg01_Anat-masked-Cropped-slice10.jpg}}
&
\subfloat[ADC]{\includegraphics[width=0.2\linewidth,height=2.3cm]{../../images_rapport/11-J00-ADC-Cropped-slice10.jpg}}
&
\subfloat[BVf]{\includegraphics[width=0.2\linewidth,height=2.3cm]{../../images_rapport/11-J00-BVf-Cropped-slice10.jpg}}
&
\subfloat[CBF]{\includegraphics[width=0.2\linewidth,height=2.3cm]{../../images_rapport/11-J00-CBF-seg-slice10.jpg}}
\\
\hline
\subfloat[CMRO2]{\includegraphics[width=0.2\linewidth,height=2.3cm]{../../images_rapport/11-J00-CMRO2-Cropped-slice10.jpg}}
&
\subfloat[SO2map]{\includegraphics[width=0.2\linewidth,height=2.3cm]{../../images_rapport/11-J00-SO2map-Cropped-slice10.jpg}}
&
\subfloat[T1map]{\includegraphics[width=0.2\linewidth,height=2.3cm]{../../images_rapport/11-J00-T1map-Cropped-slice10.jpg}}
&
\subfloat[VSI]{\includegraphics[width=0.2\linewidth,height=2.3cm]{../../images_rapport/11-J00-VSI-Cropped-slice10.jpg}}
\\
\hline
\end{tabular}
\end{center}
\caption{Rat 11, jour 00, coupe 10}
\label{11_dark_00}
\end{figure}

\begin{figure}[!p]
\begin{center}
\begin{tabular}{|c|c|c|c|}
\hline
\subfloat[Anatomique]{\includegraphics[width=0.2\linewidth,height=2.3cm]{../../images_rapport/19-J00-Coreg01_Anat-masked-Cropped-slice9.jpg}}
&
\subfloat[ADC]{\includegraphics[width=0.2\linewidth,height=2.3cm]{../../images_rapport/19-J00-ADC-Cropped-slice9.jpg}}
&
\subfloat[BVf]{\includegraphics[width=0.2\linewidth,height=2.3cm]{../../images_rapport/19-J00-BVf-Cropped-slice9.jpg}}
&
\subfloat[CBF]{\includegraphics[width=0.2\linewidth,height=2.3cm]{../../images_rapport/19-J00-segCBF-slice9.jpg}}
\\
\hline
\subfloat[CMRO2]{\includegraphics[width=0.2\linewidth,height=2.3cm]{../../images_rapport/19-J00-CMRO2-Cropped-slice9.jpg}}
&
\subfloat[SO2map]{\includegraphics[width=0.2\linewidth,height=2.3cm]{../../images_rapport/19-J00-SO2map-Cropped-slice9.jpg}}
&
\subfloat[T1map]{\includegraphics[width=0.2\linewidth,height=2.3cm]{../../images_rapport/19-J00-T1map-Cropped-slice9.jpg}}
&
\subfloat[VSI]{\includegraphics[width=0.2\linewidth,height=2.3cm]{../../images_rapport/19-J00-VSI-Cropped-slice9.jpg}}
\\
\hline
\end{tabular}
\end{center}
\caption{Rat 19, jour 00, coupe 9}
\label{19_dark_00}
\end{figure}

\begin{figure}[!p]
\begin{center}
\begin{tabular}{|c|c|c|c|}
\hline
\subfloat[Anatomique]{\includegraphics[width=0.2\linewidth,height=2.3cm]{../../images_rapport/26-J00-Coreg01_Anat-masked-Cropped-slice8.jpg}}
&
\subfloat[ADC]{\includegraphics[width=0.2\linewidth,height=2.3cm]{../../images_rapport/26-J00-ADC-Cropped-slice8.jpg}}
&
\subfloat[BVf]{\includegraphics[width=0.2\linewidth,height=2.3cm]{../../images_rapport/26-J00-BVf-Cropped-slice8.jpg}}
&
\subfloat[CBF]{\includegraphics[width=0.2\linewidth,height=2.3cm]{../../images_rapport/26-J00-CBF-seg-slice8.jpg}}
\\
\hline
\subfloat[CMRO2]{\includegraphics[width=0.2\linewidth,height=2.3cm]{../../images_rapport/26-J00-CMRO2-Cropped-slice8.jpg}}
&
\subfloat[SO2map]{\includegraphics[width=0.2\linewidth,height=2.3cm]{../../images_rapport/26-J00-SO2map-Cropped-slice8.jpg}}
&
\subfloat[T1map]{\includegraphics[width=0.2\linewidth,height=2.3cm]{../../images_rapport/26-J00-T1map-Cropped-slice8.jpg}}
&
\subfloat[VSI]{\includegraphics[width=0.2\linewidth,height=2.3cm]{../../images_rapport/26-J00-VSI-Cropped-slice8.jpg}}
\\
\hline
\end{tabular}
\end{center}
\caption{Rat 26, jour 00, coupe 8}
\label{26_dark_00}
\end{figure}

\begin{figure}[!p]
\begin{center}
\begin{tabular}{|c|c|c|c|}
\hline
\subfloat[Anatomique]{\includegraphics[width=0.2\linewidth,height=2.3cm]{../../images_rapport/30-J00-Coreg01_Anat-masked-Cropped-slice11.jpg}}
&
\subfloat[ADC]{\includegraphics[width=0.2\linewidth,height=2.3cm]{../../images_rapport/30-J00-ADC-Cropped-slice11.jpg}}
&
\subfloat[BVf]{\includegraphics[width=0.2\linewidth,height=2.3cm]{../../images_rapport/30-J00-BVf-Cropped-slice11.jpg}}
&
\subfloat[CBF]{\includegraphics[width=0.2\linewidth,height=2.3cm]{../../images_rapport/30-J00-CBF-seg-slice11.jpg}}
\\
\hline
\subfloat[CMRO2]{\includegraphics[width=0.2\linewidth,height=2.3cm]{../../images_rapport/30-J00-CMRO2-Cropped-slice11.jpg}}
&
\subfloat[SO2map]{\includegraphics[width=0.2\linewidth,height=2.3cm]{../../images_rapport/30-J00-SO2map-Cropped-slice11.jpg}}
&
\subfloat[T1map]{\includegraphics[width=0.2\linewidth,height=2.3cm]{../../images_rapport/30-J00-T1map-Cropped-slice11.jpg}}
&
\subfloat[VSI]{\includegraphics[width=0.2\linewidth,height=2.3cm]{../../images_rapport/30-J00-VSI-Cropped-slice11.jpg}}
\\
\hline
\end{tabular}
\end{center}
\caption{Rat 30, jour 00, coupe 11}
\label{30_dark_00}
\end{figure}

%\FloatBarrier
\par
Avec des choix judicieux de contraste, on peut voir %\g{a} l'oeil, %
des r\'egions l\'es\'ees sur les coupes segment\'ees de cerveau. %
Plus particuli\g{e}rement, et pour tous les rats, on remarque une aire sombre sur les images en CBF au jour 00.

\par
Rappelons que ces images ont \'et\'e produites 30 minutes apr\g{e}s la reperfusion.
%Cette aire se situe sur l'h\'emisph\g{e}re visible \g{a} gauche.

\par
On trouve d'ailleurs des r\'egions similaires, mais plus petites, pour d'autres variables comme l'ADC : %
voir les figures \ref{11_dark_00}, \ref{19_dark_00}, \ref{26_dark_00} et \ref{30_dark_00}.

\par
La caract\'erisation de la r\'egion isch\'emi\'ee par des valeurs faibles du CBF a d'ailleurs \'et\'e discut\'ee \g{a} la section pr\'ec\'edente.

\etoile
Afin de d\'elimiter les parties isch\'emi\'ees sur les coupes disponibles, nous avons utilis\'e la m\^eme technique que pour les cerveaux : voir la figure \ref{cbf_seg_19}. %
Les bordures trac\'ees sur les images en CBF one \'et\'e transf\'er\'ees sur les autres images, ce qui a permis d'isoler les pixels de l'isch\'emie.

\par
Les valeurs prises, sur ces pixels, par ADC, BVf \dots T1map et VSI seront \'etudi\'ees \g{a} la sous-section suivante.

\par
Le tableau ci-dessous donne nombre de pixels, sur chaque coupe suivable temporellement, des l\'esions ainsi d\'elimit\'ees.

\begin{multicols}{2}
\begin{tabular}{|c|c|c|c|c|c|c|c|c|}
\hline
\small{Tranche}&6&7&8&9&10&11&12&13
\\
\hline
Rat 11&&&&&813&&&
\\
\hline
Rat 19&&&&742&654&&&
\\
\hline
Rat 26&772&791&558&&&&&
\\
\hline
Rat 30&&&&&&88&139&83
\\
\hline
\end{tabular}

\columnbreak
\begin{figure}[H]
\begin{center}
\includegraphics[width=0.45\linewidth, height=3cm]{../../images_rapport/19-J00-segCBF-slice9.jpg}
\end{center}
\caption{D\'elimitation d'un contour : rat 19, coupe 9. On utilise une image du CBF.}
\label{cbf_seg_19}
\end{figure}
\end{multicols}

\newpage
\FloatBarrier
\subsection{Etude de la distribution des niveaux de gris : tissus sains, tissus l\'es\'es}% Utilisation de R : statistiques sur les cerveaux segment\'es, h\'emisph\g{e}res sains.

Gr\^ace \g{a} ImageJ, on dispose maintenant de la distribution des valeurs d'ADC, BVf, CBF, CMRO2, SO2, VSI, T1 et T2 pour des coupes frontales de cerveaux de rats, %
et des coupes de tissus isch\'emi\'es.

\par
A l'aide du logiciel R, on peut repr\'esenter graphiquement la r\'epartition des valeurs prises par une variable, %
sur un tissu sain ou isch\'emi\'e.

\par
Les figures  \ref{11_dens_lch00} et \ref{19_dens_lch00} donnent, pour les variables ADC, CBF, SO2 et VSI, %
la distribution des valeurs prises sur la partie l\'es\'ee (rouge), %
sur l'h\'emisph\g{e}re contralat\'eral qui n'est pas touch\'e par l'isch\'emie (bleu), et sur le cerveau entier (gris).

\etoile
On remarque, sur les courbes des figures \ref{11_dens_lch00} et \ref{19_dens_lch00} :

\begin{itemize}
\item Une forte proportion de valeurs basses du CBF, dans la r\'egion isch\'emi\'ee du rat 19 ;
\item Une forte proportion de valeurs basses de l'ADC, dans la partie isch\'emi\'ee du rat 11 ;
\item Une r\'epartition des valeurs de SO2, dans le tissu l\'es\'e du rat 19, qui se rapproche de celle d'un tissu sain aux jours 15 et 22 ;
\item Une grande variabilit\'e des valeurs d'ADC, sur la l\'esion du rat 11, \g{a} partir du jour 15.
\end{itemize}

\par
Les deux premiers constats sont en accord avec les connaissances que nous avons du d\'ebut de l'isch\'emie : voir la section pr\'ec\'edente.

\par
Remarquons par ailleurs la forte ressemblance entre les distributions de valeurs d'ADC et de T1, %
sur la coupe num\'ero 9 du cerveau du rat 11.

\par
On remarque ces allures communes entre les distributions d'ADC et de T1 pour toutes les coupes disponibles, %
pour tous les rats et quel que soit le tissu analys\'e -l\'esion ou h\'emisph\g{e}re contralat\'eral par exemple.

\par
Ceci est surprenant car les modalit\'es d'obtention d'une image en T1, et d'une image en ADC sont tr\g{e}s diff\'erentes, %
comme nous l'avons vu en introduction.

%\par
\ligneinter
Les diagrammes en bo\^ites : \ref{11_box_lch00}, \ref{19_box_lch00}, \ref{26_box_lch00} et \ref{30_box_lch00}, %
permettent repr\'esentent la r\'epartition des valeurs des six variables : ADC, BVf, CBF, CMRO2, SO2map et VSI. %
Le code couleur est le m\^eme que pour la figure \ref{19_dens_lch00}.

\par
M\^eme s'ils donnent moins d'information que les courbes de densit\'e, les diagrammes en bo\^ite sont utiles quand on veut mod\'elier l'\'evolution d'un tissu. %
En effet, les valeur prises par la m\'ediane ou les quartiles aux diff\'erents jours d'examen apportent une information partielle, %
mais quantifi\'ee sur l'\'etat du tissu qui nous int\'eresse.

\par
Notons au passage que dans les figures qui illustrent cette section, except\'ee la derni\g{e}re \ref{19_suivi_clust_les}, %
la distribution des niveaux de gris sur l'h\'emisph\g{e}re contralat\'eral ne concerne que le jour 0 de l'exp\'erience.

\par
Ce choix de repr\'esentation, qui permet de calibrer les courbes de densit\'e sur une valeur de r\'ef\'erence, %
ne revient pas exactement \g{a} comparer l'\'etat d'un tissu l\'es\'e \g{a} celui d'un tissu sain.

\par
En effet, m\^eme les cellules de l'h\'emisph\g{e}re contralat\'eral ont une activit\'e normale pendant et par\g{e}s l'isch\'emie. %
Toutefois, les valeurs observ\'ees pour le CBF chez le rat 19 -figure \ref{19_choix_clust_les}- sont plus basses 30 minutes apr\g{e}s la reperfusion -jour 0- %
que pour les autres jours.


\begin{figure}[!p]
\begin{center}
\begin{tabular}{|c|c|}
\hline
%\subfloat[ADC, jour 0]{
\includegraphics[width=0.4\linewidth, height = 4.8cm]{../../images_rapport/11_suivi_dens_slCBF_contra00_ADC-00.pdf}
%}
&
%\subfloat[T1map, jour 0]{
\includegraphics[width=0.4\linewidth, height = 4.8cm]{../../images_rapport/11_suivi_dens_slCBF_contra00_T1map-00.pdf}
%}
\\
\hline
%\subfloat[ADC, jour 03]{
\includegraphics[width=0.4\linewidth, height = 4.8cm]{../../images_rapport/11_suivi_dens_slCBF_contra00_ADC-03.pdf}
%}
&
%\subfloat[T1map, jour 03]{
\includegraphics[width=0.4\linewidth, height = 4.8cm]{../../images_rapport/11_suivi_dens_slCBF_contra00_T1map-03.pdf}
%}
\\
\hline
%\subfloat[ADC, jour 08]{
\includegraphics[width=0.4\linewidth, height = 4.8cm]{../../images_rapport/11_suivi_dens_slCBF_contra00_ADC-08.pdf}
%}
&
%\subfloat[T1map, jour 08]{
\includegraphics[width=0.4\linewidth, height = 4.8cm]{../../images_rapport/11_suivi_dens_slCBF_contra00_T1map-08.pdf}
%}
\\
\hline
%\subfloat[ADC, jour 15]{
\includegraphics[width=0.4\linewidth, height = 4.8cm]{../../images_rapport/11_suivi_dens_slCBF_contra00_ADC-15.pdf}
%}
&
%\subfloat[T1map, jour 15]{
\includegraphics[width=0.4\linewidth, height = 4.8cm]{../../images_rapport/11_suivi_dens_slCBF_contra00_T1map-15.pdf}
%}
\\
\hline
%\subfloat[ADC, jour 22]{
\includegraphics[width=0.4\linewidth, height = 4.8cm]{../../images_rapport/11_suivi_dens_slCBF_contra00_ADC-22.pdf}
%}
&
%\subfloat[T1map, jour 22]{
\includegraphics[width=0.4\linewidth, height = 4.8cm]{../../images_rapport/11_suivi_dens_slCBF_contra00_T1map-22.pdf}
%}
\\
\hline
\end{tabular}
\end{center}
\caption{Rat 11 : ADC et T1map.
%\\
%On remarque des similitudes entre les r\'epartitions des valeurs des deux variables, %
%que l'on retrouve dans tous les types de tissus \'etudi\g{e}s et chez tous les rats. %
%\\
%Nous nous int\'eresserons assez peu, par la suite, \g{a} la T1map.
}
\label{11_dens_lch00}
\end{figure}


\begin{figure}[!p]
\begin{center}
\begin{tabular}{|c|c|}
\hline
%\subfloat[CBF, jour 0]{
\includegraphics[width=0.4\linewidth, height = 4.8cm]{../../images_rapport/19_suivi_dens_slCBF_contra00_CBF-00.pdf}
%}
&
%\subfloat[SO2map, jour 0]{
\includegraphics[width=0.4\linewidth, height = 4.8cm]{../../images_rapport/19_suivi_dens_slCBF_contra00_SO2map-00.pdf}
%}
\\
\hline
%\subfloat[CBF, jour 03]{
\includegraphics[width=0.4\linewidth, height = 4.8cm]{../../images_rapport/19_suivi_dens_slCBF_contra00_CBF-03.pdf}
%}
&
%\subfloat[T1map, jour 0]{\includegraphics[width=0.4\linewidth, height = 4cm]{../../images_rapport/19_suivi_dens_slCBF_contra00_SO2map-03.pdf}}
\\
\hline
%\subfloat[CBF, jour 08]{
\includegraphics[width=0.4\linewidth, height = 4.8cm]{../../images_rapport/19_suivi_dens_slCBF_contra00_CBF-08.pdf}
%}
&
%\subfloat[SO2map, jour 08]{
\includegraphics[width=0.4\linewidth, height = 4.8cm]{../../images_rapport/19_suivi_dens_slCBF_contra00_SO2map-08.pdf}
%}
\\
\hline
%\subfloat[CBF, jour 15]{
\includegraphics[width=0.4\linewidth, height = 4.8cm]{../../images_rapport/19_suivi_dens_slCBF_contra00_CBF-15.pdf}
%}
&
%\subfloat[SO2map, jour 15]{
\includegraphics[width=0.4\linewidth, height = 4.8cm]{../../images_rapport/19_suivi_dens_slCBF_contra00_SO2map-15.pdf}
%}
\\
\hline
%\subfloat[CBF, jour 22]{
\includegraphics[width=0.4\linewidth, height = 4.8cm]{../../images_rapport/19_suivi_dens_slCBF_contra00_CBF-22.pdf}
%}
&
%\subfloat[SO2map, jour 22]{
\includegraphics[width=0.4\linewidth, height = 4.8cm]{../../images_rapport/19_suivi_dens_slCBF_contra00_SO2map-22.pdf}
%}
\\
\hline
\end{tabular}
\end{center}
\caption{Rat 19 : CBF et SO2map.}
\label{19_dens_lch00}
\end{figure}

%\ligneinter

\begin{figure}[!p]
\begin{center}
\begin{tabular}{|c|c|c|}
\hline
\subfloat[ADC]{\includegraphics[width=0.3\linewidth,height=5cm]{../../images_rapport/11_suivi_box_volCBFdark00_ADC.pdf}}
&%
\subfloat[BVf]{\includegraphics[width=0.3\linewidth,height=5cm]{../../images_rapport/11_suivi_box_volCBFdark00_BVf.pdf}}
&%
\subfloat[CBF]{\includegraphics[width=0.3\linewidth,height=5cm]{../../images_rapport/11_suivi_box_volCBFdark00_CBF.pdf}}
\\
\hline
\subfloat[CMRO2]{\includegraphics[width=0.3\linewidth,height=5cm]{../../images_rapport/11_suivi_box_volCBFdark00_CMRO2.pdf}}
&%
\subfloat[SO2map]{\includegraphics[width=0.3\linewidth,height=5cm]{../../images_rapport/11_suivi_box_volCBFdark00_SO2map.pdf}}
&%
\subfloat[VSI]{\includegraphics[width=0.3\linewidth,height=5cm]{../../images_rapport/11_suivi_box_volCBFdark00_VSI.pdf}}
\\
\hline
\end{tabular}
\end{center}
\caption{Rat num\'ero 11 : \'evolution par jour d'examen}
\label{11_box_lch00}
\end{figure}

\begin{figure}[!p]
\begin{center}
\begin{tabular}{|c|c|c|}
\hline
\subfloat[ADC]{\includegraphics[width=0.3\linewidth,height=5cm]{../../images_rapport/19_suivi_box_volCBFdark00_ADC.pdf}}
&%
\subfloat[BVf]{\includegraphics[width=0.3\linewidth,height=5cm]{../../images_rapport/19_suivi_box_volCBFdark00_BVf.pdf}}
&%
\subfloat[CBF]{\includegraphics[width=0.3\linewidth,height=5cm]{../../images_rapport/19_suivi_box_volCBFdark00_CBF.pdf}}
\\
\hline
\subfloat[CMRO2]{\includegraphics[width=0.3\linewidth,height=5cm]{../../images_rapport/19_suivi_box_volCBFdark00_CMRO2.pdf}}
&%
\subfloat[SO2map]{\includegraphics[width=0.3\linewidth,height=5cm]{../../images_rapport/19_suivi_box_volCBFdark00_SO2map.pdf}}
&%
\subfloat[VSI]{\includegraphics[width=0.3\linewidth,height=5cm]{../../images_rapport/19_suivi_box_volCBFdark00_VSI.pdf}}
\\
\hline
\end{tabular}
\end{center}
\caption{Rat num\'ero 19 : \'evolution par jour d'examen}
\label{19_box_lch00}
\end{figure}

\begin{figure}[!p]
\begin{center}
\begin{tabular}{|c|c|c|}
\hline
\subfloat[ADC]{\includegraphics[width=0.3\linewidth,height=5cm]{../../images_rapport/26_suivi_box_volCBFdark00_ADC.pdf}}
&%
\subfloat[BVf]{\includegraphics[width=0.3\linewidth,height=5cm]{../../images_rapport/26_suivi_box_volCBFdark00_BVf.pdf}}
&%
\subfloat[CBF]{\includegraphics[width=0.3\linewidth,height=5cm]{../../images_rapport/26_suivi_box_volCBFdark00_CBF.pdf}}
\\
\hline
\subfloat[CMRO2]{\includegraphics[width=0.3\linewidth,height=5cm]{../../images_rapport/26_suivi_box_volCBFdark00_CMRO2.pdf}}
&%
\subfloat[SO2map]{\includegraphics[width=0.3\linewidth,height=5cm]{../../images_rapport/26_suivi_box_volCBFdark00_SO2map.pdf}}
&%
\subfloat[VSI]{\includegraphics[width=0.3\linewidth,height=5cm]{../../images_rapport/26_suivi_box_volCBFdark00_VSI.pdf}}
\\
\hline
\end{tabular}
\end{center}
\caption{Rat num\'ero 26 : \'evolution par jour d'examen}
\label{26_box_lch00}
\end{figure}

\begin{figure}[!p]
\begin{center}
\begin{tabular}{|c|c|c|}
\hline
\subfloat[ADC]{\includegraphics[width=0.3\linewidth,height=5cm]{../../images_rapport/30_suivi_box_volCBFdark00_ADC.pdf}}
&%
\subfloat[BVf]{\includegraphics[width=0.3\linewidth,height=5cm]{../../images_rapport/30_suivi_box_volCBFdark00_BVf.pdf}}
&%
\subfloat[CBF]{\includegraphics[width=0.3\linewidth,height=5cm]{../../images_rapport/30_suivi_box_volCBFdark00_CBF.pdf}}
\\
\hline
\subfloat[CMRO2]{\includegraphics[width=0.3\linewidth,height=5cm]{../../images_rapport/30_suivi_box_volCBFdark00_CMRO2.pdf}}
&%
\subfloat[SO2map]{\includegraphics[width=0.3\linewidth,height=5cm]{../../images_rapport/30_suivi_box_volCBFdark00_SO2map.pdf}}
&%
\subfloat[VSI]{\includegraphics[width=0.3\linewidth,height=5cm]{../../images_rapport/30_suivi_box_volCBFdark00_VSI.pdf}}
\\
\hline
\end{tabular}
\end{center}
\caption{Rat num\'ero 30 : \'evolution par jour d'examen}
\label{30_box_lch00}
\end{figure}

%\FloatBarrier
\etoile
Les diagrammes des figures \ref{11_box_lch00} \g{a} \ref{30_box_lch00} montrent syst\'ematiquement une baisse des valeurs d'ADC 30 minutes apr\g{e}s la reperfusion, %
ce qui correspond \g{a} un \'etat cytotoxique -oed\g{e}me- comme nous l'avons vu dans la section pr\'ec\'edente.

\par
Ces diagrammes indiquent aussi que les valeurs de BVf et de VSI sont, sauf pour le rat 11, corr\'el\'ees : %
Les valeurs m\'edianes et les quartiles de ces deux variables \'evoluent en effet de la m\^eme mani\g{e}re.

\par
Nous verrons une explication physiologique de cette corr\'elation dans la suite de cette section.

\etoile
La distribution des valeurs, figure \ref{26_box_lch00}, pour toutes les modalit\'es, est peu variable chez le rat num\'ero 26, et presque plus \g{a} partir du jour 8.

\par
Les diagrammes de la figure \ref{30_box_lch00}, qui correspondent aux seuls jours 0, 8 et 15 de l'exp\'erience, %
indiquent une croissance globale de toutes les modalit\'es pour le rat num\'ero 30, %
sans lien particulier entre deux variables. %
La CMRO2, seule, se stabilise d\g{e}s le jour 8 apr\g{e}s avoir pris des valeurs initiales basses.

\par
Pour l'\'elaboration, et la validation du mod\g{e}le qui va suivre, j'ai donc choisi d'\'ecarter les rats num\'eros 26 et 30. %
Compte tenu de la position : ant\'erieure \g{a} $8$ pour le rat 26, post\'erieure \g{a} $11$ pour le rat 30, %
il est possible que les donn\'ees dont on dispose corresponde \g{a} une r\'egion p\'eriph\'erique de la l\'esion.

\etoile
Les rats 11 et 19 -figures \ref{11_box_lch00} et \ref{19_box_lch00}, pr\'esentent quelques similitudes : %
outre le comportement de la SO2 d\'ej\g{a}mentionn\'e, on remarque une croissance et une plus grande variabilit\'e des valeurs d'ADC de la l\'esion avec le temps.

\par
On constate aussi une chute des valeurs du CBF :
\begin{itemize}
\item le jour 8, chez le rat 11, elle accompagne une hausse de la BVf ;
\item le jour 15, chez le rat 19, ses valeurs sont basses et varient peu comme celles de la CMRO2.
\end{itemize}

\etoile
Enfin, on remarque des valeurs de VSI beaucoup plus \'elev\'ees chez le rat 11 que chez tous les autres rats, %
ceci est valable pour la r\'egion l\'es\'ee et pour l'h\'emisph\g{e}re contralat\'eral.

\par
En raison de cette diff\'erence inn\'ee entre le rat 11 et les autres rats, %
j'ai choisi d'utiliser les donn\'ees disponibles sur le rat num\'ero 19 comme r\'ef\'erence pour mon mod\g{e}le pr\'edictif.

\FloatBarrier
\subsection{Classification avec la librairie Mclust}% On peut utiliser sweaver

C'est une id\'ee de Nicolas Glade (TIMC). Elle consiste \g{a} classer les pixels d'une image IRM en niveaux de gris. %
Ceux-ci doivent \^etre distribu\'es comme un m\'elange gaussien : %
autrement dit, il doit exister plusieurs sous populations de pixels dont les niveaux de gris sont distribu\'ees comme des \'echantillons gaussiens.

\par
La finction mclust utilise pour cela la m\'ethode du maximum de vraisemblance.
\par
La figure \ref{exem_ADC_19} illustre la classification des niveaux de gris d'une IRM en ADC.

\begin{figure}[H]
\includegraphics[width=0.8\linewidth,height=13cm]{../../images_rapport/19-J00-ADC-cerveau_clhist_clust.pdf}
\caption{Classification des valeurs d'ADC suivant un mod\g{e}le de m\'elange gaussien, avec quatre composantes.
\\
Les classes de pixels ainsi obtenues correspondent \g{a} une r\'ealit\'e physiologique.
}
\label{exem_ADC_19}
\end{figure}

\par
Les deux premiers clusters, en bleu et en rouge, de la classification correspondent aux deux pics, bien visibles, sur l'histogramme de la figure \ref{exem_ADC_19}. %
Les distributions d'ADC peuvent \^etre r\'esum\'ees par les diagrammes en bo\^ite de la figure \ref{19_box_lch00}, %
et qui concernent respectivement la r\'egion l\'es\'ee et l'h\'emisph\g{e}re contralat\'eral.

\par
L'emplacement et la forme du cluster en bleu \'evoque les zones sombres visibles en ADC, BVf ou CBF sur les images de la figure \ref{19_box_lch00} %
et dont la derni\g{e}re a servi \g{a} d\'efinir la l\'esion -voir \ref{cbf_seg_19}.

\par
Le troisi\g{e}me cluster du m\'elange correspond \g{a} un m\'elange de diff\'erentes structures. %
En effet, l'existence de deux composantes connexes sur le graphique laisse penser que ce cluster pourrait \^etre encore d\'ecompos\'e. %
Peut-\^etre touche-t-on d\g{e}s lors aux limites de la d\'ecomposition en densit\'es gaussiennes, %
ou que le nombre de clusters choisi pourrait \^etre plus important.

\par
Le dernier cluster est principalement compos\'e des pixels avec la diffusivit\'e la plus forte.

\begin{figure}[!p]
\begin{center}
\begin{tabular}{|c|c|}
\hline
\subfloat[BVf : segmentation de la l\'esion en bleu, %
parasites sur l'h\'emisph\g{e}re contralat\'eral.]{%
\includegraphics[height=7cm,width=0.4\linewidth]{../../images_rapport/19-J00-BVf-cerveau_clust.pdf}%bon
}
&
\subfloat[T1map : structures visibles, %
mais sans lien avec la l\'esion.]{%
\includegraphics[height=7cm,width=0.4\linewidth]{../../images_rapport/19-J00-T1map-cerveau_clust.pdf}%
}
\\
\hline
\subfloat[CBF : le cluster bleu correspond \g{a} la r\'egion segment\'ee de la figure \ref{cbf_seg_19}. %
Clusters vert et rouge difficiles \g{a} interpr\'eter.]{%
\includegraphics[height=7cm,width=0.4\linewidth]{../../images_rapport/19-J00-CBF-cerveau_clust.pdf}%bon
}
&
\subfloat[CMRO2 : mauvaise qualit\'e, taches bleues que l'h\'emisph\g{e}re contralat\'eral.]{%
\includegraphics[height=7cm,width=0.4\linewidth]{../../images_rapport/19-J00-CMRO2-cerveau_clust.pdf}%
}
\\
\hline
\subfloat[ADC : l\'esion d\'elimit\'ee au jour 22]{%
\includegraphics[height=7cm,width=0.4\linewidth]{../../images_rapport/19-J22-ADC-cerveau_clust.pdf}%bon
}
&
\subfloat[ADC : pas de l\'esion d\'elimit\'ee pour le nour 3.]{%
\includegraphics[height=7cm,width=0.4\linewidth]{../../images_rapport/19-J03-ADC-cerveau_clust.pdf}%
}
\\
\hline
\end{tabular}
\end{center}
\caption{Exemples de classification par m\'elange gaussien.%
\\%
A gauche : la r\'egion isch\'emi\'ee est visible.
}
\label{19_pour_contre}
\end{figure}

\etoile
\ref{19_CBF-BVf_3d00}

\subsubsection{Evaluation de la m\'ethode de classification}

Dans ce paragraphe, on compare l'efficacit\'e de deux m\'ethodes qui permettent de d\'elimiter un tissu isch\'emi\'e :
\begin{itemize}
\item La segmentation manuelle, effectu\'ee avec ImageJ -voir figure \ref{cbf_seg_19} ;
\item La classification avec Mclust, comme dans le paragraphe pr\'ec\'edent.
\end{itemize}

\par
La figure \ref{pour_contre} regroupe, sous forme graphique, des r\'esultats de classification. %
Des tissus isch\'emi\'es sont bien visibles pôur la colonne de gauche, pour les jours 0 et 22 ;
ce n'est pas le cas pour les figures de la colonne de droite.

\par
On peut opposer plusiers arguments concernant l'avantage, ou non, %
d'une d\'elimitation syst\'ematique de la r\'egion l\'es\'ee sur une image obtenue \g{a} la suite des diff\'rents examens.
\begin{enumerate}[label=\textbf{(Pour\arabic*)}]
\item\label{19_suivi_temp} Possibilit\'e d'une utilisation sur les jours autres que 00, notamment sur la deuxi\g{e}me moiti\'e du mois que durent les exp\'erimentations ;
\item\label{phy_gau} Eventuellement, signification physiologique des composantes du m\'elange gaussien, %
tandis que l'oeil lit seulement des valeurs particuli\g{e}rement basses de la variable \'etudi\'ee.
\end{enumerate}

Le point \ref{suivi_temp} laisse esp\'erer, avec un choix judicieux de variable -CBF, ADC ?- %
que l'utilisation de Mclust pourrait permettre de suivre l'\'evolution spatiale de la r\'egion l\'es\'ee. %
Toutefois, la situation que l'on peut lire en bas \g{a} gauche du tableau \ref{19_pour_contre} reste exceptionnelle, %
de plus, elle ne donne pas une caract\'erisation de la partie l\'es\'ee significativement diff\'erente de celle effectu\'ee au d\'ebut de cette section.

\etoile
Voici des arguments qui s'opposent \g{a} l'utilisation de Mclust pour caract\'eriser la r\'egion l\'es\'ee, \g{a} partir de coupes enti\g{e}res de cerveau :

\begin{enumerate}[label=\textbf{(Contre\arabic*)}]
\item Beaucoup d'images sont de mauvaise qualit\'e : voir le CBF ou la CMRO2 sur la figure \ref{19_pour_contre} ;
%
\item Les composantes d'une classification peuvent correspondre \g{a} des structures anatomiques comme nous l'avons vu dasn le premier graphique de cette sous-section, %
mais celles-ci ne correspondent pas n\'ecessairement \g{a} la partie l\'es\'ee : %
voir \g{a} ce titre la classification des valeurs mesur\'ees en T1, pour le rat num\'ero 19 au jour 00. %
Ceci va \g{a} l'encontre de l'argument \ref{phy_gau}, comme \'etant en faveur de l'utiliation de Mclust.
%
\item La classification automatique, non supervis\'ee, regroupe des pixels correspondant peu ou prou \g{a} la partie isch\'emi\'ee et des pixels de l'h\'emisph\g{e}re contralat\'eral. %
Ici encore, le jugement humain est plus pertient.
\end{enumerate}

\ligneinter
En conclusion, la classification des niveaux de gris d'une image par m\'elanges gaussiens n'est pas plus efficace que la segmentation manuelle, %
pour d\'elimiter la r\'egion l\'es\'ee du cerveau de l'un des rats de l'exp\'erience.
%apr\g{e}s avoir consacr\'e un certain temps \g{a} la classification des niveaux de gris des images dont je disposais, %
%correspondant aux valeurs des diff\'erentes variables scalaires d\'efinies sur des structures tridimensionnelles, %
%j'ai d\'ecid\'e de ne pas utiliser le paquet R Mclust pour la caract\'erisation et le suivi des l\'esion au sein des cerveaux des rats utilis\'es pendant exp\'erience, %
%faute d'un nombre suffisant, et parfois de la pertinence, des r\'esultats obtenus, %
%comparativement \g{a} ceux de la sous-section pr\'ec\'edente qui \'etaient issus d'une segmentation \g{a} l'oeil.

\par
Toutefois, la classification des niveaux de gris, restreinte \g{a} l'image de a partie l\'es\'ee, donne des r\'esultats int\'eressants.

\FloatBarrier
\subsubsection{Classification des pixels de la l\'esion}

Notons ici que la fonction mclust prend en argument, outre la base de données à classifier,
un vecteur constitué d'entiers positifs qui paramètre une famille de modèles que mclust se propose de calculer.

\par
Ainsi, la fonction calcule \emph{plusieurs} mélanges gaussiens, un pour chacun des entiers du vecteur pris en argument,
puis sélectionne le meilleur modèle avec le critère BIC.

\par
Il est d'ailleurs possibles de demander à R d'inclure, dans une représentation graphique,
les différentes valeurs de BIC associées à chaque possibilité de nombre de clusters.

\par
On peut ainsi visualiser la pertinence du choix d'un nombre de clusters particulier pour la classification.
%De telles représentations graphiques apparaissent dans les figures \ref{19_choix_clust_les}...

\begin{figure}[!p]
\begin{center}
\begin{tabular}{|c|c|}
\hline
\subfloat[CBF : donn\'ees manquantes, trois clusters]{%
\includegraphics[height=7cm,width=0.4\linewidth]{../../images_rapport/19-J00-CBF_clust1-3_lesion.pdf}
}
&
\subfloat[BVf : trois clusters.]{%
\includegraphics[height=7cm,width=0.4\linewidth]{../../images_rapport/19-J00-BVf_clust1-3_lesion.pdf}
}
\\
\hline
\subfloat[CBF : deux clusters.]{%
\includegraphics[height=7cm,width=0.4\linewidth]{../../images_rapport/19-J03-CBF_clust1-3_lesion.pdf}
}
&
%\subfloat[CMRO2 : mauvaise qualit\'e, taches bleues que l'h\'emisph\g{e}re contralat\'eral.]{%
%\includegraphics[height=7cm,width=0.4\linewidth]{../../images_rapport/19-J03-BVf_clust1-3_lesion.pdf}
%}
\\
\hline
\subfloat[CBF : deux clusters.]{%
\includegraphics[height=7cm,width=0.4\linewidth]{../../images_rapport/19-J08-CBF_clust1-3_lesion.pdf}
}
&
\subfloat[BVf : trois clusters.]{%
\includegraphics[height=7cm,width=0.4\linewidth]{../../images_rapport/19-J08-BVf_clust1-3_lesion.pdf}
}
\\
\hline
\end{tabular}
\end{center}
\caption{Classification sur des images de la l\'esion en CBF et BVf.}
\label{19_choix_clust_les}
\end{figure}

Les graphiques de la figure \ref{19_choix_clust_les} repr\'esentent la classification des niveaux de gris sur des images, en CBF et en BVf de la l\'esion du rat 19.

\par
Dans chaque cas, mclust avait le choix entre 1, 2 ou 3 clusters. Dans deux cas sur cinq, une classification \g{a} deux clusters est optimale, %
dans les trois autres cas, le nombre optimal de clusters est 3. %
Dans ces derniers cas, la diff\'erence de BIC entre les mod\g{e}les \g{a} 2 et \g{a} 3 clusters est faible : %
elle est au moins trois fois inf\'erieure \g{a} la diff\'erence de BIC entre les mod\g{e}les \g{a} deux ou un seul cluster(s).

\begin{figure}[!p]
\begin{center}
\begin{tabular}{|c|c|}
\hline
%\subfloat[CBF : donn\'ees manquantes, trois clusters]
&
%\subfloat[BVf : trois clusters.]
\\
\hline
%\subfloat[CBF : deux clusters.]
&
%\subfloat[CBF : deux clusters.]
\\
\hline
%\subfloat[CBF : deux clusters.]
&
%\subfloat[BVf : deux clusters.]
\\
\hline
\end{tabular}
\end{center}
\caption{Classification des niveaux de gris pour la partie l\'es\'ee : comparaison entre plusieurs jours et variables}
\label{19_reg_clust_les}
\end{figure}

%%% Broder %%%
Sur la figure \ref{19_reg_clust_les}, on repr\'esente la classification \g{a} deux clusters des niveaux de gris, %
pour des images en CBF, CMRO2 et SO2map de la l\'esion du rat num\'ero 19.

\par
On trouve des formes identiques, et un nombre important de pixels en commun, entre les clusters correspondant \g{a} un m\^eme jour et deux variables diff\'erentes %
-except\'ee la SO2 au jour 22 car peu de donn\'ees sont disponibles.

\par
Ces images sugg\g{e}rent l'existence d'au moins deux sous-populations de cellules nerveuses au sein de la l\'esion du rat num\'ero 19. %
L'\'evolution, pour ces deux types de cellules, des valeurs des six variables ADC, BVf, CBF, CMRO2, SO2 et VSI sont l'objet du paragraphe suivant.

\FloatBarrier
\subsection{Evolution des diff\'erentes variables, sur deux sous-ensembles de la partie l\'es\'ee.}%Vers un mod\g{e}le pour le rat num\'ero 19}

Les tableaux de la figure \ref{19_suivi_clust_les} permettent de comparer les valeurs prises par les six variables ADC \dots VSI %
sur trois r\'egions diff\'erentes de la coupe num\'ero 9 du cerveau du rat 19 :
\begin{itemize}
\item L'h\'emisph\g{e}re contralat\'eral, dont les niveaux de gris sont repr\'esent\'es en bleu clair ;
\end{itemize}

et deux sous-ensembles de la r\'egion segment\'ee :
\begin{itemize}
\item Les pixels pour lesquels le CBF, au jour 8, est inf\'erieur \g{a} 100 mL.min${}^{-1}$.(100g)${}^{-1}$ ;
\item Les pixels pour lesquels le CBF, au jour 8, est sup\'erieur \g{a} 100 mL.min${}^{-1}$.(100g)${}^{-1}$.
\end{itemize}

Ces deux sous-r\'egions de la partie isch\'emi\'ee sont respectivement appel\'es l\'esion 1 et l\'esion 2 sur la l\'egende des graphiques, %
les diagrammes en bo\^ite qui leur correspondent sont respectvement rouges et jaunes.

\par
Remarquons que la largeur des bo\^ites dans les graphiques de la figure \ref{19_suivi_clust_les} %
est proportionnelle au nombre de pixels de chacune des trois r\'egions d\'efinies pr\'ec\'edemment.

\par
On peut donc voir que les deux sous-r\'egions de la partie l\'es\'ee ont des tailles voisines, en nombre de pixels.

\begin{figure}[!p]
\begin{center}
\begin{tabular}{|c|c|}
\hline
%\subfloat[BVf : d\'elimitation approximative, parasites.]
&
%\subfloat[T1map : structures visibles, %
%mais sans lien avec la l\'esion.]
\\
\hline
%\subfloat[CBF : pas mieux que l'oeil.. %
%Clusters vert et rouge difficiles \g{a} interpr\'eter.]
&
%\subfloat[CMRO2 : mauvaise qualit\'e, taches bleues que l'h\'emisph\g{e}re contralat\'eral.]
\\
\hline
%\subfloat[ADC : r\'esultat encourageant]
&
%\subfloat[CBF : pas exploitable pour distinguer la r\'egion l\'es\'ee, sur le cerveau entier.]
\\
\hline
\end{tabular}
\end{center}
\caption{Suivi pour le rat 19, pour six variables diff\'erentes.}
\label{19_suivi_clust_les}
\end{figure}

\FloatBarrier
\subsection{Mod\g{e}le ph\'enom\'enologique : \'etat d'un pixel en fonction du temps.}

Il est difficile, d'entr\'ee de jeu, de construire un mod\g{e}le physiologique de tissu isch\'emi\'e. %
Un tel mod\g{e}le doit rendre compte, \g{a} partir de m\'ecanismes physiques et chimiques qui se d\'eroulent \g{a} l'\'echelle de la cellule, %
de l'\'evolution dans le temps du tissu isch\'emi\'e. %
Un tel mod\g{e}le existe pour la vascularisation d'une tumeur, voir \cite{Kelly_PMB_06}.

\par
A l'aide des donn\'ees issues des paragraphes pr\'ec\'edents, nous allons mod\'eliser la plus petite quantit\'e de tissu isch\'emi\'e visible \g{a} l'IRM : le voxel. %
Plus pr\'ecis\'ement, l'\'evolution des six variables ADC \dots VSI , suivie \g{a} l'\'echelle tissulaire avec les diagrammes %
des figures \ref{19_suivi_temp} et \ref{19_suivi_clust_les}, va \^etre reconstruite pixel par pixel dans des cas particuliers.

\par
Notons au passage que, sur toutes les repr\'esentations graphiques, les grandeurs BVf et VSI semblent corr\'el\'ees.

\par
Sur le plan physiologique, on suppose que la (BVf) d\'epend uniquement du volume %
des vaisseaux sanguins pr\'esents au d\'ebut de l'AVC, donc du VSI. %
Cette hypoth\g{e}se se fonde sur :
\begin{itemize}
\item L'absence de cr\'eation de nouveaux vaisseaux sanguins, contrairement au cas de l'angiogen\g{e}se provoqu\'ee par une tumeur ;
\item L'absence d'h\'emorragie interne pendant l'AVC isch\'emique.
\end{itemize}

\par
Dans toute la suite de cette section, les \'equations d'\'evolution de la BVf et du VSI seront li\'ees.

\subsubsection{Hypoth\g{e}ses simplificatrices}

L'objet dont on \'etudie l'\'evolution est le pixel.
\begin{description}
\item[Etat d'un pixel :] L'\'etat d'un pixel est enti\g{e}rement d\'etermin\'e par les valeurs prises, en ce pixel, des six variables ADC, BVf, CBF, CMRO2, SO2 et VSI ;
\item[Pixels voisins :] L'\'evolution de l'\'etat d'un pixel n'est pas influenc\'e par l'\'etat des autres pixels, y compris ses plus proches voisins ;
\item[Autonomie :] Les \'equations qui r\'egissent l'\'evolution de l'\'etat d'un pixel ne changent pas avec le temps.
\end{description}

La premi\g{e}re hypoth\g{e}se simplificatice exclut une perturbation exog\g{e}ne de l'\'etat d'un pixel, %
comme la r\'epercussion de la reperfusion post-isch\'emie sur un tissu sain. %
Les diagrammes de la figure \ref{19_suivi_clust_les} semblent contredire cette hypoth\g{e}se, concernant le CBF dans l'h\'emisph\g{e}re contralat\'eral.

\par
La deuxi\g{e}me hypoth\g{e}se ne concerne pas les stades pr\'ecoces de l'isch\'emie : ondes de d\'epolarisation, diffusion de l'ADC avec l'oed\g{e}me cytotoxique. %
D'ailleurs, les premi\g{e}res observations de cette exp\'erience sont effectu\'ees 30 minutes par\g{e}s la reperfusion.

\begin{equation}
\text{ADC}_{i,j}(t+1)=\frac{1}{9}\sum\limits_{\|k,l-i,j\|_{\infty}=1}\text{ADC}_{k,l}(t)%
\text{ Version discr\g{e}te de l'\'equation de diffusion de l'ADC. D'apr\g{e}s \cite{Duval_JCBFM_02}}
\end{equation}

%Le membre de droite correspond \g{a} la moyenne ces valeurs d'ADC \g{a} un instant discret $t$ : %
%celle du pixel consid\'er\'e $(i,j)$ et celles de ses $8$ voisins sont prises en compte, %
%Le membre de gauche est la valeur d'ADC au temps $t+1$.

\par
En fin, les \'equations qui r\'egissent l'\'evolution d'un pixel changent dans les jours et semaines qui suivent l'isch\'emie, %
ces changements sont conditionn\'es par les valeurs d'ADC, de BVf, \dots et de VSI.

\subsubsection{Premier r\'egime d'isch\'emie}

On distingue trois \'etapes pour l'\'evolution des valeurs d'ADC, \dots VSI sur les pixels du domaine \og{} l\'esion 1 \fg{}, %
ces valeurs sont repr\'esent\'ees en rouge sur les diagrammes de la figure \ref{19_choix_clust_les}.

\begin{enumerate}[label=\textbf{(L\'esion 1 - \arabic*)}]
\item Cette \'etape est commune aux pixels des deux r\'egions isch\'emi\'ees.

\par
Au cours des trois premiers jours de l'exp\'erience, on remarque un retour des valeurs de CBF, %
dont 75\% \'etaient inf\'erieures \g{a} 25 ml.min${}^{-1}$.(100g)${}^{-1}$, vers une distribution commune aux deux sous-r\'egions.

\par
Notons CBF${}_{trans}$ cette distribution transitoire. Nous supposerons par la suite, puur supplifier les calculs, %
que cette distribution est gaussienne, et que ses m\'ediane et quartiles sont lisibles sur la figure \ref{19_suivi_clust_les}. %
Dans le cadre de nos simulations, nous prendrons donc pour CBF${}_{trans}$ un \'echantillon de loi $\mathcal{N}(140,70)$.

\par
On mod\'elise le retour des valeurs de CBF \g{a} cette distribution par l'\'equation diff\'erentielle du premier ordre :
\begin{equation}
\der{CBF}=\frac{1}{\tau}\times \left(CBF_{trans}-CBF\right)% \Ta
\label{cbf_is1}
\end{equation}
pour un certain temps caract\'eristique $\tau$, que l'on exprime en jours.

\par
On d\'etermine la valeur de ce coefficient de sorte que 3 jours suffisent au CBF pour retrouver la distribution CBF${}_{trans}$ :
\begin{itemize}
\item Pour chaque pixel, la valeur de CBF \g{a} un temps $t$ -que l'on exprim\'era en jours- est donn\'ee par : %
$CBF=CBF_{trans}+e^{-\frac{t}{\tau}}\times\left(CBF_{trans}-CBF\right)$. La valeur $\tau$ est d\'efinie par l'in\'egalit\'e :
\[|m^0(t)-m^3|\leq|m^3-Q_1^3|\]
o\g{u} $m^0$ et $m^3$ sont respectivement les valeurs m\'edianes du CBF aux jours 0 et 3, lisibles sur la figure \ref{19_suivi_clust_les}, %
$Q_1^3$ le premier quartile du CBF au jour 3, et $m^0(t)$ la valeur de CBF obtenue en r\'esolvant l'\'equation \ref{cbf_is1} avec la valeur initiale $m^0$.

\par
L\'in\'egalit\'e pr\'ec\'edente se traduit par : $\dfrac{3}{\tau}\leq\ln\left(\dfrac{m^3-m^0}{m^3-Q^3_1}\right)$, %
le 3 de la premi\g{e}re fraction correspond au nombre de jours \'ecoul\'es.
\item On pose donc : $\tau =\dfrac{1}{3}\times\ln\left(\dfrac{m^3-m^0}{m^3-Q^3_1}\right)$.
\item Application num\'erique : $\tau \simeq 0.3 \text{(jour)}$.
\end{itemize}
De la m\^eme mani\g{e}re, on \'ecrit les \'equations satisfaites par SO2, CMRO2 et VSI :
\begin{equation}
\der{SO2}=2,5\times (SO2_{asy2}-SO2)% \Ta
\label{so2_s_l2}
\end{equation}
et
\begin{equation}
\der{CMRO2}=0,25\times (CMRO2_{trans}-CMRO2)% \Ta
\end{equation}
A ce stade, les variables CBF et VSI semblent corr\'el\'ees : on pose donc
\begin{equation}
\der{VSI}=0,15\times\der{CBF}
\label{vsi_is}
\end{equation}
pour pr\'edire l'\'evolution de VSI.

\par
Le coefficient 0,15 est choisi pour compenser la diff\'erence d'\'echelle des valeurs de CBF (de l'orde de 10${}^2$) et de VSI (1 \g{a} 10).

\par
Enfin, et conform\'ement au paragraphe pr\'ec\'edent, on relie l'\'evolution de la BVf et celle du VSI par :
\begin{equation}
\der{BVf}=\der{VSI}
\label{bvf_is}
\end{equation}
Pas de changement d'\'echelle ici.

\par
Les conditions initiales $SO2_{asy2}$ et $CMRO2_{trans}$ sont d\'efinies de la m\^eme mani\g{e}re que $CBF_{trans}$ : %
ce sont des distributions spatiales de CMRO2 et de SO2. %
Pour nos simulations, on prendra des \'echantillons gaussiens de lois respectives $\mathcal{N}(75,15)$ et $\mathcal{N}(6,7)$.
\item Apr\g{e}s un temps compris entre trois et huit jours, les \'equations satisfaites par les cinq variables ci-dessus changent : %
par exemple, les valeurs du CBF ne sont plus distribu\'ees comme la variable transitoire CBF${}_{trans}$.

\par
L'\'evolution du CBF sera d'ailleurs diff\'erente entre les deux r\'egions l\'es\'ees.

\par
La distribution que le CBF doit atteindre de fa\c con transitoire est donn\'ee par le diagramme de la figure \ref{19_suivi_clust_les} : %
voir la bo\^ite rouge au centre de la case \og{} CBF\fg{}. L'\'equation suivie par le CBF est :
\begin{equation}
\der{CBF}=0, 1\times \left(CBF_{trans1}-CBF\right)% \Ta
\end{equation}
CBF${}_{trans1}$ sera simul\'e selon la loi $\mathcal{N}(80,43)$.

\par
CMRO2, BVf et VSI \'evoluent de la m\^eme mani\g{e}re, au moment o\g{u} l'on commence \g{a} distinguer les deux r\'egions l\'es\'ees.

\par
Choisissons de mod\'eliser, en premier lieu, l'\'evolution de la CMRO2, qui est la plus facile \g{a} lire. %
La m\'ediane et les quartiles de cette variable, prise au jour 8 sont, dans l'ordre : 3, 5 et 7.

\par
Avec la la m\'ethode d\'etaill\'ee pr\'ec\'edemment pour le CBF, on peut \'ecrire l'\'equation :
\begin{equation}
\der{CMRO2}=0, 1\times \left(CMRO2_{trans1}-CMRO2\right)% \Ta
\end{equation}
CMRO2${}_{trans1}$ sera simul\'e selon la loi $\mathcal{N}(5,3)$.

VSI et BVf \'evoluent respectivement d'apr\g{e}s les \'equations :
\begin{equation}
\dot{VSI}=0,3\times\dot{CMRO2}
\label{vsi_is1_tr}
\end{equation}
\begin{equation}
\dot{BVf}=\dot{VSI}
\label{bvf_is1_tr}
\end{equation}

\par
\emph{Dans la suite de la section, on remplacera la notation $\der{V}$ par $\dot{V}$, pour une variable $V$ d\'ependant du temps.}
\item C'est le r\'egime asymptotique que l'on peut observer pour les variables BVf, \dots VSI sur les diagrammes de la figure \ref{19_suivi_clust_les}. %
Il concerne tous les pixels qui correspondent au jour 8, \g{a} des valeurs de CBF inf\'erieures \g{a} 100, et des valeurs de CMRO2 inf\'erieures \g{a} 10.

\par
En effet, on peut remarquer que la distribution des valeurs prises par ces variables sur la premi\g{e}re r\'egion l\'es\'ee se stabilise \g{a} partir du jour 15 (au moins).

\par
Les valeurs asymptotiques des variables BVf \dots VSI sont globalement inf\'erieures \g{a} celles qui caract\'erisent un tissu sain.

\par
BVf, CBF, CMRO2, SO2 et VSI satisfont les \'equations suivantes :
\begin{equation}
\dot{CBF}=5\times\left(CBF_{asy1}-CBF\right)
\end{equation}
\begin{equation}
\dot{CMRO2}=0,15\times\left(CMRO2_{asy1}-CMRO2\right)
\end{equation}
\begin{equation}
\dot{SO2}=1\times\left(SO2_{asy1}-SO2\right)
\end{equation}

Les deux variables VSI et BVf suivent un r\'egime commun, distinct de l'\'evolution des autres variables :
\begin{equation}
\dot{VSI}=0,2\times (VSI_{asy1}-VSI)
\end{equation}
et une derni\g{e}re \'equation, li\'ee \g{a} la pr\'ec\'edente :
\begin{equation}
\dot{BVf}=0,5\times\dot{VSI})
\end{equation}
Les m\'ethodes utilis\'ees pour obtenir ces \'equations sont les m\^emes que pr\'ec\'edemment, %
elles s'appuient essentiellement sur une observation des m\'edianes et quartiles des variables BVf, %
\dots VSI que l'on trouve sur les diagrammes de la figure \ref{19_suivi_clust_les}.

\par
$SO2_{asy1}$, $CBF_{asy1}$ et $VSI_{asy1}$ sont distribu\'es selon les lois $\mathcal{N}(75,25)$, $\mathcal{N}(50,40)$ et $\mathcal{N}(4,2)$.
\end{enumerate}

\subsubsection{Deuxi\g{e}me r\'egime d'isch\'emie}

\begin{enumerate}[label=\textbf{(L\'esion 2 - \arabic*)}]
\item Voir le paragraphe pr\'ec\'edent.
\item Cette fois-ci, les valeurs de CBF augmentent globalement. % : d'apr\g{e}s les figures \ref{19_suivi_clust_les} et \ref{19_reg_clust_les}, %
%CBF d\'epasse 100 ml.min${}^{-1}$.(100g)${}^{-1}$.

On peut \'ecrire l'\'equation suivante pour le CBF :
%on remplace simplement CBF${}_{trans1}$ par la variable spatiale CBF${}_{trans2}$. %
%Dans le cadre des simulations, les valeurs de CBF${}_{trans2}$ seront donn\'ees par un \'echantillon gaussien de loi $\mathcal{N}(220,43)$.
\begin{equation}
\der{CBF}=0,3\times \left(CBF_{trans2}-CBF\right)% \Ta
\label{cbf_is2}
\end{equation}
o\g{u} CBF${}_{trans1}$ par la variable spatiale CBF${}_{trans2}$. %
Cette r\'epartition spatiale de CBF est encore suppos\'ee gaussienne, on prendra $\mathcal{N}(220,43)$ pour les simulations.

\par
L'\'evolution de CMRO2, BVf et VSI s'effectue dans le m\^eme sens, et \g{a} la m\^eme vitesse que celle de CBF. On utilisera les \'equations :
\begin{equation}
\dot{VSI}=0,15\times\dot{CBF}
\label{vsi_is2_tr}
\end{equation}
\begin{equation}
\dot{BVf}=\dot{VSI}
\label{bvf_is2_tr}
\end{equation}
\begin{equation}
\dot{CMRO2}=0,2\times\dot{CBF}
\label{cmro2_is2_tr}
\end{equation}
\item %Le d\'epassement du seuil de CBF : 100 ml.min${}^{-1}$.(100g)${}^{-1}$, %
%et du seuil de 10 pour la CMRO2, caract\'erisent les pixels de la deuxi\g{e}me r\'egion l\'es\'ee : %
%voir la figure \ref{19_reg_clust_les}.
%
%\par
Afin de rendre compte de l'\'evolution des variables BVf, CBF, \dots VSI apr\g{e}s le jour 8, on a besoin d'un nouveau syst\g{e}me d'\'equations. %
Contrairement au cas pr\'ec\'edent, il ne s'agit visiblement pas d'un retour \g{a} l'\'equilibre.

\par
Par exemple, CBF et CMRO2 baissent entre les jours 8 et 15, et augmentent ensuite.

\par
Toujours avec la m\^eme m\'ethode, on construit les \'equations suivantes, %
qui r\'egissent l'\'evolution de BVf, CBF, CMRO2, SO2 et VSI :
\begin{equation}
\dot{CBF}=-15\times\left(CBF_{trans3}-CBF\right)
\end{equation}

La densit\'e transitoire $CBF_{trans3}$ du CBF est d\'etermin\'ee par les r\'esultats de \ref{19_suivi_clust_les}. %
Pour simplifier, on suppose qu'il s'agit d'un \'echantillon de loi $\mathcal{N}(120,40)$.

\par
On lie les \'equations d'evolution de CMRO2, VSI et BVf \g{a} celle du CBF :

\begin{equation}
\dot{CMRO2}=0,15\times\dot{CBF}
\end{equation}
\begin{equation}
\dot{VSI}=0,3\times (VSI_{asy2}-VSI)
\end{equation}
et% une derni\g{e}re \'equation, li\'ee \g{a} la pr\'ec\'edente :
\begin{equation}
\dot{BVf}=2\times\dot{VSI})
\end{equation}

Ces \'equations concernent les pixels o\g{u} les valeurs de 100 ml.min${}^{-1}$.(100g)${}^{-1}$ pour le CBF, %
et 10 pour la CMRO2, ont \'et\'e d\'epass\'ees.
%\item mm
\end{enumerate}



\subsubsection{Retour \g{a} l'\'equilibre : CBF dans l'h\'emisph\g{e}re contralat\'eral}

Sur la figure \ref{19_suivi_clust_les}, on observe des valeurs du CBF globalement plus basses pour le jour 00 que pour les autres jours : %
par exemple, la m\'ediane du CBF se situe en dessous de 100 ml min${}^{-1}$(100g)${}^{-1}$. %
Les valeurs du CBF aux jours suivants semblent osciller autour d'un \'equilibre.

\par
Avec la m\'ethode utilis\'ee aux paragraphes pr\'ec\'edents, on \'ecrit l'\'equation suivante pour l'\'evolution du CBF, %
en utilisant la figure \ref{19_suivi_clust_les}.

\begin{equation}
\dot{CBF}=3\times (CBF_{sain}-CBF)
\end{equation}


%%%%%%%%%%%%%%%%%%%%%%%%%%%%%%%%%%%%%%%%%%%%%%%%%%%%%%%%%%%%%%%%%%%%%%%%%%%%%%%%%%%%%%%%%%%%%%%%%%%%%%%%%%%%%%%%%%%%%%%%%%%%%%%%%%%%%%%%
%%%%%%%%%%%%%%%%%%%%%%%%%%%%%%%%%%%%%%%%%%%%%%%%%%%%%%%%%%%%%%%%%%%%%%%%%%%%%%%%%%%%%%%%%%%%%%%%%%%%%%%%%%%%%%%%%%%%%%%%%%%%%%%%%%%%%%%%
%%%%%%%%%%%%%%%%%%%%%%%%%%%%%%%%%%%%%%%%%%%%%%%%%%%%%%%%%%%%%%%%%%%%%%%%%%%%%%%%%%%%%%%%%%%%%%%%%%%%%%%%%%%%%%%%%%%%%%%%%%%%%%%%%%%%%%%%




\begin{comment}
L'hypoth\g{e}se \ref{nprog} distingue d'entr\'ee de jeu le mod\g{e}le de celui consid\'er\'e dans \cite{Duval_JCBFM_02} : %
dans ce dernier, le coefficient de diffusion ADC diffuse lui-m\^eme \g{a} travers les pixels, et respecte l'\'equation disscr\g{e}te :
\[\text{ADC}_{i,j}(t+1)=\frac{1}{9}\sum\limits_{\|k,l-i,j\|_{\infty}=1}\text{ADC}_{k,l}(t)\]



\par
L'hypoth\g{e}se \label{var_comp} doit \^etre affaibie pour deux raisons :
\begin{itemize}
\item J'ai d\'ecid\'e de prendre en compte des perturbations exog\g{e}nes des valeurs prises par les diff\'erentes variables, %
en particulier dans les r\'egimes d\'equilibre r\'egis par des \'equations stochastiques.
\item Sur les figures \ref{19_suivi_clust_les}, on note qu'il n'est pas possible de discerner les distributions de valeurs prises par les six variables choisies, %
sur les deux composantes d\'efinies par les valeurs de CBF au jour 8.
\end{itemize}

\par
Le dernier point qui pr\'ec\g{e}de donnera lieu \g{a} un affaiblissement de l'hypoth\g{e}se d'autonomie \ref{auto} : %
chaque pixel se trouvant initialement dans un \'etat correspondant \g{a} l'isch\'emie subira un changement d'\'etat, %
se produisant \g{a} un temps al\'eatoire compris entre les jours 3 et 8 de la simulation.

\par
Une telle \'echelle de temps, suivant laquelle les observations effectu\'ees sur le rat num\'ero 19 %
sugg\g{e}rent l'existence de deux r\'egipes diff\'erents pour les pixels repr\'esentant la l\'esion, %
co\"incide avec celle qui r\'egit les m\'ecanismes de formation d'oed\g{e}mes vasog\'eniques, %
et l'infiltration de cellules immunitaires parfois d\'el\'et\g{e}res dans le parenchyme c\'er\'ebral. %
L'approche qui consiste \g{a} introduire dans le mod\g{e}le de ce facteur temporel me para\^it conc pertinente, %
m\^eme si le lien avec les diff\'erentes \'etapes de la cascade isch\'emique n'est pas \'elucid\'e \g{a} ce stade.

\ligneinter
Je distingue ainsi sept \'etats : cinq pour les pixels isch\'emi\'es, deux pour les autres, possibles pour l'\'evolution.

\par
Chacun de ces \'etats est caract\'eris\'e par un ensemble de valeurs possibles pour les valeurs des six variables, %
et un syst\g{e}me d'\'equations d'\'evolution.

\par
Celles-ci, \'ecrites comme des \'equations diff\'erentielles ordinaires mettant en jeu les diff\'erentes variables avec les unit\'es vues \g{a} la section pr\'ec\'edente, %
seront impl\'ement\'ees en rempla\c cant les d\'eriv\'ees par des taux d'accroissement, l'unit\'e de temps adopt\'ee pour les \'equations \'etant le jour ; %
cette dur\'ee sera \'egalement l'unit\'e de temps \'el\'ementaire du mod\g{e}le discret, impl\'ement\'e dans la suite de ce travail.

\etoile
On note que l'hypoth\g{e}se de non progressivit\'e, que je conserve dans toute la sute de ce travail, %
revient \g{a} utiliser des automates pour le mod\g{e}le dynamique, et non des automates cellulaires. %
%Une critique de l'hypoth\g{e}se \ref{nprog} peut \^etre faite \g{a} la lumi\g{e}re de faits physiologiques reconnus.
\end{comment}

\begin{comment}
\begin{description}

%
%
\item[L\'esion 2 : \'evolution finale. ] Elle est d\'ecr\'et\'ee par le retour \g{a} la normale de la SO2.
%%% Equations : début. %%%
\par
L'ADC suit l'\'equation $\Ta$ :
\begin{equation}
\dot{ADC}=\tau_1ADC^{\alpha_1}
\end{equation}

La SO2 d\'evie alors l\'eg\g{e}rement de la normale, et suit une \'equation $\Ta$ :
\begin{equation}
\dot{SO2}=1\times (SO2_{asy2}-SO2)
\end{equation}

$SO2_{asy2}$ est issue des mesures exp\'erimentales, qui sont r\'esum\'ees dans la figure \ref{19_suivi_clust_les}.
\par
Dans ce mod\g{e}le ph\'enom\'enologique, l'\'evolution de SO2 dicte celles de CBF et de la CMRO2 :
\begin{equation}
\dot{CBF}=-10\times\dot{SO2}
\end{equation}
\begin{equation}
\dot{CMRO2}=-1\times\dot{SO2}
\end{equation}
Enfin le VSI et la BVf suivent ensemble un r\'egime $\Tb$.
%%% Fin. %%%
%
\item[R\'egime sain perturb\'e] l'\'evolution des valeurs, calcul\'ees ou mesur\'ees, pour les six variables ADC \dots VSI %
ne se r\'esume par \g{a} des oscillations autour d'un \'equilibre. Deux explications sont possibles pour ces \'evolutions :
\begin{itemize}
\item L'existence de m\'ecanismes de r\'epercussion progressive des \'ev\g{e}nements subis par la partie isch\'emi\'ee, %
dans l'ensemble du cerveau. Un exemple de tel m\'ecanisme est bien connu : les ondes de d\'epolarisation, qui apparaissent dans les minutes qui suivent l'isch\'emie.
%
\item La r\'epercussion, sur le d\'ebit sanguin des r\'egions non l\'es\'ee, de la reperfusion de la partie isch\'emi\'ee.
\end{itemize}

\par
Dans le cadre de ce mod\g{e}le, je privil\'egie la seconde explication : par \ref{nprog}, %
les effets progressifs de changements d'\'etat des pixels se cantonnent aux premi\g{e}res minutes de l'isch\'emie, %
les perturbations ressenties localement en dehors de la zone l\'es\'ee sont alors de nature exog\g{e}ne.

\par
Dans un souci de simplicit\'e, je distingue, dans le mod\g{e}le du tissu sain, %
un retour \g{a} des valeurs normales, puis des fluctuations autour de l'\'equilibre, pour le CBF.
%%% Equations : début. %%%


Autres variables : r\'egime $\Tb$.
%%% Fin. %%%
\item[Etat sain permanent : ] il est annonc\'e par une valeur de CBF sup\'erieure ou \'egale \g{a} 100. %
Toutes les variables, ADC comprise, suivent le r\'egime bruit\'e $\Tb$.

\par
Dans le cas du tissu sain, le bruit est particuli\g{e}rment important pour la plupart des variables. %
La simplification d\'ecid\'ee ci-dessus est discutable, eu \'egard l'\'evolution de la CMRO2.
\end{description}


\ligneinter
Les diff\'erents r\'egimes mis en \'evidence dans cette sous-section, et les \'equations qui les d\'efinissent satisfont %
un jeu d'hypoth\g{e}ses plus faible que celui du mod\g{e}le 1 qui pr\'ec\g{e}de. En voici un r\'esum\'e :

\begin{modmerate}{Mod\g{e}le 2}{\textbf{(H${}^{\ast}$\arabic*)}}
\item Identique \g{a} \ref{nprog}.
\item\label{var_compf} Le mod\g{e}le d\'efini par les \'equations qui pr\'ec\g{e}dent, %
qui contiennent en plus des six variables, \emph{une composante stochastique}, est bien pos\'e.
\item\label{autof} Les \'equations du mod\g{e}le sont autonomes, sauf celles du r\'egime suivi par les pixels situ\'es sur la l\'esion initiale %
qui incorporent une dur\'ee fix\'ee initialement. %
Dans le cadre des simpulations cette dur\'ee est d\'efinie al\'eatoirement sur les pixels qui sont entr\'es dans l'\'etat isch\'emi\'e.
\end{modmerate}
\end{comment}

\begin{comment}
\newpage
\subsubsection{Constitution d'une base de donn\'ees pour les conditions initiales}

Les donn\'ees collect\'ees, pour le jour 0 comme pour les autres, sur le rat num\'ero 19 %
et pour les six modalit\'es choisies ADC, BVf, CBF, CMRO2, SO2map et VSI peuvent facilement \^etre combin\'ees, %
pour former la condition initiale des simulations.

\par
Toutefois, cette approche na\"ive donne un r\'esultat tr\g{e}s insuffisant \g{a} la r\'ealisation de simulations du mod\g{e}le num\'ero 2 : %
voir \g{a} ce tire la figure \ref{sim_ini}, partie gauche.

\par
En effet, la base de donn\'ees multiparam\'etrique est d\'efinie sur l'intersection des ensembles de pixels disponibles pour les six modalit\'es, %
d'o\g{u} un manque de donn\'ees pour commencer les calculs.

\par
Afin de constituer une condition initiale utilisable \g{a} partir des donn\'ees du jour 0, j'ai compl\'et\'e les vecteurs contenant certaines variables %
- voir la figure pour CBF, SO2 et CMRO2 - avec des valeurs distribu\'ees suivant une loi gaussiene, reprenant, pour chaque variable, les moyenne et \'ecart-type des valeurs effectivement mesur\'ees. %
Ainsi, de nombreuses valeurs, certes virtuelles, se sont trouv\'ees en vis-\g{a} vis, constituant une base de donn\'ees plus touffue comme dans la deuxi\g{e}me figure \ref{sim_ini}.

%\par
%Les distributions 

\begin{figure}[!p]
%\begin{center}
\begin{tabular}{|c|c|}
\hline
\subfloat[Donn\'ee brute]{\includegraphics[height=7cm,width=0.4\linewidth]{../../images_rapport/19-J00-modele2_simBrut.pdf}}
&
\subfloat[Compl\'etion de CBF, SO2 et CMRO2]{\includegraphics[height=7cm,width=0.4\linewidth]{../../images_rapport/19-J00-modele2_simCom.pdf}}
\\
\hline
\end{tabular}
%\end{center}
\caption{Conditions initiales pour le rat num\'ero 19.
\\
Les pixels en r\'egime cytotoxique sont color\'es en rouge sombre, %
les autres, en bleu, sont dans un \'etat perturb\'e.
}
\label{sim_ini}
\end{figure}
\end{comment}

\begin{comment}
\subsubsection{Evolution \g{a} partir du jour 8.}

Si l'on choisit de d\'emarrer les simulations au jour 8, pour lequel on dispose de donn\'es mesur\'ees ou calcul\'ees \g{a} partir d'un examen IRM, %
on peut renforcer la deuxi\g{e}me hypoth\g{e}se en laissant, comme seules variables autres que les six \'etudi\'ees, les bruits des \'equations $\Tb$.

\par
En effet, 

\par
L'\'etape isch\'emique transitoire du mod\g{e}le 2 est \'egalement superflue, %
puisque l'\'evolution des pixels peut seulement transiter entre les \'etats l\'esion 1 d\'ebut, 1 asymptotique etc. %
On peut ainsi reprendre la version forte \ref{auto} de l'hypoth\g{e}se d'autonomie du mod\g{e}le.

\begin{modmerate}{Mod\g{e}le 1 bruit\'e, alias \og{} 3/2\fg{}}{\textbf{(H\arabic*)}}
\item L'\'evolution de l'\'etat d'un voxel n'est pas affect\'ee par l'\'etat des voxels voisins.
\item ${}_\mathbb{P}$ : pas de variable cach\'ee, seulement un bruit exog\g{e}ne.
\item L'\'evolution du syst\g{e}me est autonome.
%\item\label{bvf...}
\end{modmerate}

Le jeu d'hypoth\g{e}ses qui pr\'ec\g{e}de, plus fort que celui du mod\g{e}le 2 mais moins que celui du mod\g{e}le 1, m\'erite la d\'enomination \og{} 3demi\fg{}.

\etoile
Bien s\^ur, les simulations fond\'ees sur ce mod\g{e}le n'ont pas d'int\'er\^et pratique comparable \g{a} celles qui commencent au jour 0. %
%Toutefois, 


\begin{figure}[!p]
\begin{center}
%\begin{tabular}{|c|c|}
%\hline
%\subfloat[Compl\'etion de CBF, SO2 et CMRO2]
% etc.
%%% Ajouter une image recompos\'ee par gaussiennes ? %%%
%\\
%\hline
%\end{tabular}
\end{center}
\caption{Condition initiale pour le mod\g{e}le 3demi : donn\'ees brutes au jour 08. Tranches 9 et 10.}
\label{sim_ini_18}
\end{figure}
\end{comment}
%\fbox{\rule[-0.4cm]{0cm}{1cm} Une boite créée avec \verb!fbox! et aérée avec \verb!rule!.}