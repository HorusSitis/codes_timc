\section{M\'ethodes}

%%%% debut macro %%%%
\makeatletter
\renewcommand{\thefigure}{\ifnum \c@section>\z@ \thesection.\fi
 \@arabic\c@figure}
\@addtoreset{figure}{section}
\makeatother
%%%% fin macro %%%%
\begin{comment}
Remarque : pour renuméroter les sous-figures de la même manière
           (avec le package 'subfigure'), il suffit de rajouter
	   la ligne \let\p@subfigure\thefigure dans le préambule.
\end{comment}

\subsection{Mod\g{e}le exp\'erimental}

Ces exp\'eriences ont \'et\'e men\'ees \g{a} l'Institut de Neurosciences de Grenoble (GIN) par Benjamin Lemasson.
%, %
%et m'ont \'et\'e transmises \g{a} la suite d'une r\'eunion \g{a} laquelle j'ai assist\'e au GIN %
%avec Ang\'elique St\'ephanou, Emmanuel Barbier, et Benjamin Lemasson.

\cite{Lem_PHD_10}

\subsubsection{Chirurgie et examens}% trop flou

Occlusion intraluminale.

% Sessions d'examen : jours 00, 03 etc.

%\par
Chaque session d'examens consistait \g{a} produire $13$ images, en coupe frontale, de la t\^ete du sujet, %
sur une longueur correspondant au cerveau du rat, et pour les modalit\'es qui suivent :

\subsubsection{G\'en\'eration des images multiparam\'etriques}%, fiabilit\'e, probl\g{e}mes rencontr\'es}





\begin{figure}[!h]
\begin{tabular}{|c|c|c|c|}
\hline
\includegraphics[width=0.2\linewidth,height=3cm]{../../images_rapport/11-J03-Coreg01_Anat-masked-slice-10.jpg}
&
\subfloat[ADC]{\includegraphics[width=0.2\linewidth,height=3cm]{../../images_rapport/11-J03-CoregADC-slice-10.jpg}}
&
\subfloat[BVf]{\includegraphics[width=0.2\linewidth,height=3cm]{../../images_rapport/11-J03-CoregBVf-slice-10.jpg}}
&
\subfloat[CBF]{\includegraphics[width=0.2\linewidth,height=3cm]{../../images_rapport/11-J03-CoregCBF-slice-10.jpg}}
\\
\hline
\subfloat[CMRO2]{\includegraphics[width=0.2\linewidth,height=3cm]{../../images_rapport/11-J03-CoregCMRO2-slice-10.jpg}}
&
\subfloat[SO2map]{\includegraphics[width=0.2\linewidth,height=3cm]{../../images_rapport/11-J03-CoregSO2map-slice-10.jpg}}
&
\subfloat[T1map]{\includegraphics[width=0.2\linewidth,height=3cm]{../../images_rapport/11-J03-CoregT1map-slice-10.jpg}}
&
\subfloat[VSI]{\includegraphics[width=0.2\linewidth,height=3cm]{../../images_rapport/11-J03-CoregVSI-slice-10.jpg}}
\\
\hline
\end{tabular}
\caption{Images disponibles pour le rat 11, au jour 03, coupe num\'ero 10.
\\%\par
De gauche \g{a} droite, et de haut en bas : %
l'image brute en T2 \og{} Anatomique\fg{}, l'ADC, le BVf, le CBF, la CMRO2, la SO2, l'image en T1 et le VSI.}
\label{ex_irm_multipar}
\end{figure}

\FloatBarrier
\subsection{Segmentation des images avec ImageJ}

Les images en niveau de gris ont été pr\'ealablement trait\'ees \g{a} l'aide du logisiel ImageJ couramment utilis\'e dans la manipulation d'images.
%
\par
A partir des images, au format .tif, %transmises par Benjamin Lemasson, %
ImageJ a \'et\'e untilis\'e pour d\'elimiter les r\'egions correspondant au cerveau, pour toutes les modalit\'es -voir figure \ref{cephcer}.
\par
Ma d\'emarche a \'et\'e la suivante :
\begin{enumerate}
\item Tracer manuellement le contour des cerveaux sur les images anatomiques, prises en T2, figure \ref{cephcer} ;
\item Charger individuellement ces contours sur les images correspondant aux autres modalit\'es, ici l'ADC : les deux images centrales de la figure ;
\item Retirer toutes les valeurs correspondant \g{a} l'ext\'erieur du contour : voir l'image segment\'ee de la figure \ref{cephcer} ;
\item Automatiser la proc\'edure pour traiter et trier, \g{a} partir des contours d\'efinis manuellement, toutes les images disponibles.
\end{enumerate}

%\newcounter{stock}
\setcounter{stock}{\value{enumi}}

\begin{figure}[!h]%Ajouter les colonnes correspondant à anatomique, ADC sans contour ce ADC segmentée.
\subfloat[Image en anatomique]{\includegraphics[width=0.2\linewidth,height=4cm]{../../images_rapport/11-J03-Coreg01_Anat-masked-slice-10.jpg}}
\hfill
\subfloat[Image brute en ADC]{\includegraphics[width=0.2\linewidth,height=4cm]{../../images_rapport/11-J03-CoregADC-slice-10.jpg}}
\hfill
\subfloat[Image + contour]{\includegraphics[width=0.2\linewidth,height=4cm]{../../images_rapport/11-J03-segADC-slice-10.jpg}}
\hfill
\subfloat[Image segment\'ee]{\includegraphics[width=0.2\linewidth,height=4cm]{../../images_rapport/11-J03-ADC-bg-slice10.jpg}}
\caption{D\'elimitation du cerveau du rat num\'ero 11, jour 03, avec la modalit\'e ADC. On utilise les images anatomiques prises en ...}% en T2 ?
\label{cephcer}
\end{figure}

\FloatBarrier
Des images ont ainsi \'et\'e obtenues, par coupe, pour toutes les modalit\'es, restreintes \g{a} l'enc\'ephale. L'\'etape finale pour l'exploitation des donn\'ees IRM a \'et\'e :
\begin{enumerate}
\setcounter{enumi}{\value{stock}}
\item Regrouper les niveaux de gris, qui correspondent aux valeurs mesur\'ees ou calcul\'ees de chacune des images, dans un fichier texte exploitable sous R.
\end{enumerate}

\etoile
Les images aux coupes : 10 pour le rat num\'ero 11, 9 et 10 pour le rat 19, 6 \g{a} 8 pour le rat 26 et 11 \g{a} 13 pour le rat 30 sont disponibles pour toutes les modalit\'es, %
pour tous les jours d'examens. %
Ces coupes sont donc utilisables pour effectuer un suivi temporel des valeurs mesur\'ees ou calcul\'ees, pixel par pixel, %
ou globalement sur les coupes enti\g{e}res ou des r\'egions d'int\'er\^et.

\begin{figure}[!p]
\begin{center}
\begin{tabular}{|c|c|c|c|}
\hline
\subfloat[Anatomique]{\includegraphics[width=0.2\linewidth,height=2cm]{../../images_rapport/11-J00-Coreg01_Anat-masked-Cropped-slice10.jpg}}
&
\subfloat[ADC]{\includegraphics[width=0.2\linewidth,height=2cm]{../../images_rapport/11-J00-ADC-Cropped-slice10.jpg}}
&
\subfloat[BVf]{\includegraphics[width=0.2\linewidth,height=2cm]{../../images_rapport/11-J00-BVf-Cropped-slice10.jpg}}
&
\subfloat[CBF]{\includegraphics[width=0.2\linewidth,height=2cm]{../../images_rapport/11-J00-CBF-Cropped-slice10.jpg}}
\\
\hline
\subfloat[CMRO2]{\includegraphics[width=0.2\linewidth,height=2cm]{../../images_rapport/11-J00-CMRO2-Cropped-slice10.jpg}}
&
\subfloat[SO2map]{\includegraphics[width=0.2\linewidth,height=2cm]{../../images_rapport/11-J00-SO2map-Cropped-slice10.jpg}}
&
\subfloat[T1map]{\includegraphics[width=0.2\linewidth,height=2cm]{../../images_rapport/11-J00-T1map-Cropped-slice10.jpg}}
&
\subfloat[VSI]{\includegraphics[width=0.2\linewidth,height=2cm]{../../images_rapport/11-J00-VSI-Cropped-slice10.jpg}}
\\
\hline
\end{tabular}
\end{center}
\caption{Rat 11, jour 00, coupe 10}
\label{11_dark_00}
\end{figure}

\begin{figure}[!p]
\begin{center}
\begin{tabular}{|c|c|c|c|}
\hline
\subfloat[Anatomique]{\includegraphics[width=0.2\linewidth,height=2cm]{../../images_rapport/19-J00-Coreg01_Anat-masked-Cropped-slice9.jpg}}
&
\subfloat[ADC]{\includegraphics[width=0.2\linewidth,height=2cm]{../../images_rapport/19-J00-ADC-Cropped-slice9.jpg}}
&
\subfloat[BVf]{\includegraphics[width=0.2\linewidth,height=2cm]{../../images_rapport/19-J00-BVf-Cropped-slice9.jpg}}
&
\subfloat[CBF]{\includegraphics[width=0.2\linewidth,height=2cm]{../../images_rapport/19-J00-CBF-Cropped-slice9.jpg}}
\\
\hline
\subfloat[CMRO2]{\includegraphics[width=0.2\linewidth,height=2cm]{../../images_rapport/19-J00-CMRO2-Cropped-slice9.jpg}}
&
\subfloat[SO2map]{\includegraphics[width=0.2\linewidth,height=2cm]{../../images_rapport/19-J00-SO2map-Cropped-slice9.jpg}}
&
\subfloat[T1map]{\includegraphics[width=0.2\linewidth,height=2cm]{../../images_rapport/19-J00-T1map-Cropped-slice9.jpg}}
&
\subfloat[VSI]{\includegraphics[width=0.2\linewidth,height=2cm]{../../images_rapport/19-J00-VSI-Cropped-slice9.jpg}}
\\
\hline
\end{tabular}
\end{center}
\caption{Rat 19, jour 00, coupe 9}
\label{19_dark_00}
\end{figure}

\begin{figure}[!p]
\begin{center}
\begin{tabular}{|c|c|c|c|}
\hline
\subfloat[Anatomique]{\includegraphics[width=0.2\linewidth,height=2cm]{../../images_rapport/26-J00-Coreg01_Anat-masked-Cropped-slice8.jpg}}
&
\subfloat[ADC]{\includegraphics[width=0.2\linewidth,height=2cm]{../../images_rapport/26-J00-ADC-Cropped-slice8.jpg}}
&
\subfloat[BVf]{\includegraphics[width=0.2\linewidth,height=2cm]{../../images_rapport/26-J00-BVf-Cropped-slice8.jpg}}
&
\subfloat[CBF]{\includegraphics[width=0.2\linewidth,height=2cm]{../../images_rapport/26-J00-CBF-Cropped-slice8.jpg}}
\\
\hline
\subfloat[CMRO2]{\includegraphics[width=0.2\linewidth,height=2cm]{../../images_rapport/26-J00-CMRO2-Cropped-slice8.jpg}}
&
\subfloat[SO2map]{\includegraphics[width=0.2\linewidth,height=2cm]{../../images_rapport/26-J00-SO2map-Cropped-slice8.jpg}}
&
\subfloat[T1map]{\includegraphics[width=0.2\linewidth,height=2cm]{../../images_rapport/26-J00-T1map-Cropped-slice8.jpg}}
&
\subfloat[VSI]{\includegraphics[width=0.2\linewidth,height=2cm]{../../images_rapport/26-J00-VSI-Cropped-slice8.jpg}}
\\
\hline
\end{tabular}
\end{center}
\caption{Rat 26, jour 00, coupe 8}
\label{26_dark_00}
\end{figure}

\begin{figure}[!p]
\begin{center}
\begin{tabular}{|c|c|c|c|}
\hline
\subfloat[Anatomique]{\includegraphics[width=0.2\linewidth,height=2cm]{../../images_rapport/30-J00-Coreg01_Anat-masked-Cropped-slice11.jpg}}
&
\subfloat[ADC]{\includegraphics[width=0.2\linewidth,height=2cm]{../../images_rapport/30-J00-ADC-Cropped-slice11.jpg}}
&
\subfloat[BVf]{\includegraphics[width=0.2\linewidth,height=2cm]{../../images_rapport/30-J00-BVf-Cropped-slice11.jpg}}
&
\subfloat[CBF]{\includegraphics[width=0.2\linewidth,height=2cm]{../../images_rapport/30-J00-CBF-Cropped-slice11.jpg}}
\\
\hline
\subfloat[CMRO2]{\includegraphics[width=0.2\linewidth,height=2cm]{../../images_rapport/30-J00-CMRO2-Cropped-slice11.jpg}}
&
\subfloat[SO2map]{\includegraphics[width=0.2\linewidth,height=2cm]{../../images_rapport/30-J00-SO2map-Cropped-slice11.jpg}}
&
\subfloat[T1map]{\includegraphics[width=0.2\linewidth,height=2cm]{../../images_rapport/30-J00-T1map-Cropped-slice11.jpg}}
&
\subfloat[VSI]{\includegraphics[width=0.2\linewidth,height=2cm]{../../images_rapport/30-J00-VSI-Cropped-slice11.jpg}}
\\
\hline
\end{tabular}
\end{center}
\caption{Rat 30, jour 00, coupe 11}
\label{30_dark_00}
\end{figure}

%\FloatBarrier
\par
Avec des choix judicieux de contraste, on peut discerner, \g{a} l'oeil, des r\'egions l\'es\'ees sur les coupes segment\'ees de cerveau. %
Plus particuli\g{e}rement, et pour tous les rats, on discerne une aire sombre sur les images en CBF au jour 00, %
c'est-\g{a}-dire produites par l'examen r\'ealis\'e 30 minutes apr\g{e}s la reperfusion. %
Cette aire se situe sur l'h\'emisph\g{e}re visible \g{a} gauche, et on retrouve des r\'egions similaires, mais plus petites, %
sur des images obtenues dans les autres modalit\'es, notamment l'ADC : %
voir les figures \ref{11_dark_00}, \ref{19_dark_00}, \ref{26_dark_00} et \ref{30_dark_00}. %
L'isch\'emie focale transitoire se traduit pr\'ecis\'ement par une baisse du d\'ebit sanguin c\'er\'ebral %et la reperfusion ?
L'utilisation de cette r\'egion d'int\'er\^et est donc pertinente pour mod\'eliser l'ensemble des pixels initialement l\'es\'es.

\etoile
Une technique similaire \g{a} celle utilis\'ee pr\'ec\'edemment a \'et\'e utilis\'ee pour la segmentation des cerveaux -voir la figure \ref{cbf_seg_19}, %
afin de s\'electionner les r\'egions l\'es\'ees en CBF.

\par
Le tableau ci-dessous r\'ecapitule la taille, en nombre de pixels, des l\'esions ainsi d\'elimit\'ees sur les coupes suivables temporellement.

\begin{multicols}{2}
\begin{tabular}{|c|c|c|c|c|c|c|c|c|}
\hline
\small{Tranche}&6&7&8&9&10&11&12&13
\\
\hline
Rat 11&&&&&813&&&
\\
\hline
Rat 19&&&&742&654&&&
\\
\hline
Rat 26&772&791&558&&&&&
\\
\hline
Rat 30&&&&&&88&139&83
\\
\hline
\end{tabular}

\columnbreak
\begin{figure}[H]
\begin{center}
\includegraphics[width=0.45\linewidth, height=3cm]{../../images_rapport/19-J00-segCBF-slice9.jpg}
\end{center}
\caption{D\'elimitation d'un contour : rat 19, coupe 9. On utilise une image du CBF.}
\label{cbf_seg_19}
\end{figure}
\end{multicols}



\newpage
\FloatBarrier
\subsection{Traitement statistique avec R}% Utilisation de R : statistiques sur les cerveaux segment\'es, h\'emisph\g{e}res sains.

Les donn\'ees ainsi collect\'ees ont \'et\'e utilis\'ees pour observer l'\'evolution, au cours de l'exp\'erience, %
des diff\'erentes modalit\'es.

\par
Les diagrammes \ref{11_box_lch00}, \ref{19_box_lch00}, \ref{26_box_lch00} et \ref{30_box_lch00} permettent de lire facilement la r\'epartition des valeurs des diff\'erentes modalit\'es : ADC, BVf, CBF, CMRO2, SO2map et VSI. %
Ces valeurs sont compar\'ees avec celles mesur\'ees sur le cerveau entier, et sur l'h\'emisph\g{e}re contralat\'eral -non desservi par l'art\g{e}re op\'er\'ee- le jour 0.

\par
Notons au passage que ce dernier choix ne revient pas \g{a} comparer avec un tissu sain : bien que ne subissant pas de dysfonctionnement significatif, %
l'\'emisph\g{e}re \og{}sain\fg{} subit \'egalement, notamment le permier jour, des r\'epercussions de l'intervention ou de la reperfusion. %
L'\'evolution de l'\'etat de l'h\'emisph\g{e}re contralat\'eral est visible \g{a} la figure \ref{} et sera incorpor\'e au mod\g{e}le %\ref{}.

%\ligneinter

\begin{figure}[!p]
\begin{center}
\begin{tabular}{|c|c|c|}
\hline
\subfloat[ADC]{\includegraphics[width=0.3\linewidth,height=5cm]{../../images_rapport/11_suivi_box_volCBFdark00_ADC.pdf}}
&%
\subfloat[BVf]{\includegraphics[width=0.3\linewidth,height=5cm]{../../images_rapport/11_suivi_box_volCBFdark00_BVf.pdf}}
&%
\subfloat[CBF]{\includegraphics[width=0.3\linewidth,height=5cm]{../../images_rapport/11_suivi_box_volCBFdark00_CBF.pdf}}
\\
\hline
\subfloat[CMRO2]{\includegraphics[width=0.3\linewidth,height=5cm]{../../images_rapport/11_suivi_box_volCBFdark00_CMRO2.pdf}}
&%
\subfloat[SO2map]{\includegraphics[width=0.3\linewidth,height=5cm]{../../images_rapport/11_suivi_box_volCBFdark00_SO2map.pdf}}
&%
\subfloat[VSI]{\includegraphics[width=0.3\linewidth,height=5cm]{../../images_rapport/11_suivi_box_volCBFdark00_VSI.pdf}}
\\
\hline
\end{tabular}
\end{center}
\caption{Rat num\'ero 11 : \'evolution par jour d'examen}
\label{11_box_lch00}
\end{figure}

\begin{figure}[!p]
\begin{center}
\begin{tabular}{|c|c|c|}
\hline
\subfloat[ADC]{\includegraphics[width=0.3\linewidth,height=5cm]{../../images_rapport/19_suivi_box_volCBFdark00_ADC.pdf}}
&%
\subfloat[BVf]{\includegraphics[width=0.3\linewidth,height=5cm]{../../images_rapport/19_suivi_box_volCBFdark00_BVf.pdf}}
&%
\subfloat[CBF]{\includegraphics[width=0.3\linewidth,height=5cm]{../../images_rapport/19_suivi_box_volCBFdark00_CBF.pdf}}
\\
\hline
\subfloat[CMRO2]{\includegraphics[width=0.3\linewidth,height=5cm]{../../images_rapport/19_suivi_box_volCBFdark00_CMRO2.pdf}}
&%
\subfloat[SO2map]{\includegraphics[width=0.3\linewidth,height=5cm]{../../images_rapport/19_suivi_box_volCBFdark00_SO2map.pdf}}
&%
\subfloat[VSI]{\includegraphics[width=0.3\linewidth,height=5cm]{../../images_rapport/19_suivi_box_volCBFdark00_VSI.pdf}}
\\
\hline
\end{tabular}
\end{center}
\caption{Rat num\'ero 19 : \'evolution par jour d'examen}
\label{19_box_lch00}
\end{figure}

\begin{figure}[!p]
\begin{center}
\begin{tabular}{|c|c|c|}
\hline
\subfloat[ADC]{\includegraphics[width=0.3\linewidth,height=5cm]{../../images_rapport/26_suivi_box_volCBFdark00_ADC.pdf}}
&%
\subfloat[BVf]{\includegraphics[width=0.3\linewidth,height=5cm]{../../images_rapport/26_suivi_box_volCBFdark00_BVf.pdf}}
&%
\subfloat[CBF]{\includegraphics[width=0.3\linewidth,height=5cm]{../../images_rapport/26_suivi_box_volCBFdark00_CBF.pdf}}
\\
\hline
\subfloat[CMRO2]{\includegraphics[width=0.3\linewidth,height=5cm]{../../images_rapport/26_suivi_box_volCBFdark00_CMRO2.pdf}}
&%
\subfloat[SO2map]{\includegraphics[width=0.3\linewidth,height=5cm]{../../images_rapport/26_suivi_box_volCBFdark00_SO2map.pdf}}
&%
\subfloat[VSI]{\includegraphics[width=0.3\linewidth,height=5cm]{../../images_rapport/26_suivi_box_volCBFdark00_VSI.pdf}}
\\
\hline
\end{tabular}
\end{center}
\caption{Rat num\'ero 26 : \'evolution par jour d'examen}
\label{26_box_lch00}
\end{figure}

\begin{figure}[!p]
\begin{center}
\begin{tabular}{|c|c|c|}
\hline
\subfloat[ADC]{\includegraphics[width=0.3\linewidth,height=5cm]{../../images_rapport/30_suivi_box_volCBFdark00_ADC.pdf}}
&%
\subfloat[BVf]{\includegraphics[width=0.3\linewidth,height=5cm]{../../images_rapport/30_suivi_box_volCBFdark00_BVf.pdf}}
&%
\subfloat[CBF]{\includegraphics[width=0.3\linewidth,height=5cm]{../../images_rapport/30_suivi_box_volCBFdark00_CBF.pdf}}
\\
\hline
\subfloat[CMRO2]{\includegraphics[width=0.3\linewidth,height=5cm]{../../images_rapport/30_suivi_box_volCBFdark00_CMRO2.pdf}}
&%
\subfloat[SO2map]{\includegraphics[width=0.3\linewidth,height=5cm]{../../images_rapport/30_suivi_box_volCBFdark00_SO2map.pdf}}
&%
\subfloat[VSI]{\includegraphics[width=0.3\linewidth,height=5cm]{../../images_rapport/30_suivi_box_volCBFdark00_VSI.pdf}}
\\
\hline
\end{tabular}
\end{center}
\caption{Rat num\'ero 30 : \'evolution par jour d'examen}
\label{30_box_lch00}
\end{figure}

%\FloatBarrier
Les diagrammes \ref{11_box_lch00} \g{a} \ref{30_box_lch00} montrent syst\'ematiquement une baisse des valeurs d'ADC 30 minutes apr\g{e}s la reperfusion, %
ce qui correspond \g{a} un \'etat cytotoxique -oed\g{e}me- comme nous l'avons vu dans la section pr\'ec\'edente.

\par
Ces diagrammes indiquent aussi que les valeurs de BVf et de VSI sont, sauf pour le rat 11, corr\'el\'ees : %
les tendances d'\'evolution en position ou en dispersion sont les m\^emes ; %
rappelons qu'il s'agit l\g{a} respectivement de la proportion, en volume, de sang dans l'enc\'ephale et du diam\g{e}tre des vaisseaux sanguins.

\etoile
La distribution des valeurs \ref{26_box_lch00}, pour toutes les modalit\'es, est peu variable chez le rat num\'ero 26, et presque plus \g{a} partir du jour 8.

\par
Les diagrammes de la figure \ref{30_box_lch00}, qui correspondent aux seuls jours 0, 8 et 15 de l'exp\'erience, %
indiquent une croissance globale de toutes les modalit\'es pour le rat num\'ero 30, sans que l'on puisse voir de lien plus particulier entre deux variables ; %
la CMRO2, seule, se stabilise d\g{e}s le jour 8 apr\g{e}s avoir pris des valeurs initiales basses.

\par
Pour l'\'elaboration, et la validation du mod\g{e}le qui va suivre, j'ai donc choisi d'\'ecarter les rats num\'eros 26 et 30. %
Compte tenu de la position : ant\'erieure \g{a} $8$ pour le rat 26, post\'erieure \g{a} $11$ pour le rat 30, %
il est possible que les donn\'ees dont on dispose corresponde \g{a} une r\'egion p\'eriph\'erique de la l\'esion.

\etoile
Les rats 11 et 19 -figures \ref{11_box_lch00} et \ref{19_box_lch00}, pr\'esentent quelques similitudes : %
outre le comportement de la SO2 d\'ej\g{a}mentionn\'e, on remarque une croissance et une grande dispersion des valeurs d'ADC de la l\'esion avec le temps.

\par
On constate aussi une chute des valeurs du CBF :
\begin{itemize}
\item le jour 8, chez le rat 11, elle accompagne une hausse de la BVf ;
\item le jour 15, chez le rat 19, elle co\"incide avec des valeurs basses et peu dispers\'ees de la CMRO2.
\end{itemize}

\etoile
Enfin, on remarque des valeurs de VSI beaucoup plus \'elev\'ees chez le rat 11 que chez tous les autres rats, %
ceci est valable pour la r\'egion l\'es\'ee et pour l'h\'emisph\g{e}re contralat\'eral.

\par
En raison de cette diff\'erence inn\'ee entre le rat 11 et les autres rats, %
j'ai choisi d'utiliser les donn\'ees disponibles sur le rat num\'ero 19 comme r\'ef\'erence pour mon mod\g{e}le pr\'edictif.

\FloatBarrier
\subsection{Le paquet R Mclust : une id\'ee de Nicolas Glade (TIMC)}% On peut utiliser sweaver

% Genèse de l'idée.
%Nicolas Glade \'etait parti du constat suivant : pour certaines coupes examin\'ees, %
%les valeurs calcul\'ees en ADC pour le jour 0 peuvent \^etre regroup\'ees suivant un m\'elange de lois gaussiennes, %


%\par
%Pour cela, il avait utilis\'e  

A partir d'une base de donn\'ees g\'en\'er\'ee sur la segmentation des cerveaux de rats, Nicolas Glade a utilis\'e la fonction \'eponyme du paquet R Mclust %
pour classifier les valeurs prises par l'ADC sur une coupe de cerveau, au premier jour d'examen.

\par
En effet, la librairie Mclust permet de classifier un ensemble de donn\'ees, que l'on suppose \^etre r\'eparties selon un m\'elange gaussien, %
en utilisant une m\'ethode de maximum de vraisemblance.

\par
La figure \ref{exem_ADC_19} illustre, dans le cas tridimensionnel o\g{u} l'on effectue la classification sur plusieirs coupes, juxtapos\'ees, simultan\'ement, %
le premier r\'esultat obtenu par Nicolas Glade.

\begin{figure}[H]
\includegraphics[width=0.8\linewidth,height=13cm]{../../images_rapport/19-J00-ADC-cerveau_clhist_clust.pdf}
\caption{Classification des valeurs d'ADC suivant un mod\g{e}le de m\'elange gaussien, avec quatre composantes.
%\\
%
}
\label{exem_ADC_19}
\end{figure}

\par
Les deux premiers clusters, en bleu et en rouge, de la classification correspondent aux deux pics, bien visibles, sur l'histogramme de la figure \ref{exem_ADC_19}. %
Les distributions d'ADC peuvent \g{e}tre r\'esum\'ees par les diagrammes en bo\^ite de la figure \ref{19_box_lch00}, %
et qui concernent respectivement la r\'egion l\'es\'ee et l'h\'emisph\g{e}re contralat\'eral.

\par
L'emplacement et la forme du cluster en bleu \'evoque les zones sombres visibles en ADC, BVf ou CBF sur les images de la figure \ref{19_box_lch00} %
et dont la derni\g{e}re a servi \g{a} d\'efinir la l\'esion -voir \ref{cbf_seg_19}.

\par
Le troisi\g{e}me cluster du m\'elange correspond, selon Nicolas Glade, \g{a} un m\'elange de diff\'erentes structures ; %
en effet, l'existence de deux composantes connexes sur le graphique laisse penser que ce cluster pourrait \^etre encore d\'ecompos\'e. %
Peut-\^etre touche-t-on d\g{e}s lors aux limites de la d\'ecomposition en densit\'es gaussiennes, ou que le nombre de clusters pourrait simplement \^etre plus important, %
au d\'etriment d'une bonne lisibilit\'e de la figure.

\par
Le dernier cluster est principalement compos\'e des pixels avec la diffusivit\'e la plus forte. %
La signification physiologique de cette propri\'et\'e n'est pas un objectif de ce stage.

\begin{figure}[!p]
\begin{center}
\begin{tabular}{|c|c|}
\hline
\subfloat[BVf : d\'elimitation approximative, parasites.]{%
\includegraphics[height=7cm,width=0.4\linewidth]{../../images_rapport/19-J00-BVf-cerveau_clust.pdf}%bon
}
&
\subfloat[T1map : structures visibles, %
mais sans lien avec la l\'esion.]{%
\includegraphics[height=7cm,width=0.4\linewidth]{../../images_rapport/19-J00-T1map-cerveau_clust.pdf}%
}
\\
\hline
\subfloat[CBF : pas mieux que l'oeil.. %
Clusters vert et rouge difficiles \g{a} interpr\'eter.]{%
\includegraphics[height=7cm,width=0.4\linewidth]{../../images_rapport/19-J00-CBF-cerveau_clust.pdf}%bon
}
&
\subfloat[CMRO2 : mauvaise qualit\'e, taches bleues que l'h\'emisph\g{e}re contralat\'eral.]{%
\includegraphics[height=7cm,width=0.4\linewidth]{../../images_rapport/19-J00-CMRO2-cerveau_clust.pdf}%
}
\\
\hline
\subfloat[ADC : r\'esultat encourageant]{%
\includegraphics[height=7cm,width=0.4\linewidth]{../../images_rapport/19-J22-ADC-cerveau_clust.pdf}%bon
}
&
\subfloat[CBF : pas exploitable pour distinguer la r\'egion l\'es\'ee, sur le cerveau entier.]{%
\includegraphics[height=7cm,width=0.4\linewidth]{../../images_rapport/19-J22-CBF-cerveau_clust.pdf}%
}
\\
\hline
\end{tabular}
\end{center}
\caption{Six exemples de classification par m\'elange gaussien.%
\\%
A gauche : pour, ou presque.%
\\%
A droite : contre.%
}
\label{19_pour_contre}
\end{figure}

\etoile
\ref{19_CBF-BVf_3d00}

\subsubsection{Evaluation de la m\'ethode de classification}

%% Pour ou contre : listes. %%

% Jours autres que le premier : travail difficile pour l'oeil.
% De bonnes choses en ADC pour la deuxième moitié du mois.
% Mieux que l'oeil : distributions gaussiennes, composantes d'une décomposition putôt que simplement une région avec des valeurs plus faibles.


%\begin{figure}[!p]
%\end{figure}


% On peut être tenté d'utiliser la classification avec mclust pour suivre l'évolution temporelle de l'étendue de la lésion.
% Mieux : la classification ouvrirait une discussion sur la caractérisation de la lésion : quelle modalité est la plus pertinente ?

Certains r\'esultats obtenus laissent penser que la classification des pixels avec Mclust %
pourrait permettre d'obtenir de meilleurs r\'esultats que ceux de la sous-section pr\'ec\'edente, %
o\g{u} la partie l\'es\'ee \'etait caract\'eris\'ee par la segmentation effectu\'ee manuellement sur ImageJ comme sur la figure \ref{cbf_seg_19}.

\par
L'image en bas \g{a} gauche du tableau, figure \ref{19_pour_contre}, est en effet favorable \g{a} une telle approche, %
puisque les r\'egions l\'es\'ees sont tr\g{e}s difficiles \g{a} segmenter sur les images, en niveaux de gris, qui concernent des jours autres que 00.

\par
On peut opposer plusiers arguments concernant l'avantage, ou non, %
d'une d\'elimitation syst\'ematique de la r\'egion l\'es\'ee sur une image obtenue \g{a} la suite des diff\'rents examens.
\begin{enumerate}[label=\textbf{(Pour\arabic*)}]
\item\label{19_suivi_temp} Possibilit\'e d'une utilisation sur les jours autres que 00, notamment sur la deuxi\g{e}me moiti\'e du mois que durent les exp\'erimentations ;
\item\label{phy_gau} Eventuellement, signification physiologique des composantes du m\'elange gaussien, %
tandis que l'oeil lit seulement des valeurs particuli\g{e}rement basses de la variable \'etudi\'ee.
\end{enumerate}

Le point \ref{suivi_temp} laisse esp\'erer, avec un choix judicieux de variable -CBF, ADC ?- %
que l'utilisation de Mclust pourrait permettre de suivre l'\'evolution spatiale de la r\'egion l\'es\'ee. %
Toutefois, la situation que l'on peut lire en bas \g{a} gauche du tableau \ref{19_pour_contre} reste exceptionnelle, %
de plus, elle ne donne pas une caract\'erisation de la partie l\'es\'ee significativement diff\'erente de celle effectu\'ee au d\'ebut de cette section.

\etoile
Voici des arguments qui s'opposent \g{a} l'utilisation de Mclust pour caract\'eriser la r\'egion l\'es\'ee, \g{a} partir de coupes enti\g{e}res de cerveau :

\begin{enumerate}[label=\textbf{(Contre\arabic*)}]
\item Beaucoup d'images sont de mauvaise qualit\'e : voir le CBF ou la CMRO2 sur la figure \ref{19_pour_contre} ;
%
\item Les composantes d'une classification peuvent correspondre \g{a} des structures anatomiques comme nous l'avons vu dasn le premier graphique de cette sous-section, %
mais celles-ci ne correspondent pas n\'ecessairement \g{a} la partie l\'es\'ee : %
voir \g{a} ce titre la classification des valeurs mesur\'ees en T1, pour le rat num\'ero 19 au jour 00. %
Ceci va \g{a} l'encontre de l'argument \ref{phy_gau}, comme \'etant en faveur de l'utiliation de Mclust.
%
\item La classification automatique, non supervis\'ee, regroupe des pixels correspondant peu ou prou \g{a} la partie isch\'emi\'ee et des pixels de l'h\'emisph\g{e}re contralat\'eral. %
Ici encore, le jugement humain est plus pertient.
\end{enumerate}


% Problème des parasites pour l'extraction des données.
% Beaucoup d'images sont de mauvaise qualité.
% Souvent, les clusters sont dépourvus d'une signification qui serait comparable à celle dégagée dans le cas précédent : ADC au jour 0.
% La classification met peut-être au jour certaines structures physiologiques ! VSI, vascularisation ? ... Mais cela ne permet pas, la plupart du temps de caractériser la lésion cf...


%\begin{figure}[!p]
%\end{figure}

%\begin{figure}[!p]
%\end{figure}

%\begin{figure}[!p]
%\end{figure}

\ligneinter
En conclusion, apr\g{e}s avoir consacr\'e un certain temps \g{a} la classification des niveaux de gris des images dont je disposais, %
correspondant aux valeurs des diff\'erentes variables scalaires d\'efinies sur des structures tridimensionnelles, %
j'ai d\'ecid\'e de ne pas utiliser le paquet R Mclust pour la caract\'erisation et le suivi des l\'esion au sein des cerveaux des rats utilis\'es pendant exp\'erience, %
faute d'un nombre suffisant, et parfois de la pertinence, des r\'esultats obtenus, %
comparativement \g{a} ceux de la sous-section pr\'ec\'edente qui \'etaient issus d'une segmentation \g{a} l'oeil.

\etoile
Toutefois, certains r\'esultats obtenus avec Mclust quand on invoque la fonction au sein de la r\'egion l\'es\'ee %
m'ont encourag\'e \g{a} utiliser celle-ci pour mod\'eliser la l\'esion du rat num\'ero 19.

\par
C'est l'objet de la sous-section suivante.

\FloatBarrier
\subsubsection{Classification des pixels de la l\'esion}

% Point intéressant : la fonction mclust prend en argument, outre la base de données à classifier,
% un vecteur constitué d'entiers positifs qui paramètre une famille de modèles que mclust se propose de calculer.
% Ainsi, la fonction calcule \emph{plusieurs} mélanges gaussiens, un pour chacun des entiers du vecteur pris en argument,
% puis sélectionne le meilleur modèle avec le critère BIC.
% Il est d'ailleurs possibles de demander à R d'inclure, dans une représentation graphique,
% les différentes valeurs de BIC associées à chaque possibilité de nombre de clusters.
% On peut ainsi visualiser la pertinence du choix d'un nombre de clusters particulier pour la classsification.
% De telles représentations graphiques apparaissent dans les figures \ref{19_choix_clust_les}...


\begin{figure}[!p]
\begin{center}
\begin{tabular}{|c|c|}
\hline
%\subfloat[BVf : d\'elimitation approximative, parasites.]
&
%\subfloat[T1map : structures visibles, %
%mais sans lien avec la l\'esion.]
\\
\hline
%\subfloat[CBF : pas mieux que l'oeil.. %
%Clusters vert et rouge difficiles \g{a} interpr\'eter.]
&
%\subfloat[CMRO2 : mauvaise qualit\'e, taches bleues que l'h\'emisph\g{e}re contralat\'eral.]{%
%\includegraphics[height=7cm,width=0.4\linewidth]{../../images_rapport/19-J03-BVf_clust1-3_lesion.pdf}
%}
\\
\hline
%\subfloat[ADC : r\'esultat encourageant]
&
%\subfloat[CBF : pas exploitable pour distinguer la r\'egion l\'es\'ee, sur le cerveau entier.]
\\
\hline
\end{tabular}
\end{center}
\caption{Classification sur des images en CBF et BVf.%
\\%
Le choix de deux clusters para\^it le plus pertient : vers une mod\'elisation \g{a} deux r\'egimes ?}
\label{19_choix_clust_les}
\end{figure}





\begin{figure}[!p]
\begin{center}
\begin{tabular}{|c|c|}
\hline
%\subfloat[BVf : d\'elimitation approximative, parasites.]
&
%\subfloat[T1map : structures visibles, %
%mais sans lien avec la l\'esion.]
\\
\hline
%\subfloat[CBF : pas mieux que l'oeil.. %
%Clusters vert et rouge difficiles \g{a} interpr\'eter.]
&
%\subfloat[CMRO2 : mauvaise qualit\'e, taches bleues que l'h\'emisph\g{e}re contralat\'eral.]
\\
\hline
%\subfloat[ADC : r\'esultat encourageant]
&
%\subfloat[CBF : pas exploitable pour distinguer la r\'egion l\'es\'ee, sur le cerveau entier.]
\\
\hline
\end{tabular}
\end{center}
\caption{Images favorables \g{a} une mod\'elisation \g{a} deux r\'egimes pour l'\'evolution de la partie l\'es\'ee}
\label{19_reg_clust_les}
\end{figure}

%%% Broder %%%

\FloatBarrier
\subsection{}%Vers un mod\g{e}le pour le rat num\'ero 19}

Les r\'esultats obtenus en fin de section pr\'ec\'edente m'ont encourag\'e \g{a} suivre l'\'evolution des six modalit\'es %
ADC, BVf, CBF, CMRO2, SO2 et VSI sur des r\'egions d'int\'er\^et d\'efinies par la clusterisation de CBF au jour 8.%, SMRO2, SO2 et BVf aux jours 8 ou 15.

%
%
%

\ligneinter
Pour la suite de ce travail, j'ai choisi d\'etudier les valeurs de six modalit\'es sur les deux sous-r\'egions de la l\'esion, d\'elimit\'ee en CBF au jour 0, %
elles m\^emes d\'efinies par une valeur seuil de CBF pour le jour 8 : 100 ml.min${}^{-1}$.100g${}^{-1}$. %
On remarque sur la figure \ref{19_reg_clust_les} que ces deux r\'egions pourraient aussi bien d\'efinies par le d\'epassement, ou non, de la valeur seuil de 1000 %...
de CMRO2.

%\par
%%% Importance des différentes modalités : CBF, SO2, CMRO2 etc. %%%



%\ref{}











\begin{figure}[!p]
\begin{center}
\begin{tabular}{|c|c|}
\hline
%\subfloat[BVf : d\'elimitation approximative, parasites.]
&
%\subfloat[T1map : structures visibles, %
%mais sans lien avec la l\'esion.]
\\
\hline
%\subfloat[CBF : pas mieux que l'oeil.. %
%Clusters vert et rouge difficiles \g{a} interpr\'eter.]
&
%\subfloat[CMRO2 : mauvaise qualit\'e, taches bleues que l'h\'emisph\g{e}re contralat\'eral.]
\\
\hline
%\subfloat[ADC : r\'esultat encourageant]
&
%\subfloat[CBF : pas exploitable pour distinguer la r\'egion l\'es\'ee, sur le cerveau entier.]
\\
\hline
\end{tabular}
\end{center}
\caption{Suivi pour le rat 19 : les six modalit\'es. %
La largeur des bo\^ites est proportionnelle \g{a} la racine carr\'ee de l'effectif repr\'esent\'e, %
ce qui prouve la pertinence de la d\'ecomposition.
\\%
L'\'etat de l'h\'emisph\g{e}re contralat\'eral est aussi l'objet d'un suivi.%
}
\label{19_suivi_clust_les}
\end{figure}

\FloatBarrier
\subsection{Les hypoth\g{e}ses et le formalisme du mod\g{e}le}%Modèle proprement dit, évetuellement dans la section suivante.

L'objectif atteint au cours de ce stage a \'et\'e d'\'elaborer un mod\g{e}le dynamique pour l'\'evolution, voxel par voxel, %
des donn\'ees issues d'un examen d'imagerie effectu\'e, sur un rat, dans les minutes qui suivent une isch\'emie focale transitoire.

\par
Ce mod\g{e}le, qui n'est pas explicitement li\'e \g{a} des m\'ecanismes qui se d\'eroulent \g{a} l'\'echelle cellulaire, %
doit reproduire les r\'esultats obtenus au cours des jours, et des premi\g{e}res semaines, qui suivent l'isch\'emie.

\par
Le mod\g{e}le que j'ai construit est uniquement inspir\'e des observations qui, comme on peut le voir dans les sous-sections pr\'ec\'edentes, %
concernent exclusivement le rat 19.

\par
En revanche, les hypoth\g{e}ses simplificatrices pos\'ees pour ce mod\g{e}le, la mani\g{e}re dont elles sont d\'egag\'ees %
et les grandes lignes de l'\'evolution des voxels des r\'egions, l\'es\'ees ou non, du cerveau sur les semaines qui suivent l'isch\'emie %
sont potentiellement adaptables aux autres cas.

\etoile

Les hypoth\g{e}ses d'un mod\g{e}le \og{} id\'eal\fg{} pour l'\'evolution locale de la r\'egion isch\'emi\'ee, et du reste du cerveau, %
sont r\'ef\'erenc\'ees ci-dessous.

\par
Comme nous allons le voir, les hypoth\g{e}ses relativement fortes de ce mod\g{e}le n'ont malheureusement pas permis, %
de pr\'edire la dynamique de l'\'evolution du cerveau \'etudi\'e \g{a} l'aide des seules donn\'ees IRM au jour 0.
%-on observe la coupe num\'ero 9 issue du premier examen du rat num\'ero 19, %
%c'est donc un jeu d'hypoth\g{e}ses plus faible qui sera retenu pour les simulations.


\begin{modmerate}{Mod\g{e}le 1}{\textbf{(H\arabic*)}}
\item\label{nprog} L'\'evolution de l'\'etat d'un voxel n'est pas affect\'ee par l'\'etat des voxels voisins.
\item\label{var_comp} Les modalit\'es ADC, BVf, etc d\'efinissent, seules, et localement l'\'etat du syst\g{e}me.
\item\label{auto} L'\'evolution du syst\g{e}me est autonome.
%\item\label{bvf...}
\end{modmerate}

L'hypoth\g{e}se \ref{nprog} distingue d'entr\'ee de jeu le mod\g{e}le de celui consid\'er\'e dans \cite{Duval_JCBFM_02} : %
dans ce dernier, le coefficient de diffusion ADC diffuse lui-m\^eme \g{a} travers les pixels, et respecte l'\'equation disscr\g{e}te :
\[\text{ADC}_{i,j}(t+1)=\frac{1}{9}\sum\limits_{\|k,l-i,j\|_{\infty}=1}\text{ADC}_{k,l}(t)\]

Le membre de droite correspond \g{a} la moyenne ces valeurs d'ADC \g{a} un instant discret $t$ : %
la valeur d'ADC du pixel consid\'er\'e et celles de ses $8$ voisins sont prises en compte, %
le membre de gauche est la valeur d'ADC au temps $t+1$.

\par
L'hypoth\g{e}se \label{var_comp} doit \^etre affaibie pour deux raisons :
\begin{itemize}
\item J'ai d\'ecid\'e de prendre en compte des perturbations exog\g{e}nes des valeurs prises par les diff\'erentes variables, %
en particulier dans les r\'egimes d\'equilibre r\'egis par des \'equations stochastiques.
\item Sur les figures \ref{19_suivi_clust_les}, on note qu'il n'est pas possible de discerner les distributions de valeurs prises par les six variables choisies, %
sur les deux composantes d\'efinies par les valeurs de CBF au jour 8.
\end{itemize}

\par
Le dernier point qui pr\'ec\g{e}de donnera lieu \g{a} un affaiblissement de l'hypoth\g{e}se d'autonomie \ref{auto} : %
chaque pixel se trouvant initialement dans un \'etat correspondant \g{a} l'isch\'emie subira un changement d'\'etat, %
se produisant \g{a} un temps al\'eatoire compris entre les jours 3 et 8 de la simulation.

\par
Une telle \'echelle de temps, suivant laquelle les observations effectu\'ees sur le rat num\'ero 19 %
sugg\g{e}rent l'existence de deux r\'egipes diff\'erents pour les pixels repr\'esentant la l\'esion, %
co\"incide avec celle qui r\'egit les m\'ecanismes de formation d'oed\g{e}mes vasog\'eniques, %
et l'infiltration de cellules immunitaires parfois d\'el\'et\g{e}res dans le parenchyme c\'er\'ebral. %
L'approche qui consiste \g{a} introduire dans le mod\g{e}le de ce facteur temporel me para\^it conc pertinente, %
m\^eme si le lien avec les diff\'erentes \'etapes de la cascade isch\'emique n'est pas \'elucid\'e \g{a} ce stade.

\ligneinter
Je distingue ainsi sept \'etats : cinq pour les pixels isch\'emi\'es, deux pour les autres, possibles pour l'\'evolution.

\par
Chacun de ces \'etats est caract\'eris\'e par un ensemble de valeurs possibles pour les valeurs des six variables, %
et un syst\g{e}me d'\'equations d'\'evolution.

\par
Celles-ci, \'ecrites comme des \'equations diff\'erentielles ordinaires mettant en jeu les diff\'erentes variables avec les unit\'es vues \g{a} la section pr\'ec\'edente, %
seront impl\'ement\'ees en rempla\c cant les d\'eriv\'ees par des taux d'accroissement, l'unit\'e de temps adopt\'ee pour les \'equations \'etant le jour ; %
cette dur\'ee sera \'egalement l'unit\'e de temps \'el\'ementaire du mod\g{e}le discret, impl\'ement\'e dans la suite de ce travail.

\etoile
On note que l'hypoth\g{e}se de non progressivit\'e, que je conserve dans toute la sute de ce travail, %
revient \g{a} utiliser des automates pour le mod\g{e}le dynamique, et non des automates cellulaires. %
%Une critique de l'hypoth\g{e}se \ref{nprog} peut \^etre faite \g{a} la lumi\g{e}re de faits physiologiques reconnus.


\subsubsection{Automates pour les pixels, d'apr\g{e}s le rat num\'ero 19}


%%% Automates \g{a} dessiner : avec xymatrix %%%



\subsubsection{Formalisme des diff\'erents r\'egimes}

\begin{description}
\item[R\'egime cytotoxique :] Il correspond aux premi\g{e}res heures d'isch\'emie. %
On suppose qu'ici, la saturation des r\'egulations successives du CBF, puis de la CMRO2 a d\'ej\g{a} eu lieu, ce qui justifie des valeurs basses pour ces deux variables.

\par
Pour le besoin des simulations, un temps $T$, al\'eatoire, est d\'efini sur l'ensemble des pixels qui entrent dans ce premier \'etat isch\'emi\'e.

\par
Les valeurs anormalement basses d'ADC, qui ob\'eiront \g{a} l'\'equation de rappel % \ref{} ,
signalent des n\'ecroses qui s'expliquent avec les \'etapes de la cascade isch\'emique vues \g{a} la section pr\'ec\'edente.

\par
L'influence des valeurs de CBF et de CMRO2 sur l'\'evolution de l'ADC n'est ainsi pas visible dans l'\'equation d'\'evolution de cette variable, %
mais elles d\'eterminent le choix du r\'egime, donc la validit\'e desdites \'equations.

%%% Equations : début. %%%
\begin{equation}
\der{ADC}=0,25\left(ADC_{contra}-ADC\right) \Ta
\label{adc_is1}
\end{equation}
L'ADC a tendance \g{a} se stabiliser autour des valeurs calcul\'ees sur l'h\'emisph\g{e}re contralat\'eral, en dehors du jour 0. %
L'utilisation de cette distribution spatiale, asymptotique dans le temps, est encore inspir\'ee de \cite{Duval_JCBFM_02}.

\begin{equation}
\der{CBF}=0,5\left(CBF_{trans}-CBF\right) \Ta
\label{cbf_is1}
\end{equation}
Cette \'equation d\'ecrit une croissance et une dispersion progressive du CBF. %
Celle-ci est lisible, indistinctement du r\'egime de l\'esion adopt\'e \g{a} partir du jour 8, %
sur le graphique de la figure \ref{19_suivi_clust_les}

La saturation tissulaire en dioxyg\g{e}ne suit une \'equation, valable pour tous les r\'egimes en dehors de la \og{} l\'esion 1\fg{} :
\begin{equation}
\der{SO2}=2,5(SO2_{asy}-SO2) \Ta
\label{so2_s_l2}
\end{equation}
La loi choisie dans ce mod\g{e}le pour la distribution asymptotique de SO2 est la loi normale $\mathcal{N}(75,35)$.

\par
La consommation de dioxyg\g{e}ne suit une \'equation analogue aux deux premi\g{e}res :
\begin{equation}
\der{CMRO2}=\frac{1}{4}(CMRO2_{trans}-CMRO2) \Ta
\end{equation}
La distribution choisie pour la CMRO2 suit une loi $\mathcal{N}(6,7)$. %
Dans cette approche ph\'enom\'enologique, j'ai d\'ecid\'e de ne pas relier les \'equations r\'egissant les \'evolutions de CBF et de CMRO2, %
malgr\'e les similitudes que l'on constate entre celles-ci, sur les donn\'ees r\'esum\'ees dans la figure \ref{19_suivi_clust_les}. %
%L'impl\'ementation de ce mod\g{e}le serait de toute mani\g{e}re peu influenc\'ee 

De m\^eme que pour la CMRO2, les donn\'ees manquent pour le VSI et la BVf au jour 3. %
D'une mani\g{e}re g\'en\'erale, les \'equations r\'egissant l'\'evolution de ces deux variables seront reli\'ees dans tous les r\'egimes du mod\g{e}le, %
qu'ils concernent, ou non, une partie l\'es\'ee ou non.

\par
Afin de retracer l\'evolution de ces deux variables aux premiers jours de l'exp\'erience, %
on remarque une forte similitude entre celle-ci, et l'\'evolution sur les cinq jours d'examen du CBF.

\par
J'ai donc choisi des \'equations, pour la BVf et le VSI, li\'ees \g{a} \ref{cbf_is1}.
\begin{equation}
\der{VSI}=0,15\times\der{CBF}
\label{vsi_is}
\end{equation}
\begin{equation}
\der{BVf}=\der{VSI}
\label{bvf_is}
\end{equation}
\emph{Pour simplifier les notations et dans toute la suite de ce travail, la d\'eriv\'ee temporelle $\der{V}$ sera not\'ee $\dot{V}$ pour la variable $V$.}
%%% Fin. %%%
\par
A l'issue de ce r\'egime, d\'etermin\'ee par le temps al\'eatoire $T$ qui caract\'erise la survenue possible d'un oed\g{e}me vasog\'enique, %
et l'infiltration programm\'ee de macrophages dans le parenchype c\'er\'ebral, l'\'etat d'un pixel est d\'etermin\'e par sa valeur de CBF :
\begin{description}
\item[$CBF > 100$ :] Premier r\'egime d'\'evolution de la partie l\'es\'ee, notament caract\'eris\'ee par une baisse du dit CBF ;
\item[$CBF < 100$ :] Deuxi\g{e}me r\'egime, qui aboutit \g{a} un \'etat stationnaire, distinct de l'\'etat d'\'equilibre, suppos\'ement sain, de l'h\'emisph\g{e}re contralat\'eral.
\end{description}
\item[R\'egime isch\'emique transitoire] Il doit \^etre vu comme un sous-r\'egime du pr\'ec\'edent : %
il est valable pour un pixel si, et seulement si celui-ci se trouve en r\'egime cytotoxique depuis un temps $T$ qui est d\'ecrit ci-apr\g{e}s.

\par
On ne peut donc pas parler de transition entre deux \'etats de l'automate m\^eme si certaines \'equations diff\g{e}rent entre les deux r\'egimes.
%%% Equations : début. %%%
L'equation ma\^itresse de cette transition concerne le CBF :
\begin{equation}
\dot{CBF} = \tau_{trans}\times\left(CBF_{mel}-CBF\right)
\label{cbf_l_tr}
\end{equation}

%La validit\'e de cette \'equation est d\'etermin\'ee par 
%
La densit\'e $CBF_{mel}$ doit \^etre compatible avec le m\'elange gaussien issu des mesures au jour 8.

\par
La figure \ref{19_suivi_clust_les} montre qu'il est toutefois difficile de pr\'edire la r\'egion \g{a} laquelle appartiendra un pixel, %
pour lequel les valeurs des six variables sont connues uniquement pour le jour 0 %
- c'est l'objectif d'une simulation informatique dans la pratique m\'edicale !

\par
Ainsi, on se trouve devant le dilemme suivant :
\begin{itemize}
\item Tricher sur la simulation et incorporer une distribution $CBF_{mel}$ issue des r\'esultats exp\'erimentaux ;
\item S\'electionner arbitrairement, ou al\'eatoirement $CBF_{mel}$ sans consid\'eration pour l'anatomie du cerveau mod\'elis\'e.
\end{itemize}

Cette deuxi\g{e}me approche est retenue pour le mod\g{e}le courant, on revient sur la premi\g{e}re, moins artificielle, %
dans la section suivante.

\par
Les grandeurs $\dot{CBF}$, $\dot{CMRO2}$, $\dot{VSI}$ et $\dot{BVf}$ sont proportionnelles dans ce r\'egime, %
les coefficients doivent \^etre ajust\'es pour s'accorder aux observations.

\par
La forme de l\'equation en SO2 ne change pas.%, seule la distribution asymptotique est modifi\'ee pour les pixels de la classe \og{} l\'esion 1\fg{}; %
%dans les \'equations du mod\g{e}le, on notera $SO2_{Asy}$ cette deuxi\g{e}me distribution.

\par
L'ADC suit, dans ce r\'egime pr\'e-jour 8, une nouveau type d'\'evolution : %
l'\'equation $\Td$ qui va suivre d\'ecrit l'augmentation rapide, et la dispersion des valeurs d'ADC.
\begin{equation}
\dot{ADC} = \tau_1\times ADC^{\alpha_1}
\label{adc_is_tr}
\end{equation}
$\alpha_1$ est un r\'eel strictement positif qui doit permettre de r\'egler la dispersion, plus ou moins rapide, des valeurs d'ADC.
%%% Fin. %%%
%

Apr\g{e}s le temps $T$, on utilise les hypoth\g{e}ses \ref{nprog}, \ref{auto} et \ref{var_compf} : %
l'\'etat d'un pixel, et donc le r\'egime suivi par ses valeurs d'ADC, de BVf, CBF, CMRO2, SO2 et VSI est d\'efini par une condition en CBF :
%
\item[$CBF > 100$ : l\'esion num\'ero 1 ] cet \'etat d'automate est d\'efini par les \'equations qui suivent.
%%% Equations : début. %%%
ADC suit un nouveau r\'egime, dit $\Td$ : c'est l'\'equation qui suit, qui d\'ecrit notamment la dispersion rapide des valsurs d'ADC.

CBF, SO2 et CMRO2 suivent le m\^eme r\'egime $\Ta$. Aucun lien de cause \g{a} effet n'est \g{a} privil\'egier, \g{a} la lumi\g{e}re des r\'esultats en \ref{19_suivi_clust_les}. %
Les \'equations d'\'evolution de ces trois variables sont les suivantes :
\begin{equation}
\dot{CBF}=5\times\left(CBF_{asy1}-CBF\right)
\end{equation}
\begin{equation}
\dot{CMRO2}=0,15\times\left(CMRO2_{asy1}-CMRO2\right)
\end{equation}
\begin{equation}
\dot{SO2}=1\times\left(SO2_{asy1}-SO2\right)
\end{equation}

Les deux variables VSI et BVf suivent un r\'egime commun, distinct de l'\'evolution des autres variables :
\begin{equation}
\dot{VSI}=0,2\times (VSI_{asy1}-VSI)
\end{equation}
et une derni\g{e}re \'equation, li\'ee \g{a} la pr\'ec\'edente :
\begin{equation}
\dot{BVf}=0,5\times\dot{VSI})
\end{equation}
La distribution statistique VSI est encore issue des observations -jour 15 ou 22.
%%% Fin. %%%
%
\item[$CBF > 100$ : l\'esion num\'ero 2 ] L'\'evolution du CBF influence significativement celle des autres variables, %
et la valeur de la SO2 est choisie pour caract\'eriser la transition vers le r\'egime asymptotique.
%%% Equations : début. %%%
\begin{equation}
\dot{CBF}=-15\times\left(CBF_{trans2}-CBF\right)
\end{equation}

La densit\'e $CBF_{trans2}$ est d\'etermin\'ee par les r\'esultats de \ref{19_suivi_clust_les}, %
et diff\g{e}re de l'\'etat sain par sa dispersion, principalement.

\par
L'\'evolution de la CMRO2 est ici reli\'ee \g{a} celle du CBF :
\begin{equation}
\dot{CMRO2}=0,15\times\dot{CBF}
\end{equation}

L'ADC suit une \'equation similaire \g{a} celle du r\'egime pr\'ec\'edent :
\begin{equation}
\dot{ADC} = \tau_1\times ADC^{\alpha_2} \Td
\label{adc_is_tr}
\end{equation}
Une autre valeur $\alpha_2$, plus faible que la premi\g{e}re, doit \^etre choisie en prenant en compte \ref{19_suivi_clust_les}.

VSI et BVf suivent, quand \g{a} elles un r\'egime $\Ta$, elles sont reli\'ees de la m\^eme mani\g{e}re que pour la l\'esion 1.
\begin{equation}
\dot{VSI}=0,3\times (VSI_{asy2}-VSI)
\end{equation}
et% une derni\g{e}re \'equation, li\'ee \g{a} la pr\'ec\'edente :
\begin{equation}
\dot{BVf}=2\times\dot{VSI})
\end{equation}
%%% Fin. %%%
%
\item[L\'esion 1 : r\'egime asymptotique] Les variables CBF, CMRO2, SO2, VSI et BVf \'evoluent toutes selon une \'equation $\Tb$. %
Les param\g{e}tres correspondants sont choisis au moment des simulations.

\par
L'ADC suit encore l'\'equation $\Td$ :
\begin{equation}
\dot{ADC}=\tau_1ADC^{\alpha_1}
\end{equation}
%
\item[L\'esion 2 : \'evolution finale. ] Elle est d\'ecr\'et\'ee par le retour \g{a} la normale de la SO2.
%%% Equations : début. %%%
\par
L'ADC suit l'\'equation $\Ta$ :
\begin{equation}
\dot{ADC}=\tau_1ADC^{\alpha_1}
\end{equation}

La SO2 d\'evie alors l\'eg\g{e}rement de la normale, et suit une \'equation $\Ta$ :
\begin{equation}
\dot{SO2}=1\times (SO2_{asy2}-SO2)
\end{equation}

$SO2_{asy2}$ est issue des mesures exp\'erimentales, qui sont r\'esum\'ees dans la figure \ref{19_suivi_clust_les}.
\par
Dans ce mod\g{e}le ph\'enom\'enologique, l'\'evolution de SO2 dicte celles de CBF et de la CMRO2 :
\begin{equation}
\dot{CBF}=-10\times\dot{SO2}
\end{equation}
\begin{equation}
\dot{CMRO2}=-1\times\dot{SO2}
\end{equation}
Enfin le VSI et la BVf suivent ensemble un r\'egime $\Tb$.
%%% Fin. %%%
%
\item[R\'egime sain perturb\'e] l'\'evolution des valeurs, calcul\'ees ou mesur\'ees, pour les six variables ADC \dots VSI %
ne se r\'esume par \g{a} des oscillations autour d'un \'equilibre. Deux explications sont possibles pour ces \'evolutions :
\begin{itemize}
\item L'existence de m\'ecanismes de r\'epercussion progressive des \'ev\g{e}nements subis par la partie isch\'emi\'ee, %
dans l'ensemble du cerveau. Un exemple de tel m\'ecanisme est bien connu : les ondes de d\'epolarisation, qui apparaissent dans les minutes qui suivent l'isch\'emie.
%
\item La r\'epercussion, sur le d\'ebit sanguin des r\'egions non l\'es\'ee, de la reperfusion de la partie isch\'emi\'ee.
\end{itemize}

\par
Dans le cadre de ce mod\g{e}le, je privil\'egie la seconde explication : par \ref{nprog}, %
les effets progressifs de changements d'\'etat des pixels se cantonnent aux premi\g{e}res minutes de l'isch\'emie, %
les perturbations ressenties localement en dehors de la zone l\'es\'ee sont alors de nature exog\g{e}ne.

\par
Dans un souci de simplicit\'e, je distingue, dans le mod\g{e}le du tissu sain, %
un retour \g{a} des valeurs normales, puis des fluctuations autour de l'\'equilibre, pour le CBF.
%%% Equations : début. %%%
\begin{equation}
\dot{CBF}=3\times (CBF_{sain}-CBF)
\end{equation}

Autres variables : r\'egime $\Tb$.
%%% Fin. %%%
\item[Etat sain permanent : ] il est annonc\'e par une valeur de CBF sup\'erieure ou \'egale \g{a} 100. %
Toutes les variables, ADC comprise, suivent le r\'egime bruit\'e $\Tb$.

\par
Dans le cas du tissu sain, le bruit est particuli\g{e}rment important pour la plupart des variables. %
La simplification d\'ecid\'ee ci-dessus est discutable, eu \'egard l'\'evolution de la CMRO2.
\end{description}

\ligneinter%\newpage ?
Les \'equations qui pr\'ec\g{e}dent sont r\'eparties en trois types :

\fbox{%
\begin{minipage}{\textwidth}
\begin{description}
\item[$\Tb$ :] c'est l'\'equation, stochastique : $\dot{V}=\mathcal{B}$ o\g{u} $\mathcal{B}$ est un bruit blanc, param\'etr\'e avec les observations, %
et qui d\'ecrit un r\'egime \g{a} l'\'equilibre.

\par
Elle n'\'etablit pas de lien entre les variables.
%
\item[$\Ta$ :] c'est l'\'equation diff\'erentielle du premier ordre $\dot{V}=\tau\times\left(V^{asy}-V\right)$, %
qui d\'ecrit le retour de la variable $V$ vers une distribution asymptotique $V^{asy}$.

\par
Ces \'equations peuvent \^etre li\'ees entre elles, comme pour le BVf et le VSI ; %
le coefficient $\tau$ est obtenu \g{a} l'aide des observations de la figure \ref{19_suivi_clust_les} %
Elles sont en principe transitoires, et pr\'ec\g{e}dent un retour au r\'egime $\mathbf{T0_{\mathbb{P}}}$.
%Distribution spatiale ? Voir encore \cite{Duval_JCBFM_02}
%
\item[$\Td$ :] c'est l'\'equation qui r\'egit l'\'evolution de l'ADC dans les r\'egions l\'es\'ees, apr\g{e}s la premi\g{e}re semaine.

\par
Elle traduit une h\'et\'erog\'en\'eit\'e des structures examin\'ees et une forte diffusivit\'e des mol\'ecules d'eau, %
soit possiblement les cons\'equences de n\'ecroses massives ou d'un oed\g{e}me vasog\'enique.

\par
Les param\g{e}tres $\alpha_1$ et $\alpha_2$ doivent servir \g{a} ajuster les \'equations du mod\g{e}le : je n'ai pas choisi de valeurs.
\end{description}%
\end{minipage}
}


%Cas continu : les seconds membres \mathcal{E} seraient remplac\'es par des bruits blancs.
%Ici il s'agit de processus gaussiens à temps discret.

\etoile
Il faut noter que les \'equations qui pr\'ec\g{e}dent, qui sont \'ecrites commme un syst\g{e}me d'un temps continu $t$ exprim\'e en jours, %
seront remplac\'ees par des \'equations discr\g{e}tes pour l'impl\'ementation :
\begin{itemize}
\item Les op\'erateurs $\dfrac{\text{d}}{\text{dt}}$ sont remplac\'es par l'op\'erateur discret $\Box_{t+1}-\Box_t$ %
qui porte sur la variable toutes choses \'etant par ailleurs \'egales ;
\item Le second membre $\mathcal{E}(m,\sigma)_t$, analogue au bruit blanc des processus \g{a} temps continu, %
est ici une suite de variables al\'eatoires gaussiennes, simul\'ees chaque jour sur l'ensemble des pixels, avec une esp\'erance $m$ et un \'ecart-type $\sigma$.
\end{itemize}

\ligneinter
Les diff\'erents r\'egimes mis en \'evidence dans cette sous-section, et les \'equations qui les d\'efinissent satisfont %
un jeu d'hypoth\g{e}ses plus faible que celui du mod\g{e}le 1 qui pr\'ec\g{e}de. En voici un r\'esum\'e :

\begin{modmerate}{Mod\g{e}le 2}{\textbf{(H${}^{\ast}$\arabic*)}}
\item Identique \g{a} \ref{nprog}.
\item\label{var_compf} Le mod\g{e}le d\'efini par les \'equations qui pr\'ec\g{e}dent, %
qui contiennent en plus des six variables, \emph{une composante stochastique}, est bien pos\'e.
\item\label{autof} Les \'equations du mod\g{e}le sont autonomes, sauf celles du r\'egime suivi par les pixels situ\'es sur la l\'esion initiale %
qui incorporent une dur\'ee fix\'ee initialement. %
Dans le cadre des simpulations cette dur\'ee est d\'efinie al\'eatoirement sur les pixels qui sont entr\'es dans l'\'etat isch\'emi\'e.
\end{modmerate}

\begin{comment}
\newpage
\subsubsection{Constitution d'une base de donn\'ees pour les conditions initiales}

Les donn\'ees collect\'ees, pour le jour 0 comme pour les autres, sur le rat num\'ero 19 %
et pour les six modalit\'es choisies ADC, BVf, CBF, CMRO2, SO2map et VSI peuvent facilement \^etre combin\'ees, %
pour former la condition initiale des simulations.

\par
Toutefois, cette approche na\"ive donne un r\'esultat tr\g{e}s insuffisant \g{a} la r\'ealisation de simulations du mod\g{e}le num\'ero 2 : %
voir \g{a} ce tire la figure \ref{sim_ini}, partie gauche.

\par
En effet, la base de donn\'ees multiparam\'etrique est d\'efinie sur l'intersection des ensembles de pixels disponibles pour les six modalit\'es, %
d'o\g{u} un manque de donn\'ees pour commencer les calculs.

\par
Afin de constituer une condition initiale utilisable \g{a} partir des donn\'ees du jour 0, j'ai compl\'et\'e les vecteurs contenant certaines variables %
- voir la figure pour CBF, SO2 et CMRO2 - avec des valeurs distribu\'ees suivant une loi gaussiene, reprenant, pour chaque variable, les moyenne et \'ecart-type des valeurs effectivement mesur\'ees. %
Ainsi, de nombreuses valeurs, certes virtuelles, se sont trouv\'ees en vis-\g{a} vis, constituant une base de donn\'ees plus touffue comme dans la deuxi\g{e}me figure \ref{sim_ini}.

%\par
%Les distributions 

\begin{figure}[!p]
%\begin{center}
\begin{tabular}{|c|c|}
\hline
\subfloat[Donn\'ee brute]{\includegraphics[height=7cm,width=0.4\linewidth]{../../images_rapport/19-J00-modele2_simBrut.pdf}}
&
\subfloat[Compl\'etion de CBF, SO2 et CMRO2]{\includegraphics[height=7cm,width=0.4\linewidth]{../../images_rapport/19-J00-modele2_simCom.pdf}}
\\
\hline
\end{tabular}
%\end{center}
\caption{Conditions initiales pour le rat num\'ero 19.
\\
Les pixels en r\'egime cytotoxique sont color\'es en rouge sombre, %
les autres, en bleu, sont dans un \'etat perturb\'e.
}
\label{sim_ini}
\end{figure}
\end{comment}

\subsubsection{Evolution \g{a} partir du jour 8.}

Si l'on choisit de d\'emarrer les simulations au jour 8, pour lequel on dispose de donn\'es mesur\'ees ou calcul\'ees \g{a} partir d'un examen IRM, %
on peut renforcer la deuxi\g{e}me hypoth\g{e}se en laissant, comme seules variables autres que les six \'etudi\'ees, les bruits des \'equations $\Tb$.

\par
En effet, 

\par
L'\'etape isch\'emique transitoire du mod\g{e}le 2 est \'egalement superflue, %
puisque l'\'evolution des pixels peut seulement transiter entre les \'etats l\'esion 1 d\'ebut, 1 asymptotique etc. %
On peut ainsi reprendre la version forte \ref{auto} de l'hypoth\g{e}se d'autonomie du mod\g{e}le.

\begin{modmerate}{Mod\g{e}le 1 bruit\'e, alias \og{} 3/2\fg{}}{\textbf{(H\arabic*)}}
\item L'\'evolution de l'\'etat d'un voxel n'est pas affect\'ee par l'\'etat des voxels voisins.
\item ${}_\mathbb{P}$ : pas de variable cach\'ee, seulement un bruit exog\g{e}ne.
\item L'\'evolution du syst\g{e}me est autonome.
%\item\label{bvf...}
\end{modmerate}

Le jeu d'hypoth\g{e}ses qui pr\'ec\g{e}de, plus fort que celui du mod\g{e}le 2 mais moins que celui du mod\g{e}le 1, m\'erite la d\'enomination \og{} 3demi\fg{}.

\etoile
Bien s\^ur, les simulations fond\'ees sur ce mod\g{e}le n'ont pas d'int\'er\^et pratique comparable \g{a} celles qui commencent au jour 0. %
%Toutefois, 


\begin{figure}[!p]
\begin{center}
%\begin{tabular}{|c|c|}
%\hline
%\subfloat[Compl\'etion de CBF, SO2 et CMRO2]
% etc.
%%% Ajouter une image recompos\'ee par gaussiennes ? %%%
%\\
%\hline
%\end{tabular}
\end{center}
\caption{Condition initiale pour le mod\g{e}le 3demi : donn\'ees brutes au jour 08. Tranches 9 et 10.}
\label{sim_ini_18}
\end{figure}









%\fbox{\rule[-0.4cm]{0cm}{1cm} Une boite créée avec \verb!fbox! et aérée avec \verb!rule!.}