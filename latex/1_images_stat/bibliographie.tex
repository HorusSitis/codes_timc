\section{Analyse d'images : AVC sur des cerveaux de rats}\label{sani}

\subsection{Recherches bibiographiques : avec bibcnrs}

\begin{description}
\item[Ischemia brain ADC] un article :
\begin{itemize}
\item \cite{mej} Analyse statistique pour \'etablir la pertinence de l'utilisation clinique de l'ADC chez l'humain. Utilise : analyse de variance, intervalles de confiance. Cet article cite \dots

\end{itemize}


\item[ischaemia rat stepwise model selection] : Rien qui concerne la mod\'elisation statistique avec les r\'egresseurs ADC, BVf etc.
\item[ischaemia rat brain adc cbf] Selon \cite{leeliu}, le rapport ADC/CBF permet de pr\'edire la susceptibilit\'e du tissu c\'er\'ebral %
\g{a} une \'eveltuelle isch\'emie.
\item[ischaemia rat brain imaging] : rien d'int\'eressant pour nous.
\end{description}

\subsection{Recherche dans le Journal of Cerebral Blood Flow and Metabolism}

Des r\'esultats de recherche sur la biologie du cerveau, notamment chez l\^etre humain, sont publi\'es dans ce journal, %
ainsi que des applications pour l'imagerie m\'edicale. %
La ligne \'editoriale semble \^etre la recherche fondamentale en biologie et en imagerie m\'edicale, %
plut\^ot que celle d'applications cliniques.

\subsubsection{A Model for the Coupling between Cerebral Blood Flow and Oxygen Metabolism during Neural Stimulation \cite{bufr97}}

Les auteurs :
\begin{description}
\item[Lawrence Frank] : Universit\'e de Californie, Sans Diego. Dirige le Center for Scientific Computation in imaging, qui publie depuis 2009 %
et effectue des recherches concernant l'imagerie m\'edicale tridimensionnelle entre autres. %
Dans des articles r\'ecents -2013 \g{a} 2015, L. R. Frank explore le fonctionnement de la mati\g{e}re blanche, dont il interpr\g{e}te les cons\'equences sur la cognition et la psychologie, %
\g{a} la lumi\g{e}tre -si je puis dire- des techniques qu'il d\'eveloppe concernant l'imagerie m\'edicale. %
Il publie ainsi dans le Journal of Head Trauma Rehabilitation et Psychology Addict Behavior.
\item[Richard Buxton] : professor of Radiology, University of San Diego, Department of Radiology. Recievd his PhD in Physics from M.I.T in 1980. %
Post doc in biomedical imaging (P.E.T and M.R.I) at the Massachusetts General Hospital. Author of the book : Introduction to fMRI.
\end{description}

\ligneinter

C'est un article de mod\'elisation. Les auteurs \'etudient la relation entre le taux d'extraction d'$O_2$(OEF) dans le milieu intraenc\'ephalique, %
et le flux sanguin c\'er\'ebral (CBf), en vue d'applications \g{a} l'IRM fonctionnelle.

\par
A l'aide d'un mod\g{e}le simple, qui demande la r\'esolution d'une \'equation diff\'erentielle ordinaire et une \'equation int\'egrale, %
ils expriment (OEF) comme une fonction non lin\'eaire de (CBf), %
puis \'etendent qualitativement leur r\'esultat \g{a} des mod\g{e}les plus complexes, %
prenant en compte par exemple l'irriguation des vaisseaux capillaires. %
Ils utilisent pour cela des r\'esolutions num\'eriques.
\begin{enumerate}[label=(H\arabic*)]
\item Irriguation
\item $\epsilon=1$
\item RQS pour l'\'echange d'$O_2$ entre le plasma et le sang
\item Chaque mol\'ecule d'$O_2$ du plasma a la m\^eme probabilit\'e $k$ d'\^etre m\'etabolis\'ee.
\end{enumerate}

\par
Ensuite, les auteurs calculent la sensibilit\'e de l'IRM au flux (CBf) \g{a} l'aide de la relation quantitative obtenue avec leur mod\g{e}le simplifi\'e. %
Ils utilisent une relation polyn\^omiale entre le volume vasculaire c\'er\'ebral $V$ et (CBf) \'etablie par %\cite{}grubb
et d\'efendent la th\g{e}se du couplage, \textit{via} leur mod\g{e}le simplifi\'e, entre (CBf) et (OEF), le dernier \'etant proportionnel \g{a} la CMR$O_2$.

\par
Enfin, les auteurs commentent les hypoth\g{e}ses principales du mod\g{e}le simplifi\'e :
\begin{enumerate}
\item La constante et totale irriguation des vaisseaux capillaires. %
Ils mentionnent d'ailleurs que la question de l'influence de l'activit\'e nerveuse sur le recrutement des capillaires n'est pas \'elucid\'ee -en 1997.
\item La constance de la proportion, proche de $1$, du dioxyg\g{e}ne du milieu c\'er\'ebral qui est m\'etabolis\'e. %
La variable d'ajustement entre les CMR$O_2$ et CMR${}_{\text{glc}}$, dont la diff\'erence explique la relation entre (CBf) et (OEF) est donc le taux de m\'etabolisation a\'erobie du glucose.
\end{enumerate}

\subsubsection{Articles r\'ecents du m\^eme journal}



\subsection{Automated Segmentation and Shape Characterization of Volumetric Data \cite{frsph}}

Cet article pr\'esente une m\'ethode de calcul num\'erique, justifi\'ee sur la plan math\'ematique, %
pour l'imagerie m\'edicale tridimensionnelle, plus efficace que celles d\'ej\g{a} connnues et impl\'ement\'ees.

\par
Ainsi, les auteurs V. L. Galinsky et L. R. Frank proposent la m\'ethode dite SWD, am\'elioration d'une autre m\'ethode dite \og{} SPHARM\fg{}, %
fond\'ee sur la d\'ecomposition dans une base hilbertienne de fonctions  repr\'esentant des ondes sph\'eriques.

\ligneinter
Plus pr\'ecis\'ement, concernant le mod\g{e}le math\'ematique des ondes sph\'eriques :


\subsection{Analyse d'images}% avec R}

%\subsubsection{R}
%\input{4_rats.tex}

Concernant les coupes de cerveaux de rats du GIN :

\begin{enumerate}[label=(Objectif\arabic*)]
\item Etudier les corr\'elations, et \'etablir une hi\'erarchie entre les grandeurs ADC, BVf, etc mesur\'ees directement par IRM ;
\item D\'elimiter, manuellement ou syst\'ematiquement, les zones isch\'emi\'ees.
\end{enumerate}

\subsubsection{Outils statistiques pour la reconnaissance d'objets : Objectif num\'ero $2$}

M\'ethodologie : clustering avec R, \'eventuellement interfac\'e avec un langage de bas niveau pour augmenter les performances, %
segmentation avec ImageJ, utilisation de Matlab sur des images en niveaux de gris cod\'ees sous forme matricielle.

\etoile
\cite{dol_cmi_16} propose, dans le cadre du diagnostic du cancer du cerveau chez l'Homme, une m\'ethode de lab\'elisation des voxels d'une image tridimensionnelle %
correspondant \g{a} une tumeur. %
L'utilisation de machines \g{a} vecteurs de support (SVM) est pr\'econis\'ee, et des composantes comme la position spatiale du voxel, %
les niveaux de gris relatifs de ses plus proches voisins ou une transformation utilisant une distance g\'eod\'esique p\'enalis\'ee par les gradients de niveaux de gris, %
-voir \cite{wei_geo_08} pour plus de d\'etails sur le mod\g{e}le de graphe pond\'er\'e ; sont retenues pour les vecteurs de repr\'esentation.

