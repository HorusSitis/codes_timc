\documentclass[a4paper,10pt]{article}
\usepackage[utf8]{inputenc}

%\documentclass[a4paper,12pt]{article}
%\usepackage[utf8]{inputenc}

\usepackage[top=1cm,bottom=1.2cm,left=1.55cm,right=1.55cm]{geometry}% On peut changer en cours de route avec la commande \newgeometry

\usepackage[utf8]{inputenc}
\usepackage[english,french]{babel}
\usepackage{amsmath,amsfonts,amssymb,graphicx}
%\usepackage{amsthm}
\usepackage{framed}
\usepackage[amsthm,thmmarks,framed]{ntheorem}
\usepackage[dvipsnames]{pstricks}
%\usepackage{pstricks-add,pst-plot,pst-node}
\usepackage{pstricks,pst-plot,pst-text,pst-tree}%,pst-eps,pst-fill,pst-node,pst-math,pstricks-add,pst-xkey}
\usepackage{epic,eepic}
% Versions propos\'ees par Frédéric Junier
%\usepackage[inline]{asymptote}

\usepackage{placeins}
\usepackage[lofdepth,lotdepth]{subfig}
% Entre autres, pour sauter des lignes dans un titre de figure.
\usepackage{caption}%[hang,small]


%\usepackage{color} %(black, white, red, green, blue, yellow, magenta et cyan)
\usepackage[dvipsnames]{xcolor}
\usepackage{colortbl}

\usepackage{verbatim}

\usepackage{enumitem}
\frenchbsetup{StandardLists=true}

\usepackage[all]{xy}

\usepackage{multicol}				%%%	Jusqu' \`a nouvel ordre pas de multicol, pour cause de compatibilit\'e avec des fonctions graphiques de TeX.
%\begin{multicols}[titre]{nb colonnes}
\setlength{\columnseprule}{0.25pt}

\usepackage{parskip}
\setlength{\parindent}{0cm}

\usepackage{float}
% Pour des figures non flottantes par exemple : placer une figure où je veux.

%Je veux une mise en page pour les théorèmes lemmes preuves, définitions etc, suffisamment aérée. Est-il possible de commander un encadrement systématique ?

\newcommand{\g}[1]{\`#1}%\g{lettre}

\providecommand{\abs}[1]{\lvert#1\rvert}

%Pour tout le m\'emoire !



%Annotations d'un m\'emoire en cours de r\'edaction :
\newcommand{\tr}{\textbf{\textcolor{red}{D\'emonstration \g{a} trouver }}}
\newcommand{\re}{\textbf{\textcolor{NavyBlue}{D\'emonstration \g{a} recopier }}}
\newcommand{\es}{\textbf{\textcolor{OliveGreen}{Esquisse de d\'emonstration }}}

%Pour les formules alg\'ebriques casse-pieds :


%Pour les formules concernant les \dots :


%Autres formules pour le m\'emoire :
\newcommand{\tc}[1]{\text{#1}}


\newcommand*{\etoile}
{
\begin{center}
\hspace{1pt}\par
*\hspace{5pt}*\hspace{5pt}*
\end{center}
}


\newcommand*{\ligneinter}
{
\begin{center}
\vspace{2pt}
\hfill\rule{0.5\linewidth}{0.1pt}\hfill\null
\end{center}
\vspace{7pt}
}

{
\theoremstyle{break}
\theoremprework{\vspace{0.2cm}\begin{minipage}{\textwidth}} %Pris en compte uniquement pour le premier newtheorem
\theorempostwork{\ligneinter\end{minipage}} %Même chose
\theoremheaderfont{\scshape}
\theorembodyfont{\itshape}
\theoremseparator{ :\newline\vspace{0.2cm}}
\newtheorem{defi}{D\'efinition}%[section]
%\newtheorem{prop}{Proposition}[section]
%\newtheorem{lemm}{Lemme}[section]
} 

{%
\theoremstyle{break}
\theoremprework{\vspace{0.2cm}\begin{minipage}{\textwidth}} %Pris en compte uniquement pour le premier newtheorem
\theorempostwork{\ligneinter\end{minipage}} %Même chose
\theoremheaderfont{\bfseries}
\theorembodyfont{\itshape}
\theoremseparator{ :\newline\vspace{0.2cm}}
%\newtheorem{lemm}{Lemme}[section]
\newtheorem{prop}{Proposition}[section]
}

{%
\theoremstyle{break}
\theoremprework{\vspace{0.2cm}\begin{minipage}{\textwidth}} %Pris en compte uniquement pour le premier newtheorem
\theorempostwork{\etoile\end{minipage}} %Même chose
\theoremheaderfont{\bfseries}
\theorembodyfont{\itshape}
\theoremseparator{ :\newline\vspace{0.2cm}}
%\newtheorem{lemm}{Lemme}[section]
\newtheorem{pref}{Proposition-d\'efinition}[section]
}

{%
\theoremstyle{break}
%\theoremprework{\begin{tabular}{|p{\textwidth}|}\hline}
%\theorempostwork{\\ \hline\end{tabular}}
\theoremheaderfont{\scshape}
\theorembodyfont{\normalfont}
\theoremseparator{ :\newline\vspace{0.2cm}}
%\newtheorem{lemm}{Lemme}[section]
\newframedtheorem{theo}{Théor\`eme}[section]
}

{%
\theoremstyle{break}
\theoremprework{\vspace{0.2cm}\begin{minipage}{\textwidth}}
\theorempostwork{\ligneinter\end{minipage}}
\theoremheaderfont{\scshape}
\theorembodyfont{\itshape}
\theoremseparator{ :\newline\vspace{0.2cm}}
\newtheorem{lemm}{Lemme}[theo]
}
  
{%
\theoremstyle{break}
%\theoremprework{\begin{tabular}{|p{\textwidth}|}\hline}
%\theorempostwork{\\ \hline\end{tabular}}
\theoremheaderfont{\scshape}
\theorembodyfont{\itshape}
\theoremseparator{ :\newline\vspace{0.2cm}}
%\newtheorem{lemm}{Lemme}[section]
\newframedtheorem{coro}{Corollaire}[theo]
}



{
\theoremstyle{break}
\theoremprework{\vspace{0.5cm}}
\theorempostwork{\vspace{0.5cm}\ligneinter}
\theoremheaderfont{\scshape}
\theorembodyfont{\normalfont\small}
\theoremseparator{ :\newline\vspace{0.2cm}}
\newtheorem{exem}{Exemple}[section]
} 

{
\theoremstyle{break}
\theoremprework{\vspace{0.5cm}\begin{minipage}{\textwidth}}
\theorempostwork{\end{minipage}\ligneinter}
\theoremheaderfont{\scshape}
\theorembodyfont{\small}
\theoremseparator{ :\newline\vspace{0.2cm}}
\newtheorem{rema}{Remarque}[section]
} 

%%% Quelques environnements bien comodes %%%

\newenvironment{modmerate}[2]%
  {%
  \renewcommand{\arraystretch}{2.1}
  \begin{tabular}{|c|}
  \hline
  %\hrule%[0.2cm]{1mm}{\linewidth}
  %\paragraph{#1}
  \textbf{#1}
  \\
  \hline
  %\hrule%[0.2cm]{1mm}{\linewidth}
  \begin{minipage}{0.8\linewidth}
  \begin{enumerate}[label=#2]
  }
  %
  %%% Début ... et fin. %%%
  %
  {\end{enumerate}
  \end{minipage}
  \\
  %\hrule%[0.2cm]{1mm}{\linewidth}
  \hline
  \end{tabular}
  }

%Pour de belles EDO.
\newcommand{\der}[1]{\frac{\text{d}}{\text{d}t}#1}

% Pour les types d'équations d'évolution, modèles 2 et 3demi.
\newcommand{\Tb}{\ \mathbf{Type 0_{\mathbb{P}}}}
\newcommand{\Ta}{\ \mathbf{Type 1}}
\newcommand{\Td}{\ \mathbf{Type 2}}

%Pour stocker des valeurs enumi par exemple, si l'on quitte provisoirement l'environnement eummerate.
\newcounter{stock}

%\newcommand{\Tm}{T1}

%\setcounter{section}{-1}
%\renewcommand{\thesection}{\Roman{section}}
%\renewcommand{\thesubsection}{\Roman{section}-\Alph{subsection}}


%%%%%%%%%%%%%%%%% Préambule commun aux papiers du stage %%%%%%%%%%%%%%%%%



%opening
\title{Stage d'initiation \g{a} la recherche en math\'ematiques appliqu\'ees : pr\'evisions}
\author{Antoine Moreau}

\begin{document}

\selectlanguage{french}

\maketitle

\begin{abstract}
Voici le mini-rapport demand\'e par Anne-Laure Foug\g{e}res, responsable de la formation de master \og{} maths en action\fg{}, %
pour le d\'ebut du stage de fin d'\'etudes. Il donne, dans les grandes lignes, les pr\'evisions pour le d\'eroulement de ce stage.
\end{abstract}

\emph{%
Le stage que je vais effectuer au laboratoire TIMC-IMAG (Techniques de l'Ing\'enierie M\'edicale et de la Complexit\'e - Informatique, Math\'ematiques et Applications, Grenoble) %
concerne l'imagerie m\'edicale : \g{a} l'aide de manipulations sur des cerveaux de rat, on cherche \g{a} optimiser, %
par des moyens statistiques et num\'eriques, la pr\'ecision et notre compr\'ehension des images obtenues par IRM, une technique non invasive et tr\g{e}s r\'epandue pour le diagnostic en m\'edecine.%
}

\section{Objectifs du stage}

Dans ce stage, on s'int\'eresse aux cons\'equences d'une isch\'emie, c'est-\g{a}-dire un d\'efaut d'oxyg\'enation art\'erielle, %
dans le tissu c\'er\'ebral. Un tel ph\'enom\g{e}ne peut \^etre caus\'e, par exemple, par l'apparition d'un caillot sanguin dans le r\'eseau art\'eriel ou une rupture d'an\'evrisme.

\par
Ce ph\'enom\g{e}ne est bien plus facile \g{a} mod\'eliser qu'une croissance de tumeur par exemple, ce qui n\'ecessite la prise en compte de la structure de la tumeur, l'angiogen\g{e}se qu'elle provoque etc. %
Cela motive l'\'etude des cons\'equences anatomiques de l'isch\'emie dans le tissu c\'er\'ebral, qui doit servir d'\'etalon \g{a} des recherches ult\'erieures sur l'anatomie et l'imagerie c\'er\'ebrales.

\ligneinter
Pendant ces quelques mois on s'int\'eresse \g{a} l'\'evolution spatio-temporelle de l'hypoxie dans des cerveaux de rat. Le stage s'articule sur trois axes :
\begin{enumerate}
\item A l'aide d'images bidimensionnelles (coupes) d'un petit nombre de cerveaux de rat observ\'es sur plusieurs jours, %
je dois proposer un mod\g{e}le statistique simple permettant d'identifier les relations entre les variables ou param\g{e}tres anatomiques et pĥysiologiques du cerveau de rat %
(volume sanguin c\'er\'ebral ou perm\'eabilit\'e des vaisseaux \g{a} l'oxyg\g{e}ne par exemple).
%
\item Apr\g{e}s une recherche bibliographique sur des mod\g{e}les biologiques, et \'eventuellement \g{a} la lumi\g{e}re de l'\'etude statistique qui pr\'ec\g{e}de, %
je devrai estimer la perm\'eabilit\'e du dioxyg\g{e}ne \g{a} travers la barri\g{e}re h\'ematoenc\'ephalique.
%
\item Je dois aussi contribuer \g{a} l'\'ecriture d'un code qui permettra de simuler l'\'evolution spatio-temporelle de la pression partielle en dioxyg\g{e}ne %
-Pt$O_2$- dans le tissu c\'er\'ebral de rats.
\end{enumerate}

\newpage
\section{Encadrement}

J'effectue ce stage, pluridisciplinaire, sous la responsabilit\'e d'Ang\'elique St\'ephanou, CR CNRS habilit\'ee, en biologie computationnelle.

\par
Je rapporte \g{a} mon encadrante le d\'eroulement de mon travail plusieurs fois par semaine, une fois par jour dans la mesure du possible -je travaille en temps partiel pendant le mois de mai.

\par
Je devrai correspondre avec Emmanuel Barbier et Benjamin Lemasson, biologistes du GIN (Grenoble Institute of Neurosciences), associ\'es au projet d'augmentation virtuelle de l'IRM.

\par
Je collabore, par ailleurs, avec Nicolas Glade (Ma\^itre de conf\'erences \g{a} l'Universit\'e Joseph Fourier, CNU 27), membre de TIMC-IMAG. %
Il prend notamment la responsabilit\'e d'organiser le travail de codage des chercheurs, afin que nous disposions d'une architecture de code propre, %
c'est-\g{a}-dire lisible et facilement modifiable par tous les utilisateurs, pour nos simulations.

\etoile
En ce qui concerne les math\'ematiques appliqu\'ees, je n'ai pas d'interlocuteur privil\'egi\'e au sein de l'\'equipe avec laquelle je travaille, %
dont plusieurs membres sont Ma\^itres de Conf\'erences en section 26.


\bibliography{Biblio}
\bibliographystyle{alpha}

\end{document}
