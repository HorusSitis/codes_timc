\section{Introduction}

\subsection{Contexte du stage}

%\subsubsection{Projet IRMA}

\begin{frame}
\begin{block}{Objectifs : projet IRMA}
%\pause
\begin{enumerate}
\item<+-> Effectuer des pr\'edictions sur des variables qui ne sont pas directement mesurables ;
\item<+-> Am\'eliorer la r\'esolution spatiotemprelle de l'IRM.
\end{enumerate}
\end{block}
\includegraphics[width=0.8\linewidth,height=3.8cm]{../../images_rapport/ech_voxel.png}<+->
\end{frame}

\begin{frame}
\begin{block}{IRMA : premi\g{e}res \'etapes}
\begin{enumerate}
\item<+-> L'acquisition d'images IRM multiparam\'etriques sur des cerveaux de rats ;
\item<+-> La construction d'un mod\g{e}le dynamique 
\end{enumerate}
\end{block}
\end{frame}

%\subsubsection{}

\begin{frame}{Objectifs : AVC isch\'emique chez le rat}
\begin{itemize}
\item<+-> Pr\'edire l'\'evolution temporelle d'un tissu isch\'emi\'e ;
\item<+-> Echelle spatiale : le voxel ;
\item<+-> La d\'emarche est la m\^eme que celle d\'etaill\'ee pr\'ec\'edemment.
\end{itemize}
\end{frame}




\subsection{L'imagerie par r\'esonnance magn\'etique}



\begin{frame}
\frametitle{R\'esonance magn\'etique nucl\'eaire}
\includegraphics[height=0.9\textheight]{../../images_rapport/rmn_ensc.png}
\end{frame}

\begin{frame}
\frametitle{Modalit\'es T1, T2}
\begin{itemize}
\item<+-> On mesure le temps caract\'eristique de retour \g{a} l'\'equilible de l'aimantation $M$.
\item<+-> \[M_z(t)=M_z(0)\left(1-e^{-\frac{t}{T_1}}\right)\text{ et }M_{xy}(t)=M_{xy}(0)\times e^{-\frac{t}{T_2}}\]
\item<+-> Deux modalit\'es :
\begin{itemize}
\item<+-> $T1$, pour la composante longitudinale $M_z$ ;
\item<+-> $T2$, pour la composante transversale $M_{xy}$.
\end{itemize}
\end{itemize}
\end{frame}

\subsubsection{IRM de diffusion}
\begin{frame}
\frametitle{Coefficient de diffusion apparent (ADC)}
\begin{block}{Diffusion des mol\'ecules d'eau}
\begin{itemize}
\item<+-> Premi\g{e}re loi de Fick : $\vec{j}=-D\nabla n$
\item<+-> Valable en milieu homog\g{e}ne et isotrope.
\end{itemize}
\end{block}

\begin{block}{Fluctuations de $\vec{B}_0$}<+->
\begin{itemize}
\item<+-> Le signal IRM est sensible \g{a} la mobilit\'e des mol\'ecules d'eau ;
\item<+-> $S=S_0\exp\left(-bD\right)$
\item<+-> $b$ d\'epend de l'acquisition du signal.
\end{itemize}
\end{block}

\begin{block}{Milieu non homog\g{e}ne et isotrope}<+->
\begin{itemize}
\item<+-> $S=S_0\exp\left(-bADC\right)$
\item<+-> Unit\'e de mesure : mm${}^2$.s${}^{-1}$.
\end{itemize}
\end{block}
\end{frame}

\begin{frame}
\begin{block}{BVf : agent de contraste}
\begin{itemize}
\item<+-> Technique semi-invasive ;
\item<+-> Mesure de la fraction volumique en oxyg\g{e}ne (BVf).
%\item<+-> $BVf=\frac{3}{4\pi}\frac{\Delta R_2^{\ast}}{\gamma \Delta_{\chi_0}B_0}$.
\end{itemize}
\end{block}

\begin{block}{SO2 et l'effet BOLD}<+->
\begin{itemize}
\item<+-> Oxy-h\'emoglogine versus d\'esoxy-h\'emoglobine ;
\item<+-> Saturation tissulaire en diOxyg\g{e}ne SO2.
%\item<+-> \[s(t)=s_0\exp\left(-\frac{1}{T_2}t-BVf\gamma\frac{4}{3}\pi\Delta_{\chi_0}H_{ct}(1-SO2)B_0t\right)\]
\end{itemize}
\end{block}
\end{frame}

\begin{frame}
\frametitle{Autres techniques}
\begin{block}{Quixotic pour la CMRO2}<+->
Les auteurs de \cite{quixotic} (2011) proposent une m\'ethode (QUIXOTIC) pour mesurer la consommation c\'er\'ebrale m\'etabolique en oxyg\g{e}ne (CMRO2).
\end{block}

\begin{block}{CBF}<+->
\begin{itemize}
\item<+-> 2014 : modalit\'e IRM pur mesurer le d\'ebit sanguin c\'er\'ebral (CBF).
\item<+-> Unit\'e de mesure : ml.min${}^{-1}$.(100g)${}^{-1}$.
\end{itemize}
\end{block}
\end{frame}


