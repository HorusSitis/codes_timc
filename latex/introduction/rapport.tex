\section{Introduction}

\subsection{Contexte : stage au TIMC}

Ce travail a \'et\'e r\'ealis\'e pendant mon stage de master 2, qui s'est d\'eroul\'e au laboratoire TIMC-IMAG %
(Techniques de l'Ingénierie M\'edicale et de la Complexit\'e - Informatique, Mathématiques et Applications, Grenoble), %
sous la direction d'Angélique Stéphanou (CNRS). 

\par
Il s'inscrit dans le projet Imagerie par Résonnance Magn\'etique Augment\'ee (IRMA) qui vise \g{a} effectuer des pr\'edictions, %
\g{a} partir d'images IRM, sur des param\g{e}tres qui ne sont pas directement mesurables comme la pression art\'erielle en dioxyg\g{e}ne $P_aO_2$ %
ou avec une meilleure r\'esolution spatiotemporelle que celle des images brutes.

\par
Les premi\g{e}res \'etapes du projet comprennent :
\begin{enumerate}
\item L'acquisition d'images IRM multiparam\'etriques sur des cerveaux de rats ;
\item La construction d'un mod\g{e}le dynamique 
\end{enumerate}

Un tel mod\g{e}le doit \^etre capable de relier des m\'ecanismes physiologiques exsitant \g{a} l'\'echelle de la cellule nerveuse, %
et les observations r\'ealis\'ees \g{a} l'\'echelle du voxel.







\subsection{Imagerie par r\'esonnance magn\'etique}%State of the art, pour cette question plus générale que l'objet de ce stage.

L'imagerie par r\'esonnance magn\'etique consiste \g{a} placer un patient dans un champ magn\'etique uniforme intense. %
Typiquement, un appareil utilis\'e en clinique g\'en\g{e}re un champ de 1,5 Tesla : %
\g{a} titre de comparaison, le champ magn\'etique terrestre a une intensit\'e de 47 microTesla en France.

\par
Cette technique d'imagerie pr\'esente avantage de ne pas utiliser de radiations ionisantes, %
ce n'est pas le cas de la tomographie par \'emission de positrons par exemple. %
En cons\'equence, plusieurs examens IRM peuvent \^etre pratiqu\'es sur un m\^eme patient, sans effets secondaires.

\subsection{Principales modalit\'es}

Ce sont les diff\'erentes sortes d'examens r\'ealisables avec un appareil IRM.

\par
Elles permettent de mesurer, ou de calculer, les valeurs prises par diff\'erentes variables : %
coefficient de diffusion apparent de l'eau, saturation en oxyg\g{e}ne \dots %
d\'efinies sur toute l'\'etendue du tissu c\'er\'ebral.

\subsubsection{Relaxations T1, T2}

Le corps humain, et celui des autres animaux, contient de nombreux atomes d'hyrog\g{e}ne. %
Or le noyau d'un atome d'hydrog\g{e}ne, un proton, est de spin non nul : aurement dit, il est aimant\'e.

\par
En pr\'esence d'un champ magn\'etique intense et uniforme $\vec{B}_0$, les spins des atomes d'hydrog\g{e}ne s'orentent dans la direction de ce champ.

\par
Un champ $\vec{B}_1$, tournant et perpendiculaire \g{a} $\vec{B}_0$ est ensuite appliqu\'e \g{a} ces protons. %



\subsubsection{L'IRM de diffusion}

La mobilit\'e des mol\'ecules d'eau dans un tissu d\'epend ed la structure de celui-ci. %
Ainsi, les les parois des cellules dans un tissu sain freinent la diffusion des mol\'ecules d'eau. %
Au contraire, celles-ci sont libres de leurs moouvements au voisinage de cellules n\'ecros\'ees : %
une estimation de la diffusivit\'edes mol\'ecules d'eau permet, par exemple, de localiser un oed\g{e}me cytotoxique caus\'e par un AVC isch\'emique.3

\par
Toutefois, les structures physiologiques responsables de la mobilit\'e des mol\'ecules d'eau ne sont pas directement observables, %
\g{a} cause de leur taille -\'echelle de la cellule. %
Une cartographie du coefficient de diffusion apparent (ADC) des mol\'ecules d'eau peut toutefois \^etre calcul\'ee \g{a} partir de plusieurs images IRM du m\^eme tissu.

\par
Cette m\'ethode de construction d'images issues d'un examen IRM est appel\'ee IRM de diffusion.

\subsubsection{Agents de contraste}

Les performances de l'IRM peuvent \^etre augment\'ees par l'injection, en intraveineuse, d'un agent de contraste.

\par
De tels agents de contraste permettent par exemple de visualiser, sur des images irm, %
le pourcentage du volume occup\'e par le sang dans un tissu (Blood volume fraction ou BVf).

\par
Cette grandeur est donn\'ee par la formule \cite{Lem_PHD_10} :
\begin{equation}
BVf=\frac{3}{4\pi}\frac{\Delta R_2^{\ast}}{\gamma \Delta_{\chi_0}B_0}
\label{bvf_gd}
\end{equation}

o\g{u} $R_2^{\ast}$ est la diff\'erence des valeurs $\frac{1}{T_2^{\ast}}$ prises avant puis apr\g{e}s l'injection de l'agent de contraste, %
et $\Delta_{\chi_0}$ la diff\'erence de susceptibilit\'e magn\'etique intra- et extravasculaire induite par l'injection de l'agent de contraste.

\subsubsection{Effet BOLD}

L'h\'emoglobine agit parfois comme un agent de contraste naturel : c'est l'effet BOLD (Blood oxygen Level Dependent).

\par
Plus pr\'ecis\'ement, c'est la diff\'rence de susceptibilit\'e magn\'etique entre l'h\'emoglobine satur\'ee en oxyg\g{e}ne %
qui permet de mesurer directement la saturation tissulaire en dioxyg\g{e}ne d'un tissu vascularis\'e -$StO_2$.

\par


\par
Compte tenu de l'effet BOLD, les auteurs de \cite{christen2012l} relient la $SO_2$, la BVf et le signal IRM $s$ par la formule :
\begin{equation}
s(t)=\exp\left(-\frac{1}{T_2}t-BVf\gamma\frac{4}{3}\pi\Delta_{\chi_0}H_{ct}(1-SO2)B_0t\right)
\label{bold_so2}
\end{equation}
$H_{ct}$ est la concentration d'h\'ematocrites dans le r\'eseau microvasculaire.

\subsubsection{Quitoxic MRI pour la fraction d'extraction de dioxyg\g{e}ne (OEF) %
et le taux de dioxyg\g{e}ne m\'etabolis\'e (CMRO2).}



%QUITOXIC MRI pour la CMRO2 et l'OEF.





\subsubsection{Mesure du d\'ebit sanguin c\'ebral (Cerebral Blood Flow ou CBF)}






