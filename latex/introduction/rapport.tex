\section{Introduction}

\begin{comment}
Mon stage s'est d\'eroul\'e au laboratoire TIMC-IMAG (Techniques de l'Ingénierie M\'edicale et de la Complexit\'e - %
Informatique, Mathématiques et Applications, Grenoble), sous la direction d'Angélique Stéphanou (CNRS). %
Il s'inscrit dans le projet Imagerie par Résonnance Magn\'etique Augment\'ee (IRMA) qui vise \g{a} effectuer des pr\'edictions, %
\g{a} partir d'images IRM, sur des param\g{e}tres qui ne sont pas directement mesurables comme la pression art\'erielle en dioxyg\g{e}ne $P_aO_2$ %
ou avec une meilleure r\'esolution spatiotemporelle que celle des images brutes. %
Un autre aspect de l'augmentation virtuelle de l'imagerie par résonnance magnétique est la simulation, \g{a} partir d'une image initiale, %
de l'évolution de l'état d'un patient sur des \'echelles de temps plus ou moins longues. %
C'est l'objet de ce travail, qui doit permettre de retracer l'\'evolution, sur plusieurs semaines, de l'\'etat d'un cerveau de rat isch\'emi\'e.
\end{comment}

Augmentation virtuelle de l'IRM : un projet du TIMC et du GIN

IRM augment\'ee : mettre au point des programmes informatiques permettant d'interpr\'eter syst\'ematiquement des images IRM.
\begin{enumerate}
\item Effectuer des pr\'edictions : simuler l'\'evolution future de l'\'etat du tissu c\'er\'ebral chez un patient%
%\begin{itemize}

%\end{itemize}
\item Am\'eliorer 
\item Calculer des grandeurs, comme la pression partielle en $O_2$, qui en sont pas mesurables directement \g{a} partir des images.
\end{enumerate}






\subsection{Principe de l'IRM}%State of the art, pour cette question plus générale que l'objet de ce stage.


%IRM : historique ? Jusqu'\g{a} l'IRM multiparam\'etrique.
%Avantages et limitations, bri\g{e}vement.





%\subsection{Une utilisation remarquable : L'IRM fonctionnelle (1991)}% spécifique au cerveau







\subsection{Mesures effectu\'ees \g{a} l'aide d'un agent de contraste}

%Pratiquée au GIN






%QUITOXIC MRI pour la CMRO2 et l'OEF.

\subsection{L'effet BOLD}

%Effet BOLD




%SO2

\cite{christen2012l}


%\subsection{Une alternative : la tomographie par \'emission de positrons (TEP)}




%Sc\'ema : grille de pixels et de cellules.
%Automate cellulaire.