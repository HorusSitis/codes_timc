\section{Introduction}

\subsection{Contexte : stage au TIMC}

Ce travail a \'et\'e r\'ealis\'e pendant mon stage de master 2, qui s'est d\'eroul\'e au laboratoire TIMC-IMAG %
(Techniques de l'Ingénierie M\'edicale et de la Complexit\'e - Informatique, Mathématiques et Applications, Grenoble), %
sous la direction d'Angélique Stéphanou (CNRS). 

\par
Il s'inscrit dans le projet Imagerie par Résonnance Magn\'etique Augment\'ee (IRMA) qui vise \g{a} effectuer des pr\'edictions, %
\g{a} partir d'images IRM, sur des param\g{e}tres qui ne sont pas directement mesurables comme la pression art\'erielle en dioxyg\g{e}ne $P_aO_2$ %
ou avec une meilleure r\'esolution spatiotemporelle que celle des images brutes.

\par
Les premi\g{e}res \'etapes du projet comprennent :
\begin{enumerate}
\item L'acquisition d'images IRM multiparam\'etriques sur des cerveaux de rats ;
\item La construction d'un mod\g{e}le dynamique 
\end{enumerate}

Un tel mod\g{e}le doit \^etre capable de relier des m\'ecanismes physiologiques exsitant \g{a} l'\'echelle de la cellule nerveuse, %
et les observations r\'ealis\'ees \g{a} l'\'echelle du voxel.







\subsection{Imagerie par r\'esonnance magn\'etique}%State of the art, pour cette question plus générale que l'objet de ce stage.

L'imagerie par r\'esonnance magn\'etique consiste \g{a} placer un patient dans un champ magn\'etique intense : %
un appareil classiqu




\subsection{Principales modalit\'es}

Ce sont les diff\'erentes sortes d'examens r\'ealisables avec un appareil IRM.

\par
Elles permettent de mesurer, ou de calculer, les valeurs prises par diff\'erentes variables : %
coefficient de diffusion apparent de l'eau, saturation en oxyg\g{e}ne \dots %
d\'efinies sur toute l'\'etendue du tissu c\'er\'ebral.

\subsubsection{Relaxations T1, T2}




\subsubsection{L'IRM de diffusion}







\subsubsection{Agents de contraste}




\subsubsection{Effet BOLD}





%QUITOXIC MRI pour la CMRO2 et l'OEF.








\cite{christen2012l}%SO2



