\section{Introduction}

\subsection{Contexte : stage au TIMC}

Ce travail a \'et\'e r\'ealis\'e pendant mon stage de master 2, qui s'est d\'eroul\'e au laboratoire TIMC-IMAG %
(Techniques de l'Ingénierie M\'edicale et de la Complexit\'e - Informatique, Mathématiques et Applications, Grenoble), %
sous la direction d'Angélique Stéphanou (CNRS). 

\par
Il s'inscrit dans le projet Imagerie par Résonance Magn\'etique Augment\'ee (IRMA) qui vise \g{a} effectuer des pr\'edictions, %
\g{a} partir d'images IRM, sur des param\g{e}tres qui ne sont pas directement mesurables comme la pression art\'erielle en dioxyg\g{e}ne $P_aO_2$ %
ou avec une meilleure r\'esolution spatiotemporelle que celle des images brutes.

\par
Les premi\g{e}res \'etapes du projet comprennent :
\begin{enumerate}
\item L'acquisition d'images IRM multiparam\'etriques sur des cerveaux de rats ;
\item La construction d'un mod\g{e}le dynamique 
\end{enumerate}

Un tel mod\g{e}le doit \^etre capable de relier des m\'ecanismes physiologiques existant \g{a} l'\'echelle de la cellule nerveuse, %
et les observations r\'ealis\'ees \g{a} l'\'echelle du voxel.

\par
Le travail pr\'esent\'e ici consiste \g{a} simuler l'\'evolution d'un tissu isch\'emi\'e : %
on \'etudie les suites d'un AVC isch\'emique sur un cerveau de rat. %
Pour atteindre notre objectif, nous utiliserons la m\^eme d\'emarche que celle d\'etaill\'ee pr\'ec\'edemment.

\subsection{Imagerie par r\'esonance magn\'etique}%State of the art, pour cette question plus générale que l'objet de ce stage.

L'imagerie par r\'esonance magn\'etique consiste \g{a} placer un patient dans un champ magn\'etique uniforme intense. %
Actuellement, un appareil utilis\'e en clinique g\'en\g{e}re un champ de 1,5 \g{a} 3 Tesla : %
\g{a} titre de comparaison, le champ magn\'etique terrestre a une intensit\'e de 47 microTesla en France.

\par
Cette technique d'imagerie pr\'esente l'avantage de ne pas utiliser de radiations ionisantes, %
ce n'est pas le cas de la tomographie par \'emission de positrons par exemple. %
En cons\'equence, plusieurs examens IRM peuvent \^etre pratiqu\'es sur un m\^eme patient, sans effets secondaires.

\subsection{Principales modalit\'es}

Ce sont les diff\'erents modes d'acquisition d'images r\'ealisables avec un appareil IRM.

\par
Ces modes permettent de calculer diff\'erentes variables : %
coefficient de diffusion apparent de l'eau, saturation en oxyg\g{e}ne \dots %
\'evalu\'ees sur toute l'\'etendue du tissu c\'er\'ebral.

\par
Dans ce document, le mot variable d\'esigne une grandeur physiolgique, qui prend une valeur num\'erique en voxel de l'espace.

\subsubsection{Relaxations T1, T2}

%Le corps humain, et celui des autres animaux, contient de nombreux atomes d'hyrog\g{e}ne. %
%Or le noyau d'un atome d'hydrog\g{e}ne, un proton, est de spin non nul : c'est l'analogue quantique d'une aimantation.

%\par
En pr\'esence d'un champ magn\'etique intense et uniforme $\vec{B}_0$, les spins des noyaux atomiques s'orientent dans la direction de ce champ.

\par
Un champ $\vec{B}_1$, tournant et perpendiculaire \g{a} $\vec{B}_0$ est ensuite appliqu\'e \g{a} ces protons. %
Les spins de ces noyaux atomiques atteignent ainsi un \'etat d'excitation -figure \ref{irm_b01}.

\begin{figure}[!b]
\includegraphics[width=0.8\linewidth, height=14cm]{../../images_rapport/rmn_ensc.png}
\caption{Trois \'etapes pour cr\'eer un signal RMN.
\\
culturesciences.chimie.ens.fr}
\label{irm_b01}
\end{figure}

\par
Le retour \g{a} un alignement sur le champ magn\'etique $\vec{B}_0$ g\'en\g{e}re un signal qui peut \^etre mesur\'e : %
la r\'esonance magn\'etique nucl\'eaire (RMN).

\par
C'est ce signal qui est utilis\'e en IRM. En pratique, les phases d'excitation-d\'esexcitation sont r\'ep\'et\'ees plusieurs fois. %
Les intervalles de temps (temps d'\'echo, temps de r\'ep\'etition) varient selon le ph\'enom\g{e}ne \g{a} observer.

\etoile

\newpage
Pendant la phase de retour \g{a} l'\'equilibre, les composantes transversale $M_{xy}$, %
et longitudinale de l'aimantation $M$ qui r\'esulte de l'action de $\vec{B}_0$ et $\vec{B}_1$ sur les spins de protons, %
sont donn\'ees par les formules suivantes :

\[M_z(t)=M_z(0)\left(1-e^{-\frac{t}{T_1}}\right)\text{ et }M_{xy}(t)=M_{xy}(0)\times e^{-\frac{t}{T_2}}\]

Les temps caract\'eristiques $T_1$ et $T_2$ d\'ependent de la r\'egion du tissu que l'on observe. %
Ce sont deux variables directement mesurables par IRM : %
il suffit de calculer la vitesse initiale avec laquelle les grandeurs $M_z$ et $M_{xy}$ d\'ecroissent.

\par
Les grandeurs $T_1^{\ast}$ et $T_2^{\ast}$, que l'on rencontrera par la suite, sont analogues \g{a} $T_1$ et $T_2$. %
Elles sont obtenues en changeant les modalit\'es de l'examen -\'echo, temps de r\'ep\'etition etc.

\etoile
Dans la suite de ce travail et dans la litt\'erature scientifique, %
les grandeurs $\frac{1}{T_1}$, \dots $\frac{1}{T_2^{\ast}}$ sont not\'ees $R_1$, \dots $R_2^{\ast}$.

\FloatBarrier
\subsubsection{L'IRM de diffusion}

La mobilit\'e des mol\'ecules d'eau dans un tissu d\'epend de la structure de celui-ci. %
Ainsi, les les parois des cellules dans un tissu sain freinent la diffusion des mol\'ecules d'eau. %
Au contraire, celles-ci sont libres de leurs mouvements au voisinage de cellules n\'ecros\'ees : %
une estimation de la diffusivit\'e des mol\'ecules d'eau permet, par exemple, de localiser un oed\g{e}me cytotoxique caus\'e par un AVC isch\'emique.

\par
Toutefois, les structures physiologiques responsables de la mobilit\'e des mol\'ecules d'eau ne sont pas directement observables. %, %
%\g{a} cause de leur taille -\'echelle de la cellule. %
Une cartographie du coefficient de diffusion apparent (ADC) des mol\'ecules d'eau peut toutefois \^etre calcul\'ee \g{a} partir de plusieurs images IRM du m\^eme tissu.

\par
Cette m\'ethode de construction d'images issues d'un examen IRM est appel\'ee IRM de diffusion.

\par
L'ADC, homog\g{e}ne au coefficient $\lambda$ de l'\'equation de diffusion issue de la loi de Fick : $\partial_t [H_2O]=\lambda \nabla^2 [H_20]$, %
s'exprime le plus souvent en mm${}^2$ par seconde.

\subsubsection{Agents de contraste}

Les performances de l'IRM peuvent \^etre augment\'ees par l'injection, en intraveineuse, d'un agent de contraste.

\par
La formule ci-dessous permet, de calculer la fraction volumique de sang dans un tissu c\'er\'ebral (blood volume fraction ou BVf). %
Cette variable, sans dimension, s'exprime en pourcent.
%La grandeur $\Delta_{\chi}$, diff\'erence de sisc $\Delta R_2^{\ast}$

\begin{equation}
BVf=\frac{3}{4\pi}\frac{\Delta R_2^{\ast}}{\gamma \Delta_{\chi_0}B_0}
\label{bvf_gd}
\end{equation}

o\g{u} $R_2^{\ast}$ est la diff\'erence des valeurs de $\frac{1}{T_2^{\ast}}$ prises avant puis apr\g{e}s l'injection de l'agent de contraste, %
et $\Delta_{\chi_0}$ la diff\'erence de susceptibilit\'e magn\'etique intra- et extravasculaire induite par l'injection de l'agent de contraste.

\subsubsection{Effet BOLD}

L'h\'emoglobine agit parfois comme un agent de contraste naturel : c'est l'effet BOLD (Blood oxygen Level Dependent).

\par
Plus pr\'ecis\'ement, l'h\'emoglobine satur\'ee en oxyg\g{e}ne -oxyh\'emoglobine- et celle qui n'est pas satur\'ee en oxyg\g{e}ne n'ont pas la m\^eme susceptibilit\'e magn\'etique. %
L'IRM permet ainsi de localiser des concentrations excessives en oxy- ou d\'esoxy-h\'emoglobine, %
c'est cette derni\g{e}re qui joue le r\^ole d'un agent de contraste.

\par
La saturation tissulaire en oxyg\g{e}ne peut ainsi \^etre mesur\'ee directement : %
elle est reli\'ee, dans \cite{christen2012l}, aux variables BVf et $T_2$ par l'\'equation suivante :

\begin{equation}
s(t)=\exp\left(-\frac{1}{T_2}t-BVf\gamma\frac{4}{3}\pi\Delta_{\chi_0}H_{ct}(1-SO2)B_0t\right)
\label{bold_so2}
\end{equation}
L'\'ematocrite $H_{ct}$ est la concentration d'h\'ematies dans le sang.

\par
Les valeurs de SO2 sont encore des pourcentages : la variable est sans dimension.

\subsubsection{Quixoticic pour la mesure de l'OEF (oxygen extraction fraction) %
et de la CMRO2 (cerebral metabolic rate of oxygen).}

Les auteurs de \cite{quixotic} proposent une technique, sous l'acronyme QUIXOTIC, permettant de mesurer successivement l'OEF et la CMRO2.

\par
Les diff\'erentes modalit\'es IRM utilis\'ees dans l'exp\'erience expliqu\'ee \g{a} la section m\'ethodes, %
incluent une mesure de la CMRO2.

\subsubsection{Mesure du d\'ebit sanguin c\'ebral (Cerebral Blood Flow ou CBF)}

Les auteurs de \cite{cbf_alz} ont pr\'esent\'e en 2014 une modalit\'e d'IRM qui permet de mesurer le d\'ebit sangin c\'er\'ebral (cerebral blood flow ou CBF). %
La mesure de CBF est, 

\par
La variable CBF, intensive, s'exprime le plus souvent en mL par minutes et par 100g.


