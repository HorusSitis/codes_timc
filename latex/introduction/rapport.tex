\section{Introduction}

\subsection{IRM, IRM augment\'ee}%State of the art, pour cette question plus générale que l'objet de ce stage.


%IRM : historique ? Jusqu'\g{a} l'IRM multiparam\'etrique.
%Avantages et limitations, bri\g{e}vement.

IRM augment\'ee : mettre au point des programmes informatiques permettant d'interpr\'eter syst\'ematiquement des images IRM.
\begin{enumerate}
\item Effectuer des pr\'edictions : simuler l'\'evolution future de l'\'etat du tissu c\'er\'ebral chez un patient%
%\begin{itemize}

%\end{itemize}
\item Am\'eliorer 
\item Calculer des grandeurs, comme la pression partielle en $O_2$, qui en sont pas mesurables directement \g{a} partir des images.
\end{enumerate}

%Sc\'ema : grille de pixels et de cellules.
%Automate cellulaire.

%Effet BOLD

%QUITOXIC MRI pout la CMRO2 et l'OEF.



\subsection{L'accident vasculaire c\'er\'ebral isch\'emique}


%Epidémiologie.

%Pénombre, reperfusion, neuroprotection, fenêtre thérapeutique.

%Modèle dynamique : reconstitution d'images acquises pendant plusieurs jours d'examen, prédictions.

%Principales sources :
% 1- Durukan_PBB_07 donné le 11 août pour la cascade ischémique et des traitements ;
% 2- pin_RNAR_99 dans le dossier parent biblio ;
% 3- Girard dans cours_biologie pour les nécroses éparses en zone de pénombre ischémique ;
% 4- ...




%%% Important : mentionner deux échelles de temps pour la cascade ischémique. %%%
%% Oedème vasogénique, intervention du système immunitaire. %%